\chapterAuthor{Ajahn Ñāṇarato}
\chapter{A prática da minhoca}
\tocChapterNote{Ajahn Ñāṇarato}

O venerável Ajahn Chah, um dos mais conhecidos professores do Budismo da
Tradição da Floresta da Tailândia, costumava dizer: ``A nossa prática é
como a minhoca''. O que significa isto?

No mundo moderno queremos obter resultados o mais rapidamente possível,
fazendo juízos acerca de quão eficientes as coisas são, quão bem"-feitas,
quão atraentes, etc. Existe uma constante pressão para se estar
actualizado com os últimos avanços e temer ser deixado para trás. Mas
será que temos realmente força e confiança em nós próprios? Não será que
estamos a perder a nossa confiança e integridade ainda que acreditemos
que estamos a controlar o ``nosso mundo''? Enquanto o chamado
desenvolvimento ao nível material é tão invasivo e amplamente
disseminado por toda a parte, o que é que tem vindo a acontecer
connosco, afinal o Ser mais importante no meio de tudo isto?

Temos que correr para obter resultados e alcançar o mesmo lugar que
todos as outras pessoas, da forma mais rápida possível; esta é uma
percepção que toda a gente tem. Mas poderemos assim manter o nosso
espaço interior para perceber a beleza, a dor e as possibilidades dos
outros? Estas questões deveriam ser levadas em consideração quando
sinceramente procuramos a paz, o viver pacificamente no meio da
diversidade, e dentro deste contexto penso que a paciência é uma das
qualidades chaves que necessitamos de reconhecer mais conscientemente.
No \emph{Pāṭimokkha} \emph{Ovāda}, o conjunto de versos que sumariza os
ensinamentos do Buddha, \emph{Khanti} (Paciência) é a qualidade por
excelência da prática religiosa. Muitas vezes vemos a paciência como
antiquada e parece ser o oposto de estarmos actualizados com o
desenvolvimento do mundo.

A meu ver a qualidade ou virtude da paciência é essencial para que
vivamos juntos pacificamente. Sabemos que ser impaciente muitas vezes
bloqueia ou até mesmo destrói esse processo precioso, quer na nossa vida
individual quer em sociedade. \emph{Khanti} leva"-nos a realizar
sabedoria e compaixão, prevenindo"-nos de cair em padrões egocêntricos.

Como é que somos quando estamos pacientes? Qual é o estado do nosso
corpo e da nossa mente? É simplesmente negativo ou passivo? Não será
esse, na realidade, o momento de nos nutrirmos e de retomarmos a nossa
própria força?

No passado mês de Novembro tive a oportunidade de viajar pelo Butão
durante dezasseis dias. Desde o início que notei que havia um forte
sentimento de calma, um certo ``à vontade'' -- esta expressão ressoava
particularmente em mim. Ainda que outros países possam ter cenários mais
espectaculares e vistas fantásticas que captem o olhar, o Butão
destaca"-se por oferecer este sentimento único de ``à vontade''.

O modo tradicional de construção e de vestuário, grandes áreas de
floresta virgem e templos com uma clara e pacífica dignidade \ldots{}
toda esta simplicidade mas firmeza deu"-me um toque de preenchimento. A
integridade auto"-suficiente e a confiança estão presentes naturalmente,
sem esforço. Quando contactei com este sabor de maravilhoso
contentamento de estilo de vida, senti como se fosse bem"-vindo na
atmosfera do ``à vontade''.

Depois de ter regressado à Grã -- Bretanha, um amigo perguntou"-me ``Como
era a pobreza?''. Fiz uma pequena pausa e retorqui ``Sim é um país
pobre''. Depois perguntei"-me porque tinha feito a pausa e percebi que a
questão da pobreza e da riqueza não é tão relevante ali como noutros
países. A pergunta, à qual estamos tão acostumados a fazer, é ela
própria uma reflexão da perspectiva através da qual olhamos o mundo.

É necessário que este ponto de vista material não seja a principal
experiência da nossa vida. Se não limitarmos a maneira de saborear e de
avaliar a vida, os aspectos esquecidos desta revelar"-se-ão por si
próprios. No Butão senti um profundo sentimento de contentamento o qual
não experienciamos facilmente na sociedade moderna. Quando estamos
envoltos nestes pensamentos apercebemo"-nos de que a maneira como nos
relacionamos com o mundo poderia ser bem mais ampla e mais rica do que
na limitada esfera do bem"-estar material. No Butão a riqueza vem do
contentamento e confiança na vida de cada um. Não temos de passar o
tempo a batalhar para provarmos a nós próprios que somos alguém e para
infinitamente obtermos bens. Pelo simples facto de poder tocar esta
possível perspectiva, senti um alívio.

``Esquecido'', escrevi. Esta perspectiva não é algo para se criar, mas
sim algo para parar, reconhecer e apreciar.

No Butão a vida deve ser bastante exigente em termos físicos. Muitas
vezes podemos avistar socalcos muito inclinados com arroz cultivado em
água, desde o nível mais baixo dos estreitos vales até quase ao topo das
colinas. Sem dúvida que requer uma enorme quantidade de paciência. Mas
aqui a nossa percepção de paciência poderá ser limitada de forma
negativa. Na verdade, quando olhamos para a vida no Butão, esta
paciência traz consigo o estado de bênção. O ritmo da vida é de
paciência e cada momento e cada encontro oferecem a oportunidade para
Sermos, totalmente. Questões, argumentos e pensamentos sobre a ideia de
paciência param e tudo é simplesmente aquilo que está a acontecer.
Qualidades como a confiança, força e contentamento são então vistas
juntamente com a paciência. Elas são inseparáveis. Penso que isto mostra
a maravilhosa qualidade e benefício da paciência. Esta experiência no
Butão foi como um relembrar do claro contraste entre aquilo a que
chamamos ``desenvolvimento'' e a força que o Butão demonstra de uma
forma sem qualquer tipo de insistência.

As palavras do Venerável Ajahn Chah ``a nossa prática é como a minhoca''
foram dirigidas aos monges que viviam com ele no mosteiro. Normalmente
as pessoas pensam na vida na floresta como algo repleto de paz e talvez
de experiências maravilhosas, mas na realidade é muitas vezes neste
local que as nossas tendências mais latentes poderão manifestar"-se até á
exaustão. Irritação, dores, aborrecimento e por aí fora - as lutas
interiores -- encontram"-se presentes.

Quando se vive num mosteiro há já algum tempo começa"-se a apreciar o
valor das palavras do Venerável Ajahn Chah. Não é uma questão de quão
espertos somos ou de quão promissora era a ideia que tínhamos de início.
Tentar ser como as minhocas não é ofensivo ou desencorajador, mas é uma
forma de reconhecermos as nossas tendências (tão fortemente enraizadas)
de buscar quaisquer resultados e recompensas, visíveis e imediatos. As
minhocas têm de viver a vida pacientemente, totalmente concentradas no
momento e no seu próprio esforço e movimento. Isto desafia a
aparentemente fascinante ideia de conquista pessoal, a qual é a força
que está por detrás do desenvolvimento, particularmente nesta era
moderna. Poderemos tornar"-nos minhocas? Imagine"-se a resistência, todo o
tipo de razões ou desculpas que podem surgir quando tentamos ser de
maneira diferente. É como se a importância da pessoa fosse negada.
Vivenciar o processo, humildemente rendermo"-nos a ele. Paciência não é
algo para ser pensado ou falado mas sim para Ser.

Ser um mendicante significa que se depende da generosidade alheia para
se ter os quatro requisitos, ou seja, abrigo, comida, roupa e
medicamentos. No início podemos achar que o fato de não podermos ter e
escolher as coisas que gostamos é demasiado difícil e restritivo, mas
depois aprendemos a familiarizarmo"-nos com estas condições e
eventualmente surge a alegria. O sentimento de gratidão é profundo.
Estar satisfeito não é um estado passivo mas sim um estado que nos
confere estabilidade, confiança e felicidade. Isto é, também, o oposto
daquele que normalmente estamos no mundo. Acho que estas qualidades de
contentamento e gratidão são impossíveis de realizar se não formos
pacientes.

Quando somos pacientes ganhamos força interior e, ainda mais importante,
somos levados a perceber que este mesmo momento é, por si só, pleno.
Nada lhe falta. Não perdemos o nosso compromisso para com o imediatismo
do momento presente e nem estamos na expectativa de que o verdadeiro
significado da vida se torne disponível algures no futuro. Quando
queremos mudar a nossa atitude da exigência para a aceitação, isto
torna"-se não só relevante no contexto monástico mas também no mundo em
geral.

Gostaríamos que o mundo vivesse de uma forma mais harmoniosa e o ponto
crucial de viragem para isso é: \emph{\textbf{ou}} simplesmente
continuamos com a atitude de obter aquilo que satisfaz os nossos
desejos, livrando"-nos daquilo que não se enquadra nos nossos gostos e
opiniões, sem plenamente apreciarmos o que está disponível à nossa
volta, \emph{\textbf{ou}} cada um de nós poder receber em si a variedade
e totalidade do mundo tal como ele é, através de, graciosa e
humildemente, reaprender a virtude da paciência. Preciso de acrescentar
que, a nível individual, isto significa receber diferentes tipos de
emoções tais como medo, ódio, mágoa, desejo, etc.

A nossa prática, ou na verdade eu diria, a nossa vida, é como a minhoca.
Concentrados no momento presente e sem ambições pelo que vem a seguir,
tudo o que podemos fazer é continuar a praticar e aceitar por completo o
que temos à nossa volta. Observar como estas palavras nos afectam --
este é o ponto de partida.

\bigskip

{\raggedleft\itshape
  Tradução de Appamādo Bhikkhu
\par}
