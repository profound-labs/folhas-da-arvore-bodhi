\chapterAuthor{Ajahn Sundara}
\chapter{Liberdade na restrição}
\tocChapterNote{Ajahn Sundara}

Quando o Buddha ensinou a Primeira Nobre Verdade disse que ter a
existência humana como refúgio é uma experiência insatisfatória. Se nos
apegamos a esta estrutura mortal sofreremos. Não obter o que desejamos é
doloroso - é muito fácil percebermos isto. Obter o que não queremos pode
ser também doloroso. Mas quando caminhamos um pouco mais seguindo as
pegadas do Buddha, até mesmo obter aquilo que se quer é doloroso! Isto é
o princípio do caminho do Despertar.

Quando percebemos que obter aquilo que queremos no mundo material também
já não nos satisfaz é quando começamos a amadurecer. Já não somos
crianças, à espera de encontrar a felicidade pela obtenção daquilo que
se quer, fugindo da dor.

Vivemos numa sociedade que venera a gratificação dos desejos. Mas muitos
de nós não estamos realmente interessados em desejos gratificantes,
porque intuitivamente, sabemos que não é disso que sobrevive a
existência humana.

Vem-me à memória quando, há já muitos anos, estava a tentar perceber o
que eu achava ser a Verdade. De alguma forma sabia que havia algo a
alcançar por detrás das minhas emoções e da minha mente pensante, algo
que transcendia este mundo do nascimento e da morte. Com o passar do
tempo, o desejo de viver uma vida que fosse verdadeira e real tornou-se
a coisa mais importante do mundo. Enquanto eu tentava harmonizar os meus
pensamentos, os meus sentimentos e as minhas aspirações e chegar a um
estado de paz, tornei-me consciente de que havia algo entre as minhas
aspirações e a minha mente. Parecia existir um enorme espaço entre elas
e isso era aquilo a que eu chamava de ``meu eu'', este corpo com os seus
sentidos físicos. Por essa altura não tinha ainda percebido que o
ensinamento budista apresenta os seres humanos com um sexto sentido, a
mente, a plataforma na qual os pensamentos podem surgir.

A mente e o corpo são um reservatório de energia e descobri que a minha
energia oscilava, dependendo da forma como eu os usava. A minha forma de
me relacionar com a vida e a minha compreensão da mesma pareciam também
ser dependentes da clareza da minha mente, e por sua vez, essa clareza
era muito condicionada pelo grau de energia que eu tinha. Estava assim
bastante interessada em descobrir como viver sem gastar energia
desnecessariamente. Muitos de nós não fomos criados com um estilo de
vida muito disciplinado. Na minha família cresci numa atmosfera que
promovia uma certa liberdade de expressão. Mas seguir os nossos próprios
caprichos e fantasias e fazer aquilo que queremos, e quando queremos, na
verdade não traz muita sabedoria à nossa vida, nem muita compreensão ou
sensibilidade. Na verdade torna-nos mais egoístas. Apesar de não me ter
sido incutido qualquer grande sentido de disciplina, enquanto criança eu
apreciava a beleza de estar viva, a harmonia da vida e a importância de
não a desperdiçar. Ainda assim a ideia de viver de uma maneira restrita
e disciplinada era bastante estranha para os meus hábitos.

Quando encontrei a meditação e a prática \emph{vipassana} (investigação
interior), pareceu-me uma introdução à disciplina muito mais fácil do
que seguir preceitos ou mandamentos. Muitas vezes olhamos de forma
alarmada para tudo o que nos limita, para qualquer tipo de convenção que
nos restrinja a liberdade, e por isso, a maneira que muitos de nós
aceita a disciplina é através da meditação. Quando vemos, dentro de
nossos corações, a maneira como nos relacionamos com o mundo dos nossos
sentidos, apercebemo-nos como tudo está interligado. O corpo e mente
estão constantemente, influenciando-se entre si, jogando um com o outro.

Conhecemos bem o prazer envolvido na gratificação dos nossos sentidos
quando, por exemplo, escutamos música inspiradora ou olhamos para um
cenário maravilhoso. Mas reparem que assim que nos apegamos à
experiência, o prazer é estragado. Isto pode ser muito doloroso e muitas
vezes sentimo-nos confundidos pelo mundo sensorial. Mas ao sermos
conscientes penetramos na natureza transitória das nossas experiências
sensoriais e tornamo-nos conhecedores do perigo de nos agarrarmos a algo
que é fugaz e impermanente. Apercebemo-nos o quão ridículo é
agarrarmo-nos àquilo que está em mudança. E com esta realização,
naturalmente, recusamos desperdiçar a nossa energia seguindo algo sobre
o qual temos pouco controlo e cuja natureza é perecer.

A restrição sensorial é o resultado natural da nossa prática de
meditação. Perceber o perigo de seguirmos cegamente os nossos sentidos,
os desejos conectados com estes, e os objectos conectados com os
desejos, é um aspecto da disciplina. Compreender naturalmente impele à
aplicação desta disciplina. Não se trata de restrição sensorial por si
própria, mas sim de sabermos que os desejos sensoriais não conduzem à
paz e não nos podem levar para além das limitações da identificação com
o nosso corpo e mente.

Quando iniciamos a vida no mosteiro temos que adoptar a disciplina dos
Oito Preceitos. Os primeiros cinco preceitos apontam para aquilo que é
chamado de Acção Correcta e Discurso Correcto: refrearmo-nos de matar,
roubar, de ter uma conduta sexual incorrecta, de mentir e de tomar
drogas e outros produtos intoxicantes. Os outros três focam-se na
renúncia, tal como a renúncia de comer depois de uma determinada hora,
de dançar, de cantar, de tocar instrumentos musicais, de embelezarmo-nos
e de dormir numa cama alta ou luxuosa. Alguns destes preceitos podem
parecer irrelevantes nos dias de hoje e nesta era. A que chamamos uma
cama alta ou luxuriosa, por exemplo? Quantos de nós tem uma cama com
dossel? Ou porque dançar, cantar ou tocar um instrumento não é permitido
como prática espiritual?

Quando nos ordenamos como monja ou monge, tomamos ainda mais preceitos e
aprendemos a viver ainda com mais restrições. A renúncia ao dinheiro,
por exemplo, torna-nos fisicamente dependentes dos outros. Estes
critérios podem parecer muito estranhos numa sociedade que venera a
independência e autonomia materiais. Mas essas directrizes começam a
fazer sentido quando integradas na nossa prática da meditação. Tornam-se
numa fonte de reflexão e põe-nos em contacto com o espírito que existe
por detrás delas. Descobrimos que as restrições ajudam-nos a refinar a
nossa conduta e a desenvolver uma consciência mais profunda da nossa
actividade física e mental e da forma como encaramos a vida. Assim,
quando olhamos dentro dos nossos corações podemos ver claramente os
resultados e as consequências das acções que realizamos com o corpo, com
a fala e com a mente.

Seguindo tal disciplina também nos acalma, e requer que sejamos muito
pacientes com nós próprios e com os outros. De um modo geral somos seres
impacientes. Gostamos de obter coisas de imediato, esquecendo que muito
do nosso crescimento e desenvolvimento vem se aceitarmos o facto, de que
este corpo e mente humanos estão longe de ser perfeitos. Existe uma
coisa, todos temos \emph{kamma} (\emph{karma}), um passado que
carregamos connosco e que é muito difícil de abandonar.

Por exemplo, quando contemplamos o preceito referente a refrearmo-nos de
usar a palavra de forma inadequada (discurso incorrecto), temos a
oportunidade de aprender a não criar mais \emph{kamma} com as nossas
palavras e de prevenir que isso seja outra fonte de mágoa e de
sofrimento para nós e para os outros seres. A Palavra Correcta
(\emph{samma vaca}) é um dos mais difíceis preceitos porque as nossas
palavras têm a capacidade de revelar os nossos pensamentos e colocar-nos
em situações vulneráveis. Enquanto nos mantivermos em silêncio não é tão
difícil. Podemos até mesmo parecer bastante sábios até começarmos a
falar. Aqueles de vós que já estiveram em retiros talvez se lembrem do
pavor que sentiram ao terem de se relacionar, novamente, verbalmente com
as outras pessoas. É tão bom, não é, estar simplesmente em silêncio com
os outros; não há zangas, não há conflitos. O silêncio é um grande
precursor da paz.

Quando começamos a falar, é outro jogo. Já não nos podemos enganar a nós
próprios. Geralmente identificamo-nos fortemente com aquilo que pensamos
e então a nossa fala, a directa expressão dos pensamentos, também se
torna um problema. Mas, a menos que aprendamos a falar com maior
sabedoria, as nossas palavras continuarão a ser bastante agressivas para
nós e para os outros. Na verdade o discurso por ele próprio não é tanto
o problema mas mais o sítio de onde ele provém. Quando existe a Plena
Atenção não existem pormenores deixados para trás. Por vezes dizemos
algo que não é muito inteligente e depois pensamos que podíamos ter dito
algo melhor. Mas se tivermos plena consciência ao falar, nesse momento,
de alguma forma, a mancha dessa auto-imagem que está tão poderosamente
impressa em nós, é removida ou pelo menos diminuída.

Enquanto seguimos este caminho de prática, disciplina é algo que
realmente faz sentido. Quando começamos a entrar em contacto com a crua
energia do nosso ser, e com a crua energia de ódio, cobiça, estupidez,
inveja, ciúmes, desejos cegos, orgulho e conceitos, tornamo-nos muito
gratos por termos algo que possa conter tudo isto. O simples facto de
olhar para o estado do nosso planeta Terra é uma grande reflexão acerca
do resultado nocivo da falta de disciplina e da falta de contenção da
nossa avareza, ódio e ilusão. Então, para sermos capazes de conter a
nossa energia dentro da estrutura de uma disciplina moral, precisamos de
ser muito conscientes e cuidadosos, porque a natural tendência da nossa
mente é a de se esquecer dela própria.

Esquecemo-nos de nós próprios e do objectivo último de nossas vidas e em
vez disso enchemo-nos com coisas que não podem verdadeiramente
satisfazer ou nutrir o nosso coração. A disciplina também requer
humildade, visto que enquanto somos imaturos seguindo os nossos
impulsos, podemos sentir-nos reprimidos e inibidos pela disciplina e,
consequentemente, em vez dessa disciplina ser uma fonte de liberdade,
acabamos por sentir que estamos presos por ela.

Somos muito afortunados em ter a oportunidade de praticar e perceber que
as nossas acções, as nossas palavras ou os nossos desejos não são em
última instância, aquilo que verdadeiramente somos. Com o aprofundamento
da nossa meditação, a qualidade da impermanência de todas as coisas
torna-se mais clara. Tornamo-nos mais e mais conscientes da natureza
transitória das nossas acções, das nossas palavras, e dos nossos
sentimentos relacionados com estas. Começa a tomar sentido aquilo que
está sempre presente nas nossas experiências, mas que não é tocado por
estas. Esta qualidade de presença está sempre disponível e não é
realmente afectada pelas nossas interacções sensoriais. Quando esta
qualidade de atenção é cultivada e mantida, começamos a relacionar-nos
melhor com a nossa energia, com os nossos sentidos de contacto e com o
mundo sensorial. Descobrimos que a atenção consciente é na realidade uma
forma de protecção. Sem ela estamos simplesmente à mercê dos nossos
pensamentos ou dos nossos desejos e somos cegados por eles. Este refúgio
na Plena Atenção ou o cultivo da restrição protege para não cairmos em
dolorosos e infernais estados de mente. Outro aspecto de disciplina é o
de ter sabedoria na atenção e no uso do mundo material. O nosso contacto
imediato com o mundo físico é através do corpo. Quando aprendemos como
cuidar do mundo físico estamos a procurar as raízes das nossas vidas.
Fazemos o que é necessário para colocar o corpo e a mente em harmonia.
Isto é o resultado natural da restrição. Aos poucos tornamo-nos como uma
linda flor de lótus que representa a pureza e cresce para fora da água,
enquanto é nutrida através das suas raízes na lama. Deveis ter notado
que o Buddha é frequentemente retratado sentado numa flor de lótus, o
qual simboliza a pureza do coração humano. A menos que criemos esse
fundamento de moralidade baseada no mundo da nossa vida do dia-a-dia,
não podemos realmente realizar o nosso crescimento como a flor de lótus.
Simplesmente murchamos. Na vida monástica, usar com sabedoria os quatro
requisitos - roupa, comida, abrigo e medicamentos -- consiste numa
reflexão diária que é extremamente útil, pois a mente tende para
esquecer, fazer interpretações erradas ou tomar as coisas como
garantidas. Estes quatro requisitos são uma parte essencial da nossa
vida.

É um dever para nós, monásticos, cuidar dos nossos hábitos. Temos de
remendá-los, repará-los, lavá-los, relembrando, que apenas temos um
conjunto e de que estes hábitos provêm da generosidade de outros. E o
mesmo se passa com a comida que ingerimos. Vivemos de oferendas de
alimentos. Todos os dias as pessoas oferecem-nos uma refeição, pois não
nos é permitido armazenar comida para proveito próprio, para o dia
seguinte. Então a nossa reflexão diária antes da refeição (recitado por
alguém) é relembrar que não devemos comer sem pensar, agradecidamente,
sobre a oferta da comida. Como mendicantes também reflectimos sobre o
sítio onde vivemos. Podemos não gostar do papel de parede do nosso
quarto mas a reflexão acerca do nosso abrigo ``este quarto é apenas um
tecto sobre as nossas cabeças por uma noite'' ajuda-nos a manter as
nossas necessidades físicas em perspectiva. Temos também em consideração
que sem as ofertas destes requisitos não poderíamos ter esta vida. Esta
reflexão nutre um sentido de gratidão no coração.

Cuidar do mundo físico e daquilo que está à nossa volta é uma parte
essencial do treino da mente e do corpo e da nossa prática do
\emph{Dhamma} (\emph{Dharma}). Se não formos capazes de cuidar daquilo
que nos é imediato, como podemos pretender cuidar das realidades
últimas. Se não aprendermos a arrumar o nosso quarto todos os dias como
podemos tratar das complexidades da nossa mente?

Reflectir em coisas simples, como tomar conta do sítio onde vivemos e
não fazermos um mau uso das nossas possessões materiais é muito
importante. Naturalmente é mais difícil fazer isto, quando temos
controlo sobre o mundo material e podemos usar dinheiro para comprar
aquilo que queremos, pois facilmente temos pensamentos descuidados: «Ora
bem, perdi isto ou parti aquilo, não faz mal, arranjo outro». Outro
aspecto de disciplina e restrição é o correcto modo de vida. Para um
monge ou monja existe uma longa lista de coisas com as quais não nos
devemos envolver, como praticar adivinhação ou participar em actividades
políticas, etc\ldots{} Eu posso apreciar o valor disto cada vez mais e, mais
agora, quando vejo que em algumas partes do mundo, o \emph{Sangha} se
envolveu em coisas mundanas e os monges encontraram-se em determinadas
situações, como a de receberem itens luxuosos ou mesmo de se tornarem
proprietários de casas ou terras. O correcto modo de vida é um dos
aspectos do Nobre Caminho Óctuplo, que cobre um amplo leque de
actividades como não enganar, não persuadir, insinuar, depreciar e
permutar e, para os leigos, não se envolverem em comércio de armas,
seres humanos, carne, bebidas alcoólicas e venenos. Estas linhas guia,
evocam uma cuidadosa consideração de como queremos viver a nossa vida e,
com que tipo de profissão ou situação nos devemos envolver.

As reflexões acerca dos preceitos, dos requisitos, do correcto modo de
vida e a disciplina da nossa mente e do nosso corpo são as condições de
suporte com as quais a disciplina última se pode manifestar nos nossos
corações. Essa disciplina última é a nossa total dedicação à Verdade, ao
\emph{Dhamma}, e à constante aspiração do nosso coração humano para ir
além das nossas vidas egocêntricas. Por vezes não podemos realmente
dizer o que é, mas através da prática da meditação podemos estar
verdadeiramente em contacto com essa realidade, o \emph{Dhamma,} em nós.
Todos os caminhos e disciplinas espirituais estão aqui como condições de
suporte e meios para manter viva a aspiração de realizar a Verdade no
nosso coração. Esse é o seu verdadeiro propósito.

\bigskip

{\raggedleft\itshape
  Tradução de Appamādo Bhikkhu
\par}