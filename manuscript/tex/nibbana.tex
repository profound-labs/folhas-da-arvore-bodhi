\chapterAuthor{Ajahn Sumedho}
\chapter{Nibbāna, Aqui e Agora}
\tocChapterNote{Ajahn Sumedho}

A dificuldade com a palavra \emph{Nibbana} é que o seu significado está
para além das palavras. É, essencialmente, indefinível.

Outra dificuldade é que vários budistas vêm o \emph{Nibbana} (Sânscrito:
Nirvāna) como algo inatingível - como algo tão elevado e tão remoto que
não somos suficientemente merecedores de o realizar. Ou então vemos o
\emph{Nibbana} como um objectivo, algo desconhecido e indefinido, que
devemos de alguma maneira tentar atingir.

A maior parte de nós está condicionada desta maneira. Queremos alcançar
ou atingir algo que não temos. Desta forma, o \emph{Nibbana} é visto
como algo que, se trabalharmos arduamente, mantivermos os preceitos
morais (\emph{sila}), meditarmos diligentemente, tornarmo-nos monásticos
e devotarmos a nossa vida à prática, talvez possamos alcançar -- ainda
que não estejamos certos do que é.

Ajahn Chah usaria as palavras ``a realidade do não-apego'' como
definição do \emph{Nibbana}: realizar a realidade do ``não-apego''. Isto
ajuda a colocá-la num contexto, porque aqui o ênfase é em despertar para
a maneira como nos apegamos e agarramos inclusivamente a palavras como
``\emph{nibbana}'', ``Budismo'', ``prática'', ``\emph{sila}'' ou seja o
que for.

Diz-se frequentemente que o caminho budista é o do desapego, porém isso
pode tornar-se apenas noutra afirmação à qual nos apegamos e agarramos.
É um Catch 22\footnote{%
Catch 22 é um livro de Joseph Heller, onde predomina uma lógica
auto-contraditória, repleta de paradoxos e contra-sensos, sendo circular
e repetitiva; a irracionalidade lógica prevalece em todo o livro. Daí a
analogia, por não conseguirmos chegar a um consenso devido à limitação
das palavras e percepções.}:
não importa o quanto queremos que faça sentido,
termina-se em confusão total devido à limitação da linguagem e da
percepção. Temos de ir além da linguagem e da percepção, e o único modo
de ir além dos pensamentos e hábitos emocionais é através da tomada de
consciência - sermos conscientes dos pensamentos e das emoções. ``A ilha
para além da qual não se pode ir'' é a metáfora para este estado de
\emph{ser} desperto e consciente, em oposição ao conceito de nos
\emph{tornarmos} despertos e conscientes.

Nas aulas de meditação, as pessoas começam frequentemente com uma ilusão
básica, a qual nunca chegam a pôr à prova: a ideia de que ``eu sou
alguém com muitos apegos e muitos desejos e tenho de praticar de modo a
livrar-me destes apegos e desejos. Não me devo agarrar a nada''. Este é
geralmente o ponto de partida. Portanto, começamos a nossa prática a
partir desta base e, muitas vezes, o resultado é a desilusão e o
desapontamento, pois a nossa prática está baseada no apego a uma ideia.

Eventualmente, apercebemo-nos que independentemente do esforço que
façamos para nos libertarmos do apego e do desejo, ou de qualquer outra
coisa - tornarmo-nos monges ou ascetas, sentarmo-nos por horas e horas,
irmos a retiros vezes sem conta, fazer todas as coisas que acreditamos
que livrarão dessas tendências -- acabamos por nos sentir desapontados,
porque a ilusão básica nunca foi reconhecida.

É por isso que a metáfora de ``a ilha para além da qual não se pode ir''
é tão poderosa, pois aponta para o princípio duma consciência além da
qual não se pode ir. É muito simples, muito directo e impossível de ser
concebido. Temos de confiar nisso, temos de confiar nesta simples
capacidade que todos possuímos de estarmos totalmente presentes e
totalmente despertos, e começar a reconhecer o apego e as ideias que
temos sobre nós mesmos, sobre o mundo à nossa volta, sobre os nossos
pensamentos, percepções e sentimentos. O caminho da plena atenção é o
caminho do reconhecimento das condições tal como elas são. Reconhecemos
simplesmente a sua presença, sem as julgar, criticar ou louvar.
Aceitamos a sua presença, quer sejam condições positivas quer negativas.
E, à medida que vamos confiando cada vez mais nesta consciência ou plena
atenção, começamos a experienciar a realidade de ``a ilha para além da
qual não podemos ir''.

Quando comecei a praticar meditação, a ideia que tinha de mim era a de
alguém que estava muito confuso. Queria sair desta confusão e livrar-me
dos meus problemas, e tornar-me alguém com um pensamento lúcido, que
poderia um dia tornar-se iluminado. Foi isso que me levou à meditação
budista e à vida monástica. Mas depois, reflectindo nesta posição de que
"sou alguém que precisa de fazer algo ", comecei a ver isso como uma
condição criada - uma ideia que eu tinha criado. E se eu actuasse a
partir desse pressuposto, ainda que pudesse vir a desenvolver todo o
tipo de aptidões e viver uma vida louvável, boa e benéfica para mim e
para os outros, no fim da história, poder-me-ia sentir bastante
desapontado por não ter atingido o objectivo do \emph{Nibbana}.

Felizmente que na vida monástica tudo é direccionado para o momento
presente. Estamos sempre a aprender a pôr à prova as nossas ideias sobre
de nós mesmos e a vermos para além das mesmas. Um dos maiores desafios é
enfrentar a ideia de que ``sou alguém que precisa de fazer algo para se
tornar iluminado no futuro'', através do simples reconhecimento disso,
como uma assunção criada por nós. Aquilo que é consciente em nós, sabe
que essa ideia é algo criada a partir da ignorância, ou da falta de
compreensão. Quando vemos e reconhecemos isto totalmente, então paramos
de criar essas suposições.

Estar consciente não é fazer juízos de valor acerca dos nossos
pensamentos, emoções, acções ou palavras. Ser consciente é compreender
essas coisas integralmente - elas são o que são, neste momento. Desta
forma, descobri que é muito útil aprender a estar consciente das
condições sem as julgar. Assim, o \emph{karma} resultante das palavras e
acções passadas, tal como surge no presente, é inteiramente reconhecido
sem ser deturpado, sem se tornar num problema. É o que é. Aquilo que
surge cessa. Quando reconhecemos isto e permitimos que as coisas cessem
de acordo com a sua natureza, essa realização da cessação dá-nos uma fé
crescente na prática do desapego. Os apegos que temos, ainda que às
coisas boas, como por exemplo, ao Budismo, também podem ser vistos como
apegos que nos cegam. Isso não quer dizer que tenhamos de nos livrar do
Budismo; simplesmente reconhecemos o apego como apego e constatamos que
nós mesmos o criamos devido à nossa ignorância.

À medida que vamos reflectindo sobre isto, a tendência para nos
apegarmos às coisas vai diminuindo, e a realidade do não-apego revela-se
naquilo que podemos chamar de \emph{Nibbana}. Se virmos a partir deste
prisma, o \emph{Nibbana} existe aqui e agora. Não é algo para
alcançarmos no futuro. A realidade está aqui e agora. É mesmo muito
simples, mas está para além das palavras, da descrição. Não pode ser
outorgado nem mesmo transmitido, apenas pode ser conhecido por cada
pessoa, por si própria.

Quando começamos a realizar ou a reconhecer o desapego como o Caminho,
podemos sentir-nos bastante amedrontados com isso. Pode parecer que está
a ocorrer uma espécie de aniquilação: tudo aquilo que penso que sou no
mundo, tudo aquilo que assumo como estável e real começa a desvanecer-se
e isso pode ser assustador. Mas se tivermos fé para continuar a aguentar
essas reacções emocionais e se deixarmos que aquilo que surge, acabe, de
acordo com a sua natureza, então encontraremos a nossa estabilidade, não
em alcançar ou atingir algo, mas em ser -- ser desperto, ser consciente.

Há muitos anos, no livro de William James ``\emph{The Varieties of
Religious Experience}'' (``As Variedades da Experiência
Religiosa''), encontrei um poema de A. Charles Swinburne. Apesar de ter
aquilo que alguns descreveram como uma mente degenerada, Swinburne
produziu algumas reflexões bem poderosas:

\begin{quote}
Aqui começa o mar que não termina senão no fim do mundo.\\
Onde nos encontramos,\\
pudéssemos nós ver o próximo farol de alto mar,\\
para além destas ondas que brilham,\\
Saberíamos o que homem algum soube\\
ou que olhar humano algum perscrutou\ldots{}\\
Ah, mas aqui o coração do homem palpita, ansioso pelo\\
desconhecido com venturosa alegria,\\
A partir da margem que não tem margem para além dela,\\
presente em todo o mar.

(Extracto de ``\emph{On the Verge}'', em ``\emph{A Midsummer Holiday}'')
\end{quote}

O poema é um eco da resposta do Buddha à pergunta de Kappa, que se
encontra no \emph{Sutta Nipata:}

\begin{quote}
A seguir vem Kappa, o estudante \emph{brahmin}.

``Senhor, disse ele, há pessoas presas no meio da corrente, no terror e
no medo do ímpeto do rio do ser, onde a morte e a decrepitude os afogam.
Para o bem deles, Senhor, dizei-me onde encontrar uma ilha, dizei-me
onde existe terra firme, fora do alcance de toda esta dor''.

``Kappa, disse o Mestre, para o bem dessas pessoas presas no meio do rio
do ser, oprimidas pela morte e pela decrepitude, Eu dir-te-ei onde
encontrar terra firme''.

``Existe uma ilha, uma ilha para além da qual não podes ir. É um lugar
de `não-existência', um lugar de ``não-possessividade'' e de
``não-apego''. É o fim definitivo da morte e da decrepitude, e é por
isso que eu o chamo de \emph{Nibbana} (o extinguido, o sereno)''.

``Há pessoas que, em plena consciência, o realizaram e estão
completamente serenas, aqui e agora. Elas não se tornam escravos a
trabalhar para \emph{Mara}, para a Morte; elas não são subjugadas pelo
seu poder''.

SN 1092-5 (traduzido por Ven. Saddhatissa)
\end{quote}

Em Inglês, ``nada'' (\emph{nothingness)} pode soar como aniquilação,
como niilismo. Mas também se pode colocar a ênfase na palavra
``existência'' (\emph{thingness}), e aí tem-se algo como
``não-existência'' (\emph{no-thingness}). Portanto, o \emph{Nibbana} não
é uma coisa que se possa encontrar. É o lugar da ``não existência'', da
``não possessividade'' e do desapego. É um lugar, como disse Ajahn Chah,
onde se experiencia ``a realidade do não-apego''. \emph{Nibbana} é uma
realidade que cada um de nós pode conhecer por si mesmo - assim que
reconhecermos e realizarmos o não-apego.


\bigskip

{\raggedleft\itshape
  % TODO accent
  Tradução de Anagārika Filipe
\par}
