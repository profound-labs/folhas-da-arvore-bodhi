\chapter{A Perspectiva da Floresta}

\textbf{Ajahn Amaro}

Quando estou em contacto com os ensinamentos Dzogchen, tenho
frequentemente a estranha sensação de ouvir os ecos e ver as imagens dos
meus próprios mestres, Ajahn Chah e Ajahn Sumedho, quer, pela maneira
como esses ensinamentos descrevem princípios com os quais estou
familiarizado, quer, pelo uso de algumas das mesmas analogias e frases.
Quando me apercebi desta semelhança, dei-me conta que tinha estado a
praticar de~~um modo parecido ao Dzogchen durante pelo menos a última
metade da minha vida monástica, desde aproximadamente 1987. Se eu
tivesse sobrancelhas, acho que as teria erguido um pouco!

Mas talvez a convergência não seja tão surpreendente. Afinal, todos
temos o mesmo mestre: o \emph{Dhamma} provém do Buddha e está enraizado
na nossa própria natureza. Pode haver 84.000 distintas portas para o
\emph{Dhamma}, mas em essência há um só \emph{Dhamma}.

Existem vários ensinamentos Tibetanos que com o passar do tempo passei a
apreciar, em especial aqueles que descrevem a excelsa anatomia e nuances
de~\emph{rigpa} (1), também traduzido como conhecimento ou perspectiva.
A tradição da Floresta da Tailândia, a linhagem na qual
predominantemente treinei, depende mais da eloquência e inspiração de
alguns mestres que~improvisam temas do Dhamma que lhes ocorrem no
momento. Isto mantém os ensinamentos vivos e frescos, mas também
significa que pode haver muita inconsistência na maneira como os
conteúdos são expressos. Por isso, aprendi muito com a natureza bem
estruturada e sistematizada dos ensinamentos Dzogchen.

Os ensinamentos
de~\href{http://www.acessoaoinsight.net/arquivo_textos_theravada/tradicao_de_florestas.php\#Chah}{Ajahn
Chah}~cobriam temáticas de âmbito bastante alargado, mas ele era
particularmente notável pela maneira aberta, hábil e livre com que
falava sobre a esfera da derradeira realidade. E ele era capaz de o
fazer com qualquer pessoa que ele sentisse ser capaz de compreender,
fossem leigos ou monásticos. A sua maneira de falar sobre esse tema e
sobre a ``consciência que sabe'' -- ~a sua compreensão do conhecimento
-- reflecte muitas similaridades com o Dzogchen. Por conseguinte, pensei
que descrever algumas dessas similaridades poderia ajudar, bem como
descrever alguns dos métodos ensinados por Ajahn Sumedho, o discípulo
sénior ocidental de Ajahn Chah. Tentarei também proporcionar outros
ângulos ou pontos de vista da tradição Theravada, que têm alguma
relevância na nossa compreensão e prática nesta área.

\textbf{Quanto mais nos apressamos, mais devagar avançamos }

É fácil ficar muito ocupado com a vida espiritual, até mesmo compelido e
obcecado. Durante os primeiros 10 anos da minha vida monástica tornei-me
um monge, até certo ponto fanático. Isto pode parecer um paradoxo, mas
não é de forma nenhuma~~impossível. Eu tentava fazer tudo a 120 por
cento.

Levantava-me de manhã muito cedo e fazia todo o tipo de práticas
ascéticas, todos os tipos de~\emph{pujas} especiais~e coisas do género.
Nem sequer me deitava; não me deitei para dormir durante cerca de três
anos. Por fim dei-me conta que estava a fazer coisas a mais e não havia
nenhuma noção de espaço interno ao longo do dia.

Estava desesperadamente ocupado com a meditação. Durante aquela época, a
minha vida estava completamente preenchida. Estava sempre um pouco
empertigado e agitado. Não conseguia sequer comer ou atravessar o pátio
sem que isso fosse um problema. Por fim questionei-me: ``Porque estou a
fazer isto? É suposto viver esta vida em paz, para a realização, para a
libertação, e no entanto os meus dias estão todos atafulhados.''

Eu já devia ter percebido a mensagem há muito tempo. Costumava sentar-me
no chão duro, sendo o uso de um \emph{zafu} (almofada para meditação) um
sinal de fraqueza para mim. Uma das monjas ficou tão farta de me ver
adormecer durante todas as sessões que aproximou-se e perguntou, ``Posso
lhe oferecer uma almofada, Ajahn?''

``Não muito obrigado, não preciso.''

Ela respondeu, ``Eu acho que o Ajahn precisa.''

Finalmente, fui ter com o Ajahn Sumedho e disse, ``Decidi abrir mão das
minhas práticas ascéticas. Vou simplesmente seguir a rotina habitual e
fazer tudo de modo absolutamente normal.'' Foi a primeira vez que o vi
ficar empolgado. ``Finalmente!'' foi a resposta dele. Pensei que ele
fosse dizer, ``Bem, se achas que sim\ldots{}.''

Ele estava à espera que eu percebesse que o importante não era a
quantidade de coisas que eu fazia, as horas que passava sentado em
meditação, a quantidade de \emph{mantras} que recitava ou o quão
estritamente eu seguia as regras. O que importava era incorporar o
espírito do não-devir e do não-lutar em tudo o que fazia. De repente
percebi que a importância do não-lutar era algo que Ajahn Sumedho havia
ensinado por muitos anos; eu simplesmente tinha estado a ouvir.

Ajahn Sumedho encorajava o estar consciente daquilo a que chamamos a
``tendência para o devir''. Em pāli a palavra é ``\emph{bhava}''; na
tradição Tibetana a palavra é usada da mesma forma. Ela descreve o
desejo de nos tornarmos algo. Fazemos~\emph{isto}~para
obter~\emph{aquilo}. Trata-se de estarmos sempre ocupados, sempre a
``fazer algo'' sempre a controlar o método, as práticas, as regras e a
mecânica de modo a chegar a algum lugar. Este hábito é a causa de muitos
dos nossos problemas. ~

Para que as sementes cresçam precisamos de solo, estrume, água e luz
solar. Mas se o saco de sementes ficar no armazém, continua a faltar-nos
o elemento essencial. Enquanto arrastamos o estrume e a água de um lado
para o outro, sentimos que estamos a fazer algo. ``Estou mesmo a
trabalhar arduamente na minha prática!'' Enquanto isso, o mestre está
ali em pé ao lado do saco de sementes relembrando-nos (gesticula como se
estivesse a apontar para um saco no canto).

Ajahn Sumedho fala repetidamente sobre ser iluminado em vez de tornar-se
iluminado no futuro. ``Estejam despertos agora; sejam iluminados no
momento presente. Não se trata de fazer algo agora para se iluminarem
depois. Esse tipo de pensamento está coligado com o eu e o tempo e não
produz frutos''. Os ensinamentos Dzogchen são iguais. Não se trata de
encontrar~\emph{rigpa}~como um objecto ou de fazer algo agora para
obter~\emph{rigpa}~no futuro; trata-se~~de na verdade ser
\emph{rigpa}~agora. Assim que começamos a fazer algo com isso ou a
dizer, ``Eh, olha, consegui'' ou ``Como posso mantê-lo?'' a mente
agarra-se a esse pensamento e abandona~ \emph{rigpa}~-- a menos que o
pensamento seja visto como apenas mais uma formação dentro do espaço
de~\emph{rigpa}.

O próprio Ajahn Sumedho nem sempre tinha tido a maior clareza em relação
a este ponto. Muitas vezes ele contava a história sobre as suas próprias
obsessões em ser ``um meditador''. O método de ensino de Ajahn Chah
colocava bastante ênfase na prática de meditação formal. Mas ele também
era extremamente dedicado em não fazer da meditação formal algo distinto
do resto da vida. Falava sobre manter a continuidade da prática quer
fosse a caminhar, em pé, sentados ou deitados. O mesmo se aplicava a
comer, usar a casa de banho e trabalhar. O ponto era manter uma contínua
atenção consciente. \protect\hypertarget{R2}{}{}Ele costumava dizer,
``Se a tua paz jaz no assento de meditação, quando te levantas do
assento deixas a tua paz para trás.''

Certa vez foi oferecido a Ajahn Chah um pedaço de terra com floresta, no
topo de uma montanha na sua província natal. O generoso doador disse:
``Se você conseguir encontrar uma forma de construir uma estrada até ao
topo da montanha, construirei lá um mosteiro para si.'' Sempre disposto
a enfrentar este tipo de desafio, Ajahn Chah passou uma ou duas semanas
na montanha e encontrou um caminho até ao topo. De seguida, trouxe a
comunidade monástica inteira para construir a estrada. ~

Ajahn Sumedho era um monge recém-chegado. Ele tinha chegado há um ou
dois anos e era um meditador muito sério. Não tinha mostrado muito
interesse em deixar a vida estabelecida no mosteiro principal, Wat Nong
Pah Pong, mas juntou-se ao grupo e ali estava ele, a partir pedra
debaixo de sol, a empurrar carrinhos de mão cheios de entulho e a
trabalhar arduamente com o resto da comunidade. Depois de dois ou três
dias, Sumedho estava cheio de calor, suado e carrancudo. Ao final do
dia, depois de um turno de 12 horas de trabalho, todos se sentavam para
meditar e ficavam a cabecear devido ao cansaço. Ajahn Sumedho pensou,
``Isto é inútil. Estou a perder o meu tempo. A minha meditação
desmoronou-se por completo. Isto não ajuda a vida santa de modo
nenhum.''

Ele explicou cuidadosamente a sua preocupação a Ajahn Chah: ``Sinto que
todo o trabalho que estamos a fazer é prejudicial para a minha
meditação. Acho que seria muito melhor para mim se eu não fizesse parte
deste trabalho. Preciso fazer mais meditação, sentado e a andar, preciso
de mais prática formal. Isto ajudar-me-ia bastante e acho que seria
muito melhor.''

Ajahn Chah disse: ``Ok., Sumedho. Sim, podes fazer isso. Mas é melhor eu
informar o \emph{Sangha} para que todos saibam o que está a acontecer.''
Ele era capaz de ser assim maroto!

Na reunião do \emph{Sangha,} Ajahn Chah disse: ``Gostaria de fazer um
comunicado para todos. Eu sei que viemos todos até aqui para construir
esta estrada. Também sei que todos estamos a trabalhar arduamente a
partir pedra e a carregar entulho. Sei que é importante que este
trabalho seja feito, mas a tarefa da meditação também é muito
importante.~Tan Sumedho perguntou-me se podia praticar meditação
enquanto nós construímos a estrada e eu disse-lhe que não há problema
absolutamente nenhum com isso. Não quero que tenham pensamentos de
crítica em relação a ele. Por mim está tudo bem. Ele pode ficar sozinho
e meditar e nós continuaremos a construir a estrada.''

Ajahn Chah trabalhava de sol a sol. Quando não estava a trabalhar na
estrada, estava a receber visitantes e a ensinar \emph{Dhamma}. Ele
estava mesmo a dar o ``litro''. Enquanto isso, Ajahn Sumedho permanecia
só e meditava. Sentiu-se muito mal no primeiro dia e ainda pior no
segundo. No terceiro dia, não aguentou mais. Sentiu-se torturado e por
fim abandonou a sua solidão. Juntou-se novamente aos outros monges,
partiu pedra, carregou entulho, entregando-se totalmente ao trabalho.~

Ajahn Chah olhou para o jovem monge entusiasmado e com um largo sorriso
nos lábios, perguntou: ``Estás a gostar do trabalho, Sumedho?''

``Sim, Luang Por.''

``Não é estranho que a tua mente esteja mais contente agora no calor e
na poeira do que quando estavas a meditar sozinho?''

``Sim, Luang Por.''

A lição? Ajahn Sumedho tinha criado uma falsa divisão entre o que é e o
que não é meditação, quando na verdade não existe nenhuma diferença.
Quando nos entregamos de coração a qualquer coisa que fazemos, a
qualquer coisa que experienciamos ou ao que estiver a acontecer à nossa
volta, sem agendas ou preferências pessoais a assumir o controle, o
espaço de ~\emph{rigpa,~}o espaço da plena consciência, é exactamente o
mesmo.

\textbf{O Buddha é Consciência}

Os ensinamentos de Ajahn Chah também são similares ao Dzogchen no que
diz respeito à natureza do Buddha. Quando vamos ao fundo da questão, a
consciência da mente não é ``algo''. No entanto, ela é um atributo da
natureza fundamental da mente. Ajahn Chah referia-se a essa plena
consciência, essa natureza conhecedora da mente, como o Buddha: ``Este é
o verdadeiro Buddha, aquele que sabe (\emph{poo roo}).'' A maneira
habitual de falar sobre a consciência, tanto no caso de Ajahn Chah, como
no de outros mestres da Tradição da Floresta, era empregando o termo
``Buddha'' desta forma -- a consciência plena, a qualidade desperta da
nossa mente. Isto é o Buddha.

Ele dizia coisas como, ``O Buddha que realizou o~\emph{parinibbāna}~há
2.500 anos, não é o Buddha no qual tomamos refúgio.'' Por vezes ele
gostava de chocar as pessoas, quando sentia a necessidade de lhes chamar
a atenção para os ensinamentos. Quando ele dizia algo deste género elas
pensavam estar diante de um herético. Ele diria: ``Como pode aquele
Buddha ser um refúgio? Ele foi-se. Foi-se, completamente. Isto não é um
refúgio. Um refúgio é um lugar seguro. Então, como pode esse grande ser
que viveu à 2.500 anos proporcionar segurança? Pensar nele pode fazer
com que nos sintamos bem, mas essa sensação também é instável. É uma
sensação inspiradora, mas pode ser facilmente perturbada.''

Quando nos rendemos ao saber, então nada pode magoar o coração. Este
repouso naquilo que sabe é o que faz do Buddha um refúgio. Esta
``natureza que sabe'' da mente é invulnerável, inviolável. O que
acontece com o corpo, emoções e percepções é secundário, pois ``aquilo
que sabe'' está para além do mundo dos fenómenos. Portanto, esse é o
verdadeiro refúgio. Quer, experienciemos prazer ou dor, êxito ou
fracasso, elogio ou crítica, essa natureza conhecedora da mente é
absolutamente serena. Ela é imperturbável e incorruptível. Tal como um
espelho que não é embelezado ou maculado pelas imagens nele reflectidas,
a natureza conhecedora da mente não pode ser tocada por nenhuma
percepção dos sentidos, nenhum pensamento, nenhuma emoção, nenhum estado
de ânimo, nenhum sentimento. É de uma grandeza transcendente. Os
ensinamentos Dzogchen transmitem o mesmo: ``Não existe sequer uma ponta
de fio de cabelo de envolvimento dos objectos mentais na plena
consciência, na natureza da mente em si.'' É por isso que a plena
consciência é um refúgio; a plena consciência é o próprio centro da
nossa natureza.

\textbf{Alguém viu os meus olhos?}

Outro paralelo entre os ensinamentos do Dzogchen e de Ajahn Chah vem sob
a forma de um aviso: não procurem o incondicionado, ou~\emph{rigpa}, com
a mente condicionada. Nos versos do Terceiro Patriarca do Zen pode-se
ler: ``Procurar a Mente com a mente discriminatória é o maior de todos
os erros.'' Ajahn Chah expressava a futilidade e o absurdo dessa
tendência dando como exemplo andar a cavalo e procurar o cavalo ao mesmo
tempo. Estamos a montar o cavalo e a perguntar, ``Alguém viu o meu
cavalo? Alguém sabe do meu cavalo?'' Todos olhariam para nós como se
estivéssemos loucos. Então, cavalgamos até à próxima aldeia e
perguntamos a mesma coisa: ``Alguém viu meu cavalo?''

Ajahn Sumedho emprega um exemplo semelhante. Ao invés de procurar um
cavalo, ele usa a imagem de procurar os próprios olhos. O próprio órgão
com o qual vemos está a realizar a procura, no entanto prosseguimos na
busca exterior: ``Alguém viu os meus olhos? Não consigo ver os meus
olhos em nenhum lugar. Eles devem estar por aqui algures mas não consigo
encontrá-los.''

Não podemos ver os nossos olhos, mas conseguimos ver. Isso significa que
a consciência não pode ser um objecto. Mas pode haver consciência! Ajahn
Chah e outros mestres da Tradição da Floresta empregavam a expressão,
``ser o saber.'' É como \emph{ser} \emph{rigpa}. Nesse estado, a mente
conhece a sua própria natureza, o \emph{Dhamma} é ciente da sua própria
natureza. E é tudo. Assim que tentamos fazer disso um objecto, cria-se
uma estrutura dualista, um sujeito \emph{aqui} olhando para um objecto
\emph{ali}. E só existe solução quando abrimos mão dessa dualidade e
abandonamos essa ``busca''. Então o coração permanece somente no saber.
Mas o hábito é pensar, ``Não estou a procurar o suficiente. Ainda não os
encontrei. Os meus olhos devem estar aqui em algum lugar. Afinal de
contas consigo ver. Tenho de esforçar-me mais para os encontrar.''

Já alguma vez estiveram numa entrevista num retiro, na qual depois de
descreverem a vossa prática de meditação o professor olha para vocês e
diz, ``É necessário mais esforço!''? Vocês pensam, ``Mas estou a
esforçar-me o mais possível!'' Necessitamos do esforço, mas precisamos
de o fazer de forma inteligente. O tipo de esforço que precisamos
desenvolver é aquele que envolve ser mais claro mas ``forçar'' menos.
Esta qualidade de saber descontrair é vista como crucial, não somente
nos ensinamentos Dzogchen, mas também na prática monástica Theravada.

É irónico que essa descontracção seja justamente construída sobre uma
ampla gama de práticas preparatórias. No treino~\emph{ngondro}~Tibetano
o praticante realiza 100.000 prostrações, 100.000 visualizações, 100.000
\emph{mantras}, seguindo-se anos de estudos, mantendo as virtudes
(\emph{sīla}), e assim por diante. Na tradição Theravada também
temos~\emph{sīla}: as práticas de virtude para os leigos e para as
comunidades monásticas, bem como o refinamento do treino na disciplina
do~\emph{Vinaya}. Realizamos muitas práticas devocionais e cânticos, e
uma quantidade enorme de treino nas técnicas de meditação, como a
atenção plena na respiração, a consciência focada no corpo e assim por
diante. Temos também a prática de viver em comunidade. (Um dos monges
seniores do meu \emph{Sangha} certa vez referiu-se ao treino comunitário
monástico como sendo a prática das 100.000 frustrações -- não somos
qualificados até que tenhamos alcançado a centésima milésima!) Portanto,
há um trabalho preparatório enorme, que é necessário realizar para que
esse descontrair seja efectivo.

Gosto de pensar nessa descontracção como usar uma mudança acima. Usamos
a ``quinta'' mantendo a mesma velocidade, mas com menos rotações. Até eu
contar a Ajahn Sumedho que havia desistido das minhas práticas
ascéticas, eu estava em quarta e a correr. Havia sempre uma pressão, uma
atitude de ir até ao limite. Quando reduzi um grau e já não estava
\emph{tão} fanático em relação às regras e a fazer \emph{sempre} tudo
perfeito, esse pequeno elemento de descontracção permitiu que tudo isso
acabasse; simplesmente porque larguei o \emph{stress}, parei de forçar.
A ironia é que continuava a realizar 99.9 por cento das minhas tarefas e
práticas espirituais, mas realizava-as sem estar obcecado. Podemos
descontrair sem indulgência e consequentemente desfrutar dos frutos do
nosso trabalho. Isto é o que queremos dizer com libertar-se do sentido
de devir e aprender a \emph{ser}. Se estivermos demasiado tensos e
ávidos por querer chegar ao outro lado, estaremos destinados a cair do
arame.

\textbf{Realização da Cessação}

Outro importante aspecto do conhecimento é a sua ressonância com a
experiência da cessação, \emph{nirodha}. A experiência de~\emph{rigpa}~é
sinónimo da experiência de~\emph{dukkha-nirodha}, a cessação do
sofrimento.

Parece bom, não é? Praticamos para nos libertarmos do sofrimento porém
ficamos tão apegados ao trabalho com as coisas da mente que
quando~\emph{dukkha} cessa e o coração torna-se espaçoso e vazio,
podemos sentir-nos perdidos. Não sabemos como não interferir com essa
experiência: ``Ah! -- Uau! -- tudo está aberto, claro, espaçoso \ldots{}
e agora, o que é que eu faço?'' O nosso condicionamento diz-nos, ``É
suposto, eu estar a fazer algo. Isto não é progredir no caminho.'' Não
sabemos como estar despertos sem interferir com o espaço dessa
experiência.

Quando esse espaço mental surge, podemos ficar baralhados ou nem
repararmos nele. É como se cada um de nós fosse um ladrão que após
arrombar uma casa, olha em volta e decide, ``Bem, não há muito coisa
para levar, vou continuar a procurar noutro lugar.'' Não compreendemos
que quando há desapego~o \emph{dukkha}~cessa. Em vez disso ignoramos
aquela qualidade serena, aberta, clara e continuamos em busca da próxima
coisa e depois da seguinte e assim por diante. Como diz a expressão, não
``saboreamos o néctar,'' o néctar do sumo de~\emph{rigpa}. Simplesmente
atravessamos o balcão dos sumos a correr. ``Parece que aqui não há nada.
Tudo parece demasiado entediante: nenhuma luxúria ou medo, ou outros
assuntos para tratar''. Assim mantemo-nos ocupados com atitudes do tipo:
``Eu estou a ser irresponsável; deveria ter um objecto no qual me
concentrar ou pelo menos deveria estar a contemplar a impermanência; não
estou a lidar com os meus problemas. Rápido, deixa-me ir para encontrar
algo desafiante para resolver.'' Apesar das nossas melhores intenções,
deixamos de saborear o néctar que se encontra exactamente ali.

Quando deixamos de agarrar algo, surge a verdade última. É tão simples
quanto isto.

Ananda e outro monge estavam a discutir sobre a natureza do estado
imortal e decidiram consultar o Buddha. Eles queriam saber: ``Qual é a
natureza da imortalidade?'' Eles prepararam-se para uma longa e extensa
explanação. Mas a resposta do Buddha foi breve e sucinta. Ele respondeu,
``A cessação do apego é a imortalidade.'' E pronto. Relativamente a este
ponto, os ensinamentos Dzogchen e Theravada são idênticos. Quando o
apego cessa, há \emph{rigpa}, há imortalidade, o fim do sofrimento,
\emph{dukkha-nirodha}.

O primeiro ensinamento do Buddha sobre as Quatro Nobres Verdades trata
disso directamente. Existe uma maneira diferente de tratar cada uma das
quatro verdades. A Primeira Nobre Verdade --~\emph{dukkha}, insatisfação
-- ``deve ser completamente compreendida.'' Precisamos reconhecer:
``Isto é~\emph{dukkha}. Isto não é~\emph{rigpa}. Isto é~\emph{marigpa},
falta de plena consciência, ignorância, e é insatisfatório.''

A Segunda Nobre Verdade, a causa de~\emph{dukkha}, é o desejo egoísta, a
cobiça, a qual deve ser abandonada.''

A Quarta Nobre Verdade, o Nobre Caminho Óctuplo, ``deve ser cultivado e
desenvolvido.''

Mas o que é interessante, especialmente neste contexto, é que a Terceira
Nobre Verdade, \emph{dukkha-nirodha}, o fim de~~\emph{dukkha}, ``deve
ser realizado.'' Isto significa: estejam atentos, quando o
\emph{dukkha}~cessa e, reparem no que acontece. Observem: ``Ah! Tudo
está bem.'' Aqui é quando pomos uma mudança acima -- quando podemos
simplesmente ser, sem devir.

``Ah ah'' -- o sabor do néctar de~\emph{rigpa}~-- ``Ah, isto é
perfeito.''

A realização consciente do fim de~\emph{dukkha}, do vazio e do espaço na
mente são considerados elementos cruciais da prática correcta dentro da
tradição Theravada. Realizar \emph{nirodha}~é de certo modo o aspecto
mais importante ao trabalharmos com as Quatro Nobres Verdades. Parece
secundária, é a menos tangível de todas, mas é aquela que contém a jóia,
a semente da iluminação.

Embora a experiência de~\emph{dukkha-nirodha}~não seja algo, isso não
quer dizer que não haja nada ou nenhuma qualidade. Na verdade é a
experiência da verdade última. Se não estivermos apressados em busca do
próximo feito e tomarmos atenção ao fim de~\emph{dukkha}, abrimo-nos
para a pureza, luminosidade e paz. Ao permitirmos que o nosso coração
desfrute plenamente daquilo que está a acontecer (as chamadas
``experiências normais'') ele floresce e abre-se, lindamente adornado,
tal orquídea dourada, tornando-se cada vez mais claro e luminoso.

\textbf{Não feito disto}

Todos os praticantes budistas, independentemente da sua tradição, estão
familiarizados com as três características da existência -~\emph{anicca,
dukkha, anattā}~(impermanência, insatisfação e não-eu). Elas são o
``primeiro capítulo, primeira página'' do Budismo. Mas no Theravada
também se fala das outras três características da existência, a um nível
mais subtil:~\emph{suññata, tathatā, atammayatā}.~\emph{Suññatā}~é
vacuidade. O termo deriva de dizer ``não'' ao mundo fenomenológico:
``Não vou acreditar nisto. Isto não é completamente
real.''~\emph{Tathatā}~significa ``assim, característica de presença''.
É uma qualidade muito semelhante a \emph{suññatā,} ~mas deriva de dizer
``sim'' ao universo. Não há nada, no entanto há algo. A qualidade de
``característica de ser'' é igual à textura da realidade última.
\emph{Suññatā}~e~\emph{tathatā}~- vacuidade e particularidade ~-- os
ensinamentos expressam-se nestes termos.

A terceira qualidade, ~ \emph{atammayatā}, não é muito bem conhecida. No
Theravada, ~\emph{atammayatā} tem sido mencionada como o conceito
último. Literalmente significa, ``não é feito disto.'' Mas
\emph{atammayatā}~pode ser interpretado de várias e diferentes maneiras,
proporcionando uma variedade de subtis gradientes do seu significado.
Bhikkhu Bodhi e Bhikkhu Ñanamoli (na sua tradução do \emph{Majjhima
Nikāya}) interpretam \emph{atammayatā}~ como ``não identificação'' --
~adoptando~o lado do ``sujeito'' da equação. Outros tradutores
interpretam como ``não fabricar'' ou ``não conceber'', dessa forma
apontando mais para o elemento ``objecto'' da equação. De qualquer modo,
a referência é feita em primeiro lugar à qualidade da consciência
anterior à dualidade sujeito-objecto (ou sem esta).~

As origens ancestrais Indianas deste termo parecem basear-se numa teoria
de percepção sensorial na qual ``a mão que agarra'' proporciona a
analogia principal: a mão assume a forma daquilo que ela apreende. O
processo da visão, por exemplo, é explicado como o olho enviando uma
espécie de onda, que depois assume a forma daquilo que vemos e trá-la de
volta. De modo semelhante para com o pensamento: a energia mental
molda-se ao seu objecto, (isto é, um pensamento), e depois retorna para
o sujeito. Essa ideia está implícita no termo ``\emph{tan-mayatā},''
``consistindo disso.'' A energia mental daquele que experiencia
(sujeito) assume a mesma natureza do objecto detectado.

A qualidade oposta\emph{, atammayatā}, refere-se a um estado no qual a
energia da mente não ``sai'' em direcção ao objecto e o ocupa. Ela não
perfaz nem ``algo'' objectivo, nem um ``observador'' subjectivo que
conhece. Por conseguinte, a não identificação refere-se ao aspecto
subjectivo e a não fabricação refere-se ao aspecto objectivo.

A maneira como em geral a vacuidade é discutida nos círculos
Dzogchen~~deixa bem claro que esta é uma característica da realidade
última. Mas em outros usos da vacuidade e da ``característica de
presença'' (\emph{tathatā}), ainda pode haver a noção de um agente (um
sujeito), que é um ``este'' a olhar para aquilo e esse aquilo é vazio.
Ou, se esse aquilo é assim, desta forma. ~\emph{Atammayatā}~é a
compreensão de que, na verdade, não pode haver nada além da realidade
última. Não há o aquilo. Com o largar, com o completo abandono de
``aquilo'', todo o universo relativo do sujeito-objecto, até mesmo no
seu nível mais subtil, dissolve-se.

Eu gosto particularmente da palavra ``\emph{atammayatā}'' devido à
mensagem que ela transmite. Entre as suas várias qualidades, este
conceito lida profundamente~~com a noção persistente da especulação
incessante, ``O que é aquilo ali?'' Existe aquele indício de que algo
ali pode ser um pouco mais interessante do que o que está aqui. Até
mesmo a noção mais subtil de ignorar isto para obter aquilo, não estar
satisfeito com isto e querer tornar-se aquilo, é um erro.
\emph{Atammayatā}~é aquela qualidade em nós que sabe, ``Não existe
aquilo. Só existe isto.'' Daí, até mesmo o aspecto ``isto'' torna-se
irrelevante. ~\emph{Atammayatā}~ajuda o coração a romper os hábitos mais
subtis de inquietação, bem como acalmar as repercussões da raiz
dualista, sujeito e objecto. Este abandono leva o coração a uma
compreensão: há apenas a completude do \emph{Dhamma}, o espaço pleno e
preenchimento. As aparentes dualidades disto e daquilo, sujeito e
objecto são vistas como essencialmente desprovidas de sentido.

Uma das maneiras que podemos empregar isto num nível prático é com uma
técnica frequentemente sugerida por Ajahn Sumedho. Pensando que a mente
está no corpo, dizemos, ``a minha mente'' (\emph{aponta para a cabeça})
ou ``a minha mente'' (\emph{aponta para o peito}). Não é verdade? ``Está
tudo na minha mente.'' Na verdade entendemos tudo mal. O corpo está na
nossa mente ao invés da mente no corpo, certo?

O que sabemos sobre o nosso corpo? Podemos vê-lo, Podemos ouvi-lo.
Podemos cheirá-lo. Podemos tocá-lo. Onde ocorre a visão? Na mente. Onde
experimentamos o toque? Na mente. Onde experimentamos o olfacto? Na
mente.

Tudo o que sabemos do corpo, agora e no passado, foi conhecido através
da intervenção da nossa mente. Nunca aprendemos nada sobre o nosso
corpo, a não ser através da mente. Portanto, durante toda a nossa vida,
desde a infância, tudo o que sempre aprendemos sobre o nosso corpo e o
mundo ocorreu na nossa mente. Então, onde se encontra o nosso corpo? Não
quer dizer que não exista um mundo físico, mas o que podemos dizer é que
a experiência do corpo e a experiência do mundo ocorrem dentro da nossa
mente. Não ocorre em nenhum outro lugar. Tudo acontece aqui. E neste
``aqui'', a externalidade do mundo e o seu sentido de separação
terminam. A palavra ``cessação,'' (\emph{nirodha}), também pode ser aqui
empregue. Paralelamente a esta interpretação mais familiar, a palavra
também significa ``refrear, restringir, parar'', portanto, significa que
a separação findou. Quando compreendemos que contemos o mundo inteiro
dentro de nós, a sua peculiaridade, a sua diferenciação, é questionada.
Somos então capazes de melhor reconhecer a sua verdadeira natureza.~

Esta mudança de visão é uma pequena ferramenta de meditação bastante
interessante que podemos usar a qualquer momento, como por exemplo, na
meditação a andar. É um mecanismo muito útil pois conduz-nos à verdade
da questão. Sempre que o empregamos viramos o mundo do avesso, pois
somos então capazes de ver que este corpo é de facto apenas um conjunto
de percepções. Isto não nega o nosso livre funcionamento, mas coloca
tudo num novo contexto. ``Tudo acontece dentro do espaço
de~\emph{rigpa}, dentro do espaço da mente que sabe.'' Ao vermos as
coisas desta forma, de repente vemos o nosso corpo, a mente e o mundo a
chegar a um acordo, a uma peculiar realização da perfeição. Tudo
acontece aqui. Este método pode parecer um pouco obscuro, mas algumas
vezes as ferramentas mais abstrusas e subtis podem produzir as mudanças
mais radicais no coração.

\textbf{Investigação reflectiva}

Investigação Reflectiva era outro dos métodos que Ajahn Chah costumava
empregar para manter o conhecimento, ou devemos dizer, manter o
Conhecimento Correcto. Ela envolve o uso deliberado do pensamento para
investigar os ensinamentos, bem~~como certos apegos, medos e esperanças
e especialmente o próprio sentimento de identificação. Ele falava sobre
isto quase~~como se tivesse a dialogar consigo próprio.

O pensamento é frequentemente retratado como o grande vilão nos círculos
de meditação: ``Pois é, a minha mente \ldots{} Se pelo menos conseguisse
deixar de pensar, então seria feliz.'' Mas na verdade, a mente pensante
pode ser o mais maravilhoso dos auxiliares quando usada da forma
correcta, particularmente quando se investiga o sentimento de
individualidade. Quando desprezamos desta forma o uso do pensamento
conceptual, perdemos uma oportunidade. Quando estiverem a experimentar,
a ver ou a fazer algo, coloquem uma questão do tipo: ``O que é que está
consciente dessa sensação? Quem é o detentor deste momento? O que é que
sabe~\emph{rigpa}?''

O uso deliberado do pensamento (investigação reflectiva) pode revelar um
conjunto de assumpções inconscientes, hábitos e comportamentos
compulsivos que realizamos. Isto pode ser muito útil e pode produzir
grandes realizações interiores (insights). Podemos estabelecer a nossa
atenção de forma plena, estável e aberta e depois perguntar: ``O que é
que percebe isso? O que está consciente deste momento? Quem é que está a
sentir a dor? Quem está a ter esta fantasia? Quem está a pensar acerca
do jantar?'' Nesse momento abre-se uma fissura. Milarepa certa vez disse
algo nesse sentido: ``Quando o fluxo do pensamento discursivo é
interrompido, abre-se o portal para a libertação.'' Exactamente da mesma
forma, quando colocamos este tipo de perguntas, é como aplicar uma
punção ao emaranhado nó da identificação, afrouxando os seus fios. Isto
quebra o hábito, quebra o padrão do pensamento discursivo. Quando
perguntamos ``quem'' ou ``o quê'', por um instante a mente pensante
tropeça, fica baralhada. Nesse espaço, antes que ela possa formular uma
resposta ou uma identidade, existe paz e liberdade intemporais. Através
desse espaço pacífico surge a qualidade inata da mente, a essência da
mente. É só através de frustrar os nossos julgamentos habituais, as
realidades parciais a partir das quais inconscientemente determinamos a
existência, que somos forçados a afrouxar o nosso apego e a abandonar a
nossa maneira equivocada de pensar.

\textbf{Medo da Liberdade}

O Buddha disse que o desapego da noção do ``eu'' é a felicidade suprema
(por exemplo no \emph{Udana} II.1 e IV.1). Mas ao longo dos anos
tornamo-nos fãs deste personagem (do `eu' da personalidade), não é?
Ajahn Chah certa vez disse, ``É como ter um amigo querido que conhecemos
durante toda a nossa vida. Companheiros inseparáveis. De súbito vem o
Buddha e diz que vós e o vosso amigo tendes de se separar.'' Isto parte
o coração. O ego fica destroçado. Há um sentimento de perda e
diminuição. Depois vem o sentimento angustiante do desespero.

Para a noção do ``eu'', ``ser'' é sempre definido como ``ser algo''. Mas
a prática e os ensinamentos claramente enfatizam o ser ``não-definido'',
uma consciência sem limites, incolor, infinita, omnipresente -- dêem o
nome que quiserem. Quando ``ser'' fica desta forma indefinido, parece
uma morte aos olhos do ego. E a morte é a pior coisa. Os hábitos
baseados no ego esperneiam com fúria e procuram algo para preencher o
espaço vazio. Qualquer coisa serve: ``Rápido, dêem-me um problema, uma
prática de meditação (\emph{isso} é apropriado!). Ou que tal algum tipo
de memória, um sonho, uma responsabilidade que não realizei; alguma
coisa em relação à qual possa sentir angústia ou culpa, \emph{qualquer
coisa}!''

Experienciei isso várias vezes. Nesse espaço, é como se houvesse um cão
faminto na porta a tentar desesperadamente entrar: ``Por favor,
deixem-me entrar, deixem-me entrar.'' O cão faminto quer saber: ``Quando
é que este sujeito vai dar-me atenção? Ele já está ali sentado há horas
como se fosse um raio de um Buddha. Será que ele não percebe que estou
faminto, aqui fora? Não percebe que está frio e húmido? Não se importa
comigo?''

``Todos os~\emph{sankharas}~são impermanentes. Todos
os~\emph{Dhammas}~são presentes e vazios. Não há nada mais\ldots{} ``
{[}\emph{faz ruídos como um cão faminto abandonado}{]}

Estas experiências proporcionaram alguns dos momentos mais reveladores
na minha própria prática e exploração espiritual. Elas contêm uma fome
tão avassaladora de ``ser''. Qualquer coisa serve, qualquer coisa, só
para ser algo: um fracassado, alguém bem sucedido, um messias, uma praga
para o mundo, um assassino de massas. ``Permita que eu seja algo, por
favor, Deus, Buddha ou quem quer que seja.''

Ao que a sabedoria do Buddha responde, ``Não.''

É preciso termos enormes recursos e força interior para sermos capazes
de dizer ``não'' desta maneira. As súplicas patéticas do ego tornam-se
fenomenalmente~~intensas e viscerais. O corpo pode abanar e as nossas
pernas começarem a contorcer-se para correr. ``Tirem-me daqui!'' Pode
até~~acontecer que os pés comecem a mexer-se em direcção à porta, tão
forte é o anseio.

Neste ponto, estaremos a focar a luz da sabedoria exactamente na raiz da
existência dualista. Esta raiz é uma das fortes. É necessário muito
trabalho para chegar até ela e cortá-la. Podemos, portanto, contar com
bastante fricção e dificuldades quando nos envolvermos neste tipo de
trabalho.

A ansiedade intensa surge. Não se deixem intimidar. Ponham o impulso de
lado. É normal experienciar perda e fortes sentimentos de pesar. Há um
pequeno ser que acaba de morrer. O coração sente uma sensação de perda.
Fiquem com isso presente e deixem que passe. A sensação de que ``algo
será perdido se eu não seguir esta ansiedade'' é a mensagem enganosa do
desejo. Quer, seja um pequeno momento subtil de inquietação quer, uma
grande declaração -- ``Vou morrer de desgosto se não seguir este
anseio!'' -- compreendam que tudo isso não passa de uma enganosa sedução
do desejo.

Há um verso maravilhoso num poema de Rumi que diz, ``Quando é que
ficaste remotamente diminuído por teres morrido?'' Permitam que o
borbulhar do ego nasça e permitam que morra. Depois, espantem-se(!), não
só o coração não foi diminuído, como na verdade está maior, mais
radiante e jubiloso que nunca. Há espaço, contentamento e uma
tranquilidade que não podem ser alcançadas através da cobiça ou da
identificação com qualquer atributo da vida.

Não importa quão genuínos os problemas, as responsabilidades, as paixões
ou as experiências aparentem ser. Não temos de ser nada disso. Não há
nenhuma identidade que \emph{precisemos} ser. Não nos devemos apegar a
nada.

Nota: (1) \emph{Rigpa:} é o conhecimento proveniente da realização da
nossa própria natureza, quando realizamos e sabemos que existe uma
liberdade primordial, anterior à nossa identificação com a mente.

Tradução de Appamādo Bhikkhu
