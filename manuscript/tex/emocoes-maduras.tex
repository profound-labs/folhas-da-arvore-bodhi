\chapter{Emoções maduras}

\textbf{Ajahn Vajiro}

O Buddha fala nos seus ensinamentos sobre os \emph{Brahma Viharas}.
Estes são normalmente traduzidos como o divino ou reinos celestiais,
sendo esta uma interpretação baseada na tradução literal: \emph{Brahma}
-- Deus; \emph{Vihara} -- \emph{habitat}, habitação. Os \emph{Brahma
Viharas} podem ser trazidos dos céus à terra se considerarmos que
enquanto emoções eles impelem e encorajam a transcendência das
limitações da existência humana. Esta ``transcendência da limitação'' é
uma definição de crescimento. Posso agradecer a semente desta ideia a um
amigo que mencionou que os \emph{Brahma Viharas (metta, karuna, mudita}
e \emph{upekkha}) poderiam ser considerados emoções maduras. O que se
segue são algumas reflexões adicionais e não uma análise detalhada e
abrangente dos \emph{Brahma Viharas}, a qual poder ser encontrada em
livros sobre Budismo.

Parece-me claro que emoções são algo que nos move. Penso nelas como algo
que produz, alimenta ou nos dirige ao movimento. Elas providenciam o
combustível que catalisa o movimento ou a acção, em direcção a um
determinado objecto ou situação, ou ao afastamento deste. Movemo-nos e
agimos através do corpo, da fala e da mente e esse movimento é uma
resposta a um estímulo dos sentidos. É nesta resposta que podemos
observar como surgem as emoções. Antes do movimento ocorre o estímulo
dos sentidos -- a isto chama-se contacto. Um sentimento surge e depois
uma percepção -- é aqui que entram as emoções maduras. Em Pali não
existe uma tradução directa para a palavra emoção. Uma emoção é uma
mistura de percepção e de \emph{sankhara} (padrão de hábito), os quais
podem ser conscientemente treinados e educados. Emoções maduras são
aquelas emoções que constituem a resposta de uma pessoa desenvolvida, às
situações da vida.

Por vezes, o objectivo do Budismo é descrito de uma forma que me leva a
pensar que se procura ter um coração frio, sem emoção ou paixão: onde
não há resposta, sentimento, desejo ou motivação. Isto entra em conflito
com a imagem que temos do Buddha, como alguém com uma forte motivação,
com uma compaixão tão forte que o levou a viver uma vida em benefício
dos outros seres humanos.

Emoções maduras também são aquelas emoções que permitem que os outros
amadureçam. Assim, quando alguém age ou responde com uma emoção madura,
os outros seres humanos são levados a transcenderem-se, a crescer para
além das suas próprias limitações. Isto parece abstracto, mas quando
reflectimos sobre a melhor maneira dos pais poderem ajudar os seus
filhos para que amadureçam, vemos que é através da expressão de emoções
maduras.

As quatro ``emoções maduras'', tal como são aqui explicadas, podem ser
compreendidas, na prática, como estando interligadas. Elas são divididas
apenas por conveniência, para que as possamos analisar e explicar. São
como diferentes aspectos do mesmo sítio, diferentes formas de
descrevermos o céu. Descrevemos os seus diferentes aspectos para que
isso nos ajude a encontrar uma maneira de nos apercebermos delas, de
forma a expressá-las na nossa vida.

\emph{Metta} -- amabilidade/bondade -- quando estabelecida
interiormente, encoraja a aceitarmo-nos, bem como aos outros, sendo
assim possível obtermos compreensão. A compreensão implica sabedoria e é
a sabedoria que nos permite encontrar o caminho, para irmos mais além e
abrirmos mão daquilo que limita e obstrui o coração. A bondade, quando
expressa aos outros, ajuda que se aceitem a si próprios. Esta é uma
aceitação emocional instintiva ou do coração, permitindo que as acções
do corpo, da fala e da mente (que são uma resposta ao que é visto como
``os outros'') sejam amáveis -- e não motivadas pelo não gostar, pela
aversão ou pelo medo. O efeito é \emph{Metta} sem limites, radiante e
atraente, aquecendo o coração daqueles que são frios e refrescando o
coração daqueles que são demasiado fervorosos.

\emph{Karuna} -- compaixão -- funciona quando nos apercebemos da dor -
angústias, aflições, agonias, tormentos e nervosismos - dos outros, de
uma forma clara, pois deixamos que essas emoções sejam também parte da
nossa experiência. Passam estas assim a ser algo que se moveu para além
do universo do ignorado e inconsciente, passando a fazer parte do
universo do incluso, aceite e consciente. A compaixão tem a
característica de ``espaço'', contribuindo que as coisas existam tal
como são, que se transformem e que findem. Permite particularmente que a
dor cesse. Isto significa que deve-se ser paciente, sem pressa de querer
forçar o fim da dor ou de tentar livrar-se desta de uma forma intrusiva.

\emph{Karuna} é o lado activo da Sabedoria e o supremo purificador. A
compaixão do Buddha permitiu-lhe perceber que ainda existem coisas que
um ser iluminado pode fazer. Foi a compaixão que o motivou a ensinar
``para o benefício daqueles que tem apenas um pouco de poeira nos
olhos''. Misericórdia pode ser visto como uma forma de compaixão. Não é
uma palavra muito usada porém, evocativa da qualidade do coração que
está disposta a suportar os fardos dos outros, sempre disposta a ajudar
da melhor maneira possível, estando atenta aos pedidos de ajuda e
agindo. Os ``pedidos'' por vezes não são muito óbvios. Podem ser tão
simples como ajudar a preparar um evento, ou limpar o espaço depois
deste. Sempre que reparamos que alguma assistência é necessária e
estamos dispostos a prestá-la, estamos a praticar \emph{karuna.}

\emph{Mudita} é normalmente traduzida como alegria empática. Apreço,
alegria e ``trazer alegria'' são palavras que evocam em mim as
qualidades do coração que são o oposto de inveja e ciúmes, são o oposto
das características que tencionam deitar alguém a baixo, subjugado a um
nível inferior.

\emph{Mudita} implica a total consciência. Precisamos de descriminar, de
ser conscientes, de sermos receptivos à possibilidade de apreciação.
Algo que é particularmente encorajador é sermos conscientes do bem, das
virtudes e da sabedoria nos outros e em nós. O que \emph{mudita} permite
é o surgir de uma aspiração de sermos ou praticarmos as características
positivas que observamos nos outros. Luang Por Sumedho disse que quando
podemos apreciar a beleza de uma rosa em plena abertura, podemos ser
movidos por \emph{mudita}. A sugestão é a de praticarmos a todos os
níveis. Por vezes, quando olhamos para uma rosa, podemos ser apanhados
no chamado ``realismo'' e apenas observar uma flor que irá murchar.
Podemos ser um pouco como ``\emph{Scrooge}'' com ``\emph{bah humbug}''
(\emph{humbug} -- embuste), uma resposta amarga a qualquer sugestão de
que beleza pode ser apreciada sem cair no desejo de possuir ou de nos
apegarmos. O equilíbrio surge quando \emph{upekkha} está presente.

\emph{Upekkha} -- novamente aqui vai primeiro a tradução mais comum --
equanimidade. Prefiro ``serenidade'', com a sugestão implícita de
aceitar a limitação e de nos erguermos acima dela. A frase ``sê sereno
na unidade do todo'' sempre me impressionou como uma bela sugestão para
o meu coração, quando existe frustração com a cadência da vida, com as
limitações do universo, com as minhas próprias limitações ou com as dos
outros. Tem de haver uma aceitação mais consciente da realidade, para
permitirmos que o coração se treine de forma a transcender essa
limitação.

Ao nível mundano, se eu quiser treinar-me a dactilografar com os dez
dedos, primeiro tenho de aceitar que neste momento não tenho essa
habilidade. Só então é que posso honestamente fazer um esforço para
aprender a treinar os dedos e os olhos a trabalhar em conjunto de uma
forma automática. Se eu não estiver disposto a aceitar o facto de que de
momento não tenho essa habilidade e, se ainda assim eu quiser
dactilografar com os dez dedos então posso fingir, mas a única pessoa
que eu estaria a enganar seria a mim próprio. Fazemos isto em grande
escala, quando gostaríamos de ser considerados maduros e realizados, mas
somos incapazes de aceitar as nossas limitações. Podemos fingir ser
maduros quando de facto não estamos seguros quanto às nossas emoções ou
intenções, mas que nos iria conduzir por emoções imaturas e
prejudiciais. No caso de dactilografar com dez dedos não existe nenhum
mal real consequente. No caso da pessoa que finge para si e para os
outros que é crescida, é mais perigoso quer, para si mesma quer, para os
outros.

Os quatro \emph{Brahma Viharas} funcionam em conjunto. Ajahn Buddhadasa
descrevia \emph{upekkha} como aquele que supervisiona os outros três. Em
situações benéficas e harmoniosas \emph{mudita} é a motivação madura do
coração. Se for possível aliviar uma situação onde exista dor ou
desconforto devemos invocar a compaixão (\emph{karuna}). Uma situação
feia e desagradável necessita de \emph{metta}. Aceitação, um dos
aspectos de \emph{metta}, ressoa com a aceitação da limitação que está
implícita em \emph{upekkha} e é por isso que \emph{metta} é um princípio
tão importante.

Para a maioria dos seres humanos, e até mesmo para os animais,
\emph{metta} (tal como podemos encontrar na aceitação que uma mãe tem
pelos filhos) é a primeira emoção que nos possibilita crescer e começar
a amadurecer. Se não houver uma expressão de \emph{metta} para com os
nossos descendentes, particularmente para com uma criança, essa prole
irá morrer rapidamente ou tornar-se-á imatura e pervertida. É esta a
motivação primordial que permite aos jovens amadurecerem. Os jovens
expressam-no através da maneira como abordam e aprendem sobre o estado
no qual se encontram. As crianças pequenas apanham coisas do chão sem
qualquer sentido discriminatório e para grande horror dos adultos,
põe-nas na boca. Nesta acção da criança existe um nível de aceitação
muito primário e uma ausência de descriminação, a qual só opera quando a
criança começa a almejar algo para além dela própria.

A compaixão ajuda-nos a reconhecer as mudanças e desenvolvimentos que
são parte das transformações naturais que ocorrem enquanto somos bebés,
crianças, adolescentes, adultos e idosos. Ela reconhece também a dor da
separação daquilo que nos é conhecido e que faz parte deste processo,
lidando com as mudanças de uma maneira sensível.

\emph{Mudita} permite que desfrutemos da vida; a beleza e as maravilhas
desta estranha experiência de sermos uma vida individual sensível, de
alguma forma misteriosamente ligada ao todo. E quando permitimos que
todo o medo do desconhecido caia por terra, a maravilha do desconhecido
passa a ser apreciada e desfrutada.

O que nos move na vida, a meio de incertezas e mudanças é aquilo que nos
traz liberdade. As nossas intenções movem-nos e são o campo da nossa
maior liberdade. Usar e estimular esta liberdade com sabedoria é o
desafio.

Tradução de Appamādo Bhikkhu
