\chapterAuthor{Ajahn Vajiro}
\chapter{Plasticidade Cerebral}
\tocChapterNote{Ajahn Vajiro}

Foi"-me recomendado um livro sobre plasticidade cerebral: como funciona o
cérebro, como se organiza, como acontece o processo de aprendizagem nos
seres humanos, como ocorrem as sinapses neuronais, como é possível
mudar\ldots{}. Como sabem, durante muito tempo pensava"-se que o cérebro
era uma espécie de mapa fixo e, que as ligações neuronais uma vez
estabelecidas, nunca se alteravam. Recentemente têm sido realizados
estudos que demonstram como cérebro é flexível, sendo inclusive possível
evitar doenças como Alzheimer e outros problemas relacionados com o
envelhecimento, através de dedicação e aprendizagem persistentes e
consistentes. Uma das recomendações é, por exemplo, aprender uma nova
Língua.

Acho isto maravilhoso, pois parte do ensinamento do Buddha consiste em
treinarmos (\emph{Dhamma}) e poder mudar o curso da nossa vida. Esse
treino e essa vontade de treinar é o que tem apoiado a
\emph{buddhasassana} (desapensação do Buddha) ao longo destes 2600 anos.
O Buddha ensinou o \emph{Dhamma"-Vinaya}. Ele revelou o \emph{Dhamma} e
ensinou o \emph{Vinaya}. Quando o Buddha estava próximo da hora da sua
morte Ananda perguntou: ``Quem irá liderar a comunidade, quem nos
ensinará depois do teu \emph{paranibbāna}?'', ao que o Buddha respondeu:
``Deixem o \emph{Dhamma"-Vinaya} ser o vosso professor''.

Para mim, o \emph{Dhamma} tem sido definitivamente inspirador. Lembro"-me
de alguns livros que li quando andava na Universidade: ``Está no aqui,
agora'' -- de Ram Das, o ``\emph{Dhammapada}'' e outros livros de
\emph{Dhamma}; mas foi quando vi o \emph{Vinaya} que a fé surgiu nas
suas diferentes possibilidades da vida. Foi quando vi os monges da
floresta, quando vi como seres humanos praticavam o \emph{Vinaya}, que a
fé brotou verdadeiramente em mim. Vi como uma forma, uma estrutura, na
qual as pessoas persistentemente treinaram durante anos, que as havia
transformado e, como estas por sua vez, no decorrer do tempo,
contribuíram para transportar essa estrutura para as gerações vindouras.

No tempo do Senhor Buddha havia \emph{Arahants}, seres iluminados, que
achavam que não era necessário seguirem o \emph{Vinaya} (até mesmo os
seres iluminados cometem erros!). Porém o Buddha criticou"-os dizendo:
``É em benefício das gerações vindouras que devem sustentar o
\emph{Vinaya}, o treino, para que algo possa continuar no futuro''. O
\emph{Vinaya} leva"-nos ao colo quando somos ``pequenos''; quando
começamos o treino somos encorajados a seguir estritamente todas as
regras, mesmo as menores, da forma mais perfeita possível. Isso irá
apoiar"-nos e proteger"-nos. Depois, conforme vamos ficando mais fluentes
e confiantes no \emph{Vinaya,} o encorajamento não é para colocarmos o
\emph{Vinaya} de lado mas sim para passá"-lo às gerações vindouras para
que o treino possa continuar. O \emph{Vinaya} implica um comportamento
exemplar, quer, da expressão física quer, da verbal. É, extremamente
útil e importante estar conscientes das nossas intenções, conjuntamente
com o treino do \emph{Vinaya}. O \emph{Vinaya} não é só o conjunto de
todas as regras básicas, mas também a maneira como realizamos as nossas
acções.

Lembro"-me claramente de pensar que nunca iria fazer vénias -- ``Eu não
faço vénias a imagens douradas! Definitivamente vamos parar ao inferno
se o fizermos!'' -- e mais tarde ter vontade de aprender a fazê"-las.
Lembro"-me muito bem dessa mudança ocorrer em mim, no momento em que me
apercebi de que, o que mais me interessava era o que estava a acontecer
internamente: o que significava o acto de baixar a cabeça perante alguém
ou alguma coisa, e de que forma isso me afectava. Estava interessado em
investigar a resistência que surgia e, mais tarde, a alegria e a vontade
de o fazer. Existe algo de agradável quando nos entregamos e dizemos
``Não depende de mim! Simplesmente quero aprender com isto''.
Particularmente quando faço vénias ao senhor Buddha, existe em mim uma
forte noção de querer aprender com isso, através desta forma na qual
pratico neste preciso momento.

Uma das coisas que fazemos aqui no mosteiro é cantar em conjunto. Para
alguns de vós os cânticos talvez pareçam bastante caricatos. O formato
dos cânticos da manhã e da noite não existia na altura do Buddha. Ele
foi na verdade compilado recentemente (em termos da história do Budismo)
por Ajahn Buddhadhassa (o primeiro professor do Ajahn Sukkhacitto), no
sul da Tailândia. Ele compilou"-os de forma bastante elaborada e também
os traduziu para tailandês. Até essa altura não existiam cânticos em
tailandês. Quando ouvi dizer que os cânticos tinham sido traduzidos para
tailandês (na altura ainda não me tinha ordenado) lembro"-me de pensar:
espero muito seriamente que os cânticos nunca sejam traduzidos para
Inglês! Pois acho que vai soar muito estranho! Eventualmente os cânticos
foram também traduzidos para Inglês e, na realidade, ainda soa bastante
estranho!

Quando pensamos sobre o significado das palavras dos cânticos, também
isso pode parecer um pouco peculiar. Apesar de existirem partes que são
bastante bonitas, há outras onde existem aparentes incoerências, como
dizermos que cada uma das jóias da Jóia Tripla é o nosso único refúgio.
No entanto isto dá"-nos uma indicação da possibilidade de nos entregarmos
completamente a um refúgio. De cada vez que dizemos, ``Este é o meu
único refúgio'' podemos dizê"-lo com todo o coração. Na verdade isto é
necessário. Se queremos confiar neste processo então ele tem de ser
levado a cabo com todo o coração. Tem de haver a noção de que isto é o
mais importante. Se realmente quisermos voltar atrás e programar e
alterar as conexões neuronais da mente, temos de ter em atenção que uma
das coisas que os cientistas dizem é que se fizermos algo sem muita
atenção, sem nos entregarmos por inteiro ao que estamos a fazer, então
as conexões neuronais não se alteram, os neurónios não se reorganizarão;
o processo não ocorre por completo. Deve haver uma total entrega para o
cérebro realmente mudar. Para aprendermos algo de novo temos de ter
completa atenção, temos de estar totalmente presentes. Tarefas múltiplas
não resultam para este tipo de alterações cerebrais. Temos de estar
focados e ser persistentes. É realmente interessante quando estamos a
fazer algo estarmos completamente entregues a isso. Muitos dos famosos
professores dir"-vos-ão que esta transformação não ocorre se não
estiverem inteiros no que estão a fazer. Tem na verdade de ser uma
``questão de vida ou de morte''. Para que isto realmente funcione tem de
haver ``suor e lágrimas''. E certamente, se quiserem aprender os
cânticos e saberem"-nos de cor, têm de permitir que eles vos transformem.
Deve haver um verdadeiro empenho envolvido no processo.

Para mim aprender os cânticos de cor tem sido algo muito revelador.
Lembro"-me de quando era um jovem adulto estudante, pensar que decorar
coisas era estúpido, ``não"-inteligente'', como um tipo inferior de
aprendizagem. No meu tempo era considerado o que pessoas estúpidas
faziam. Também achava não ser capaz de o fazer. Lembro"-me claramente de
pensar ``Nunca serei capaz de decorar nada''. E claro, por essa altura
também já havia esquecido do pouco de poesia que havia decorado. Quando
me tornei parte da comunidade monástica também pensei muito sinceramente
que decorar os cânticos era impossível mas decidi que pelo menos iria
tentar. E assim o fiz. Aprendi o mais fácil dos cânticos -- \emph{Yan
dunnimitam avamagalañca Yo cāmanāpo} -- qualquer pessoa que folheie o
livro dos cânticos pode ver que este é um dos cânticos mais curtos que
podemos possivelmente aprender e que consiste em três repetições nas
quais apenas temos de alterar uma palavra. Portanto aprendi-o. Levou"-me
um dia completo de obsessão para o aprender e percebi que o conseguia
fazer. Foi um começo. Decidi então decorar algo que queria aprender e o
seguinte, penso ter sido o \emph{Sutta Karanyametta}. Nessa altura eu
era \emph{Anagarika}\footnote{%
  \emph{Anagarika}: no Budismo Theravada um(a) \emph{Anagarika(ā)}
  (lit. ``sem lar'') é aquele que deixou a maioria das suas posses e
  responsabilidades para se dedicar a tempo inteiro à prática budista. É o
  primeiro estágio da ordenação monástica, anterior à ordenação de monge
  noviço. O \emph{Anagarika} vive sob os Oito Preceitos.
}.
Comecei a perceber o que é necessário para
decorar algo e uma das primeiras coisas com que me deparei foram os
obstáculos que nos impedem de o fazer. Isso é \emph{Dhammanusati}, tal
como já devem ter ouvido, isto são, ``As quatro bases da plena atenção''
e o primeiro aspecto de \emph{Dhammanusati,} que é bastante claro, são
os obstáculos. E o que é que impede de concentrarmo"-nos em algo? É muito
interessante. Começamos a observar a tendência para sentir aversão, para
a distracção, para a inquietação - pensando que não sabemos o que fazer
- simplesmente querendo ir dormir. Podemos então ver tudo isso de forma
bastante clara e começamos a perceber como a mente funciona.

Comecei também a perceber como as coisas se interligam na mente, como a
memória se processa e como é que as informações são assimiladas. As
pessoas acham que eu sei bastante bem a maior parte dos cânticos, mas
isso não é completamente verdade. Sei um pouco de alguns deles. Coloco
uma enorme quantidade de esforço em algumas coisas. Não acho fácil,
nunca achei, mas sei o que é necessário para o fazer. Se assumo esse
compromisso então sei que pode ser feito. Já não tenho essa dúvida.
Simplesmente sei que requer esforço da minha parte e confio nisso,
apesar de não ser necessariamente fácil para mim. Faço esse esforço e
estou preparado para o fazer, pois sei que quando o realizamos, algo de
facto muda e isso é pode ser passado às gerações seguintes. Somos assim
transportados pela estrutura e podemos então descontrair.

Desta forma os cânticos transportam"-nos e existe algo de muito bonito
acerca de sermos ``levados ao colo'' por uma estrutura. Não nos irá
carregar para sempre, pois se não pusermos esforço e energia nisto,
cairá por terra. Assim, existem alturas em que temos realmente de nos
entregar e é muito bonito quando o fazemos, quando estamos presentes e
nos entregamos ao significado dos cânticos. E isso também ajuda,
esperamos, a que outros levem esse treino adiante, para o futuro, para
que continue para as gerações vindouras. Trata"-se de uma forma de
retribuição. Existe algo de maravilhoso acerca de dar para as novas
gerações algo que nos foi passado.

Se olharem para as palavras do cântico desta noite vêm que existe um
significado muito bonito nesta reflexão acerca da \emph{Jóia Tripla}, o
relembrar do supremo louvor. Existe algo de extrema beleza pois trata"-se
do mais belo refúgio para seres humanos, o refúgio na \emph{Jóia
Tripla}, \emph{Buddha, Dhamma e Sangha} -- não existe outro refúgio
igual. ``O Buddha é o meu excelente refúgio''. Se estiverem sonolentos
durante a meditação, ou se a mente estiver a deambular demasiado --
talvez nunca fiquem sonolentos durante a vossa meditação ou talvez a
vossa mente nunca divague e esteja sempre supremamente presente! Mas
caso aconteça, assim como por vezes acontece à minha mente\ldots{} -- é
de grande ajuda termos algo que sabemos de cor, algo que deixamos que
nos transforme e que ajuda o coração a elevar"-se. Relembrarmo"-nos das
qualidades da \emph{Jóia Tripla} é um dos quarenta objectos para
meditação mencionados no \emph{Visuddhi Magga} e nos \emph{Suttas}. A
contemplação do \emph{Dhamma} é uma forma básica de ultrapassar a
sonolência, recordando as partes que sabemos de cor mentalmente, de trás
para a frente. Vejamos então, por exemplo, se conseguimo"-nos lembrar das
nove qualidades do Buddha, das seis qualidades do \emph{Dhamma,} dos
quatro pares de seres nobres, e de inclusive percorrer com a nossa mente
todos estes temas; isto se tivermos a tendência para divagar durante a
meditação ou se acontecer por vezes ficarmos um pouco sonolentos. Ter
estes temas presentes ajuda"-nos e pode ser uma maneira de mudar o nosso
estado de ânimo ou outras características da nossa \emph{chitta}
(coração"-mente), a \emph{cittavisaddhi} (características do nosso
coração"-mente). Podemos assim modificá"-las imediata e completamente.
Claro que também existem outras maneiras de o fazer como a respiração ou
a postura, mas podemos fazê"-lo com algo que aprendemos de cor, que temos
no nosso coração e podemos contemplar. É algo de grande ajuda.

Na contemplação de tudo isto o refúgio supremo é o Buddha, daí ser
mencionado em primeiro lugar -- consciência, é o que o Luang Por Sumedho
enfatiza sempre. Isso torna tudo claro pois se não houver um
conhecimento do que está a acontecer haverá dor. É a \emph{Génese
Dependente} -- \emph{Paticcha Samupadda}. Se há ignorância há
\emph{dukkha} (insatisfação, sofrimento). Se não soubermos o que está a
acontecer, se não houver clareza acerca do que está a acontecer, então
irá haver \emph{stress}. Se estivermos conscientes do que está a
acontecer então as possibilidades de haver stress são poucas, muito
menores, ou pelos menos não desenvolvemos mais (\emph{stress}).

A dúvida é algo frequentemente problemática: não sabermos o que está a
acontecer, estarmos incertos, não estarmos claros, não estarmos seguros
sobre nós próprios ou sobre o que fazer. A maior parte de nós tenta
ultrapassar o estado de dúvida tentando encontrar certezas: ``Quero ter
a certeza, quero ter a total certeza, quero ter a certeza de que é
absolutamente certo, quero ter a certeza que é a opção certa''. Mas até
mesmo no simples caso de aprender um cântico haverá alturas em que não
estamos seguros qual é a próxima palavra. Posso dizer"-vos que na minha
experiência, quando estamos a cantar sem o livro (porque se estamos a
falar em aprender os cânticos de cor não faz sentido utilizar"-se o
livro), e temos a noção de que não sabemos o que vem a seguir (e de
forma alguma vamos ter a certeza pois sem o livro\ldots{}), se
repousarmos na nossa consciência por um breve momento, o que vem a
seguir será revelado. Se o tivermos aprendido será revelado. Não
ultrapassamos o estado de dúvida através de querermos agarrar a certeza.
Ultrapassamos o estado de dúvida através de uma entrega à consciência.
Não se extingue a dúvida procurando que as coisas sejam de determinada
forma. Tentar que as coisas sejam de determinada forma é uma história
sem fim. A única maneira de ultrapassarmos a dúvida é confiar na
consciência e isso é assustador. Até mesmo quando estamos a cantar o
\emph{pāṭimokkha}\footnote{%
  \emph{Pāṭimokkha}: é o código básico da disciplina monástica
  Theravada, consistindo em 227 regras para os monges (\emph{bhikkhus}) e
  de 311 para as monjas (\emph{bhikkhunis)}. Encontra"-se no
  \emph{Suttavibhanga}, uma das divisões do \emph{Vinaya Pitaka}. É
  recitado pela comunidade monástica na lua cheia e na lua nova.
}
é um pouco assustador, ainda que tenhamos alguém a
fazer o ponto. Temos de ser capazes de entrar naquele estado em que
simplesmente descansamos na sabedoria da consciência. Confiar. Esta é
uma forte recomendação para todos vós: não encontrarão uma saída para a
dúvida através da tentativa de encontrar certezas.

Acho que é tudo por hoje e espero que isto vos encoraje a mudar essas
células cerebrais! Não tenham medo de ficar com falta de espaço no vosso
cérebro por o encherem com os cânticos. Na verdade o que acontece
(segundo os cirurgiões cerebrais) quando aprendemos algo bem,
tornamo"-nos mais eficientes. Quando aprendemos algo correctamente
estamos a proporcionar a possibilidade do cérebro utilizar um menor
número de células cerebrais para executar as tarefas necessárias. Quando
não estamos seguros por não termos decorado bem os cânticos (neste
caso), estamos na realidade a utilizar mais células cerebrais e a criar
mais caos. Assim, quando aprendemos algo devidamente são necessárias
menos células cerebrais para tornar as coisas claras. Este livro
recomendado pelo Ajahn Jayassaro relata os vários tipos de experiências
que os cirurgiões cerebrais efectuaram com variados tipos de pessoas.
Tenho estado a contemplar isto e a perceber que faz sentido. O cérebro é
plástico. Através de treino, o cérebro pode tornar"-se mais eficiente e
eu desejo"-vos o melhor no vosso treino para a total libertação de toda a
confusão.

\emph{Evam.}

\bigskip

{\raggedleft\itshape
  Tradução de Appamādo Bhikkhu
\par}
