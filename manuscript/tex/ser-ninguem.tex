\chapterAuthor{Ajahn Sumedho}
\chapter{Ser Ninguém}
\tocChapterNote{Ajahn Sumedho}

Tentem  reparar nos pequenos fins da vida dando particular atenção ao
fim da expiração. Desta maneira, na vida diária, podemos reparar nos
pequenos fins tão comuns, que ninguém nota. Tenho descoberto que esta
prática é uma mais-valia, pois é uma forma de reparar na natureza
mutável do reino condicionado no qual vivemos o nosso dia-a-dia. Tanto
quanto percebo, estes são os estados mentais normais que o Buddha
apontava e não os estados de concentração sobre-desenvolvidos.

No primeiro ano da minha prática, estava por minha conta e conseguia
alcançar estados mentais de concentração elevada, dos quais gostava
muito. Depois fui para Wat Pah Pong (Tailândia), onde a ênfase era
baseada em rotinas e viver de acordo com a disciplina do \emph{Vinaya}.
Ali tínhamos de ir à ronda da mendicância todas as manhãs, e fazer os
cânticos da manhã e os cânticos da noite todos os dias. Se somos novos e
saudáveis, então é esperado que se vá a muitas e longas rondas da
mendicância - havia também pequenos percursos que os monges mais velhos
e frágeis podiam fazer. Nesses dias, eu era muito vigoroso, e então
participava sempre nessas extensas rondas da mendicância, regressando
assim cansado. Ao chegarmos tínhamos então a refeição e à tarde havia
tarefas para fazer. Não era possível, sob essas condições, permanecer
num estado concentrado. A maior parte do dia ficava tomada pelas rotinas
diárias.

Cansei-me de tudo isso e fui procurar Luang Por Chah (Ajahn Chah) e
disse-lhe: ``Não consigo meditar aqui!'' e ele começou a rir-se de mim e
a dizer a todos que ``o Sumedho não consegue meditar aqui!'' Eu estava a
ver a meditação como sendo essa experiência muito especial que tive e
gostei muito; Luang Por Chah estava obviamente a mostrar-me que as
rotinas diárias - o acordar, a ronda da mendicância e as tarefas - tudo
isto servia para praticar a plena atenção. Ele não me pareceu, de todo,
preocupado em apoiar a minha vontade de não fazer todas as pequenas
tarefas diárias, motivado pela privação das fortes experiências
sensoriais e, portanto, acabei por me conformar e aprender a meditar na
comum e rotineira vida diária. Porém, a médio prazo, isso tornou-se
muito útil.

A vida nem sempre era como eu queria, pois normalmente queremos algo
especial: adoramos ter fantásticas experiências de cor, maravilhosas
realizações multicoloridas e incríveis momentos de felicidade, êxtase e
arrebatamento -- não apenas estar feliz e calmo mas sim estar para lá do
céu!

Mas, reflectindo na vida nesta forma humana - é assim mesmo - trata-se
de termos a capacidade de nos sentarmos pacificamente e ficarmos
contentes com aquilo que temos; é isso que faz com que a nossa vida seja
uma experiência quotidiana alegre e não de sofrimento. E é deste modo
que a maior parte da nossa vida pode ser vivida -- não se pode viver em
estados de euforia e êxtase a lavar pratos, não é? Costumava ler sobre
as vidas dos santos que ficavam tão arrebatados em êxtases que não
podiam fazer nada num nível mais prático. Ainda que o sangue fluísse das
suas palmas e pudessem fazer as façanhas que fariam com que as pessoas
de fé ficassem pasmadas só de olhar, no que toca a coisas práticas ou
realistas, eram bastante incapazes.

E apesar de tudo, quando contemplamos a disciplina do \emph{Vinaya} em
si mesma, vemos que ela é um treino em vigilância. É a vigilância que
deve estar presente quando fabricamos os mantos, recolhemos as ofertas
de alimentos, comemos ou tomamos conta do \emph{kuti} (abrigo), e que
nos indica o que fazer nesta ou naquela situação. É um conselho muito
prático sobre a vida diária de um \emph{bhikkhu}. Um dia vulgar na vida
de Bhikkhu Sumedho não é explodir em êxtase mas sim acordar e ir à
casa-de-banho, tomar banho, colocar o manto, fazer isto e aquilo; é
somente estar vigilante enquanto estamos vivos nesta forma e aprender a
despertar para o modo natural das coisas, despertar para o
\emph{Dhamma}.

É por isso que quando contemplamos a cessação não procuramos o fim do
universo, mas apenas a exalação da respiração, ou o final do dia, ou o
fim do pensamento, ou o fim do sentimento. Dar atenção a isto significa
que prestamos atenção ao fluxo da vida -- temos mesmo que reparar na
maneira como este fluxo é, em vez de esperar por uma fantástica
experiência de luzes maravilhosas descendo sobre nós, fazendo
\emph{zapping} ou outra coisa do género.

Agora, contemplem apenas a respiração normal do corpo. Irão reparar se
estão a inspirar; é fácil concentrarmo-nos nisso. Quando enchemos os
pulmões, temos uma sensação de crescimento,  desenvolvimento e força.
Quando dizemos que alguém está ``inchado de orgulho'' então
provavelmente está a inspirar. É difícil sentir-se inchado enquanto se
expira. Expanda o seu peito e terá a sensação de ser alguém grandioso e
poderoso. Quando comecei a prestar atenção à expiração, a minha mente
vagueava; expirar não me pareceu tão importante como inalar -- exalamos
só para chegar à próxima inspiração.

Reflictamos: podemos observar a respiração, mas o que é aquele que
observa? O que é isso que observa e conhece a inspiração e a expiração
-- não é a respiração, pois não? Podemos observar o pânico que surge
quando tentamos respirar e não conseguimos; mas o observador, aquele que
sabe, não é uma emoção, não é o ataque de pânico, não é expiração ou
inspiração. Então, o nosso refúgio no Buddha é ser esse saber; ser a
testemunha em vez de ser a emoção, a respiração ou o corpo.

Com o som do silêncio, algumas pessoas ouvem flutuações de um som ou um
som de fundo contínuo. Então podemos contemplar esse som - dar-lhe
atenção - conseguem reparar nele se puserem os dedos nos ouvidos?
Conseguem ouvi-lo num lugar onde alguém está a usar uma moto-serra? E
enquanto fazem exercícios? Ou quando se encontram num estado emocional
frágil? Podem usar este som do silêncio como algo que vos recorda que
devem voltar a vossa atenção e reparar nele -- porque está sempre
presente aqui e agora. E aí está ``aquilo que'' presta atenção.

Existe na mente o desejo de lhe chamar algo, de ter um nome para isso,
tê-lo listado como uma realização ou projectar algo sobre isso. Reparem
na tendência em querer torná-lo em algo. Alguém disse que é
provavelmente o som do sangue a circular no ouvido, outros chamam-lhe o
``som cósmico'', ``a ponte com o Divino.'' Isto soa melhor que ``o
sangue no ouvido.'' Pode ser o som do Cosmos ou até pode ser que tenha
uma doença no ouvido. Mas não tem que ser nada; é aquilo que é, é ``como
é.'' O que quer que seja, pode ser usado como reflexão porque quando
está presente, não existe o sentido de si próprio, existe vigilância,
existe a capacidade de reflectir.  

É mais como um caminho a direito sobre o qual podemos andar, e que não
nos deixa desequilibrar.  É algo que pode ser usado para o nosso estado
consciente na vida diária, enquanto vestimos o manto, escovamos os
dentes, fechamos uma porta, quando entramos na sala de meditação ou
quando nos sentamos. Muito da vida diária é apenas ``normal'' porque
ansiamos por aquilo que consideramos serem as coisas importantes da vida
- como por exemplo a meditação. Assim, caminhar de onde se vive até à
sala de meditação pode ser uma experiência completamente descuidada --
apenas um hábito -- \emph{clop, clop, clop, pum bang}! Então sentamo-nos
por uma hora, tentando ser vigilantes.

Deste modo, começamos a perceber uma maneira de estar vigilante, de
trazer a vigilância para as actividades rotineiras e para as
experiências da vida. Tenho uma pequena imagem no meu quarto - à qual
sou muito afeiçoado - de um velho homem com uma caneca de café na sua
mão, olhando pela janela para um jardim inglês, enquanto chove. O título
da imagem é ``Esperando.'' É assim que penso em mim: um velho homem com
a caneca do café, sentado na janela, esperando, esperando... observando
a chuva, o sol ou outra coisa. Não vejo uma imagem depressiva mas antes
uma imagem pacífica. Esta vida é apenas sobre esperar, não é? Esperamos
o tempo todo -- e então reparamos nisso. Não esperamos por alguma coisa
em particular  mas podemos simplesmente esperar. Então respondemos às
coisas da vida, ao momento do dia, aos deveres, às coisas em movimento e
mudando, como a sociedade em que estamos. Essa resposta não vem da força
do hábito, da ganância, ódio ou ilusão mas é a resposta da sabedoria e
atenção.

Quantos de vós sentem que têm uma missão na vida? Algo que têm que
realizar, um certo tipo de tarefa importante que vos foi atribuída por
Deus, pelo destino ou algo. As pessoas são frequentemente apanhadas
nesse ponto de vista de serem alguém com uma missão. Quem consegue estar
presente com as coisas como elas são, tal como o corpo que cresce, fica
velho e morre, respira e é consciente? Podemos praticar, viver dentro
dos preceitos morais, fazer o bem, responder às necessidades da vida com
plena atenção e sabedoria -- mas não existe alguém que tenha que fazer
alguma coisa. Não existe alguém com uma missão, ninguém especial; não
estamos a fazer uma pessoa, ou um santo, ou um \emph{avatar}, um
\emph{tulku} ou um messias, ou Maitreya. Mesmo que pensemos: ``não sou
ninguém'', mesmo ``sendo ninguém'' é ser alguém nesta vida, não é?
Podemos sentir-nos orgulhosos de ``sermos ninguém'' tal como ``sermos
alguém'' e estar ilusoriamente apegados a ``ser ninguém''. Mas
independentemente daquilo em que acreditamos, que somos ninguém ou
alguém, que temos uma missão, que somos um estorvo e um peso para o
mundo, ou como nos vemos a nós mesmos, então o saber estará lá para ver
a cessação de tal percepção.

As percepções aparecem e desaparecem, não é? ``Sou alguém, sou uma
pessoa importante na vida'': isso começa e termina na mente. Repare no
finalizar de ``ser alguém importante,'' ou de ``não ser ninguém'' ou
qualquer outra coisa -- tudo acaba, não é? Tudo o que surge, cessa e com
essa consciência não nos agarramos à ideia de sermos alguém ou ninguém.
É o fim de toda essa massa de sofrimento -- de ter que desenvolver algo,
tornar-se alguém, mudar alguma coisa, fazer com que tudo esteja no sítio
certo, livrar-se dos obstáculos interiores ou salvar o mundo. Mesmo os
melhores ideais, os melhores pensamentos podem ser vistos como
\emph{dhammas} que surgem e cessam na mente.

Podemos pensar que isto é uma filosofia estéril sobre a vida porque
existe muita emoção e sentimento em ser-se alguém que vai salvar todos
os seres sencientes. As pessoas com auto-sacrifício, que têm missões,
que ajudam os outros e que têm algo importante a realizar, são uma
inspiração. Mas quando observamos isso sob o \emph{Dhamma}, vemos as
limitações das aspirações e a sua cessação. Existe o \emph{Dhamma} de
tais aspirações e acções, em vez de alguém que tem que se tornar em algo
ou fazer alguma coisa. Toda a ilusão é abandonada e o que sobra é a
pureza da mente. Então a resposta à experiência vem da sabedoria e
pureza em vez das convicções pessoais e missões com o seu sentido de si
mesmo e dos outros, e todas as complicações que advêm de todo esse
padrão ilusório.

Podemos confiar nisto? Podemos confiar em largar tudo, cessar, não ser
ninguém, não ter qualquer missão, não ter que se tornar em coisa alguma?
Conseguimos confiar nisso ou sentimos que é assustador, árido ou
depressivo? Talvez queiramos mesmo uma inspiração. ``Diga-me que tudo
está bem; diga-me que gosta de mim, que o que faço está correcto e que o
Budismo não é uma religião egoísta onde nos tornamos iluminados para o
nosso próprio proveito; diga-me que o Budismo está aqui para salvar
todos os seres vivos. É isso que você vai fazer, Venerável Sumedho? Você
é mesmo Mahayana ou é Hinayana?''

O que estou a salientar é que a inspiração é uma experiência. Idealismo:
não tentar rejeitá-la ou julgá-la de qualquer maneira, mas reflectir
sobre ela, conhecer o que está na mente e como é fácil ficar iludido
pelas nossas próprias ideias e opiniões elevadas. Também podemos ver,
como podemos ser insensíveis, cruéis e indelicados devido ao apego que
temos a pontos de vista sobre ser-se gentil e sensível. É aqui que está
a verdadeira investigação do \emph{Dhamma}.

Recordo-me da minha própria experiência: sempre tive a sensação de que
de alguma forma eu era alguém especial. Costumava pensar: ``Bem, eu devo
ser uma pessoa especial. Há muito tempo, quando eu era criança, era
fascinado pela Ásia e assim que pude, estudei chinês na universidade;
portanto certamente devo ter sido uma reincarnação de alguém ligado ao
Oriente.''

Mas considerem isto como uma reflexão: não interessa quantos sinais de
ser alguém especial, ou de vidas passadas das quais nos possamos
recordar, ou das vozes de Deus, ou mensagens do Cosmos, o que quer que
seja - não para negar que essas coisas não são reais - mas que são
impermanentes. Elas são \emph{anicca, dukkha, e anatta}. Reflectimos
sobre elas como elas realmente são -- o que surge cessa: uma mensagem de
Deus é algo que chega e cessa na mente, não é? Deus não está
continuamente a falar com alguém a menos que queira considerar-se o
silêncio como voz de Deus. Então, isso não quer dizer muito, pois não?
Podemos chamar ao ``som do silêncio'' o que quisermos -- voz de Deus, ou
do Divino, chamada do Cosmos, o sangue a passar nos tímpanos, etc. Mas o
que quer que seja, poderá ser usado para desenvolver a vigilância e a
reflexão -- é para isso que estou a apontar, como usar estas coisas sem
as tornar em algo.

Assim, as missões que temos são respostas naturais que acontecem na
nossa vida -- elas deixam de ser pessoais. Já não se trate de mim,
Sumedho Bhikkhu, com uma missão, como se eu fosse especialmente
escolhido de cima, mais do que qualquer um de vós. Deixa de ser isso.
Toda essa percepção e maneira de pensar são abandonadas. E quer eu salve
ou não o mundo, ou milhares de seres, ou ajude os pobres nos bairros de
Calcutá, ou ajude a curar leprosos, e faça todo o género de bons
trabalhos - não é por ter a ilusão de ser uma pessoa, mas como uma
resposta natural da sabedoria.

Nisto eu confio; é isto que é \emph{saddha} -- é a fé na palavra do
Buddha. \emph{Saddha} é a real fé e confiança no \emph{Dhamma}; em
esperar ``ser ninguém'' e não se tornar em algo mas ser capaz de apenas
esperar e responder. E não há muito para responder, apenas esperar --
caneca de café, observar a chuva, o pôr-do-sol, o envelhecer,
testemunhar o envelhecimento, as entradas e saídas do mosteiro -- as
ordenações e as desistências, as aspirações e depressões, os altos e os
baixos, dentro da mente, fora do mundo. E ocorre uma resposta porque
temos vigor, inteligência e talento, e então a vida pede-nos para
respondermos de forma hábil e com compaixão, algo que somos propensos e
capazes de fazer. Gostamos de ajudar os outros. Não nos importamos de ir
para uma colónia budista de leprosos -- eu iria de bom grado -- ou
trabalhar nos bairros pobres de Calcutá ou algo assim; eu não teria
objecções - esse tipo de coisas, apela bastante ao meu sentido de
responsabilidade!

Mas não é uma missão, não sou eu que tenho de fazer algo; é a confiança
no \emph{Dhamma.} Assim, a resposta à vida torna-se clara e benéfica
porque não surge de mim como pessoa com as suas ilusões provindas da
ignorância condicionando as formações mentais. Assim observamos a
inquietação, as obsessões da mente e deixamo-las findar. Permitimos que
partam e assim elas cessam.

\bigskip

{\raggedleft\itshape
  Tradução de Appamādo Bhikkhu e Pedro Cruz
\par}
