\chapterAuthor{Ajahn Vimalo}
\chapter{Silenciosa Atenção e Mente"-Espelho}
\tocChapterNote{Ajahn Vimalo}

Tenho fé numa maneira de ser silenciosa, na presença intuitiva da mente, numa
abertura mental onde os pensamentos ocorrem, mas não são levados demasiado a
sério. Tenho fé numa total atenção. Acredito que se a mente for conduzida a um
estado de silenciosa atenção, então, transforma"-se num espelho que reflecte a
própria natureza.

Fé em ser consciente é aprender a permanecer e apreciar o estar
presente. Esta é a minha prática. Quando as coisas se tornam difíceis,
sento"-me com os olhos abertos e trago a minha mente para o presente,
tanto quanto possível. Tento sentir a mente. Isto pode"-se tornar numa
experiência muito bonita. Ainda que a consciência seja algo momentâneo,
surgindo a cada momento pelo contacto da visão com os objectos visíveis
(por exemplo), podemos apreciá"-la na nossa experiência.

Isto pode trazer paz e alegria às nossas vidas. As pessoas podem"-se
focar naquilo que é negativo. Podemos fazer isso em relação a nós
próprios e depois ficarmos deprimidos. Este exame da presença de mente é
algo diferente. A mente presente é a ``porta para a realidade imortal''.
É a plenitude da mente, o vazio da mente. Existe algo nesse estado que
não podemos caracterizar. Tal como um espelho que permite que tudo nele
se reflicta. Se conseguirmos permanecer ou funcionar nessa dimensão
estamos a Caminhar. Existe a seguinte expressão no \emph{Dhammapada}:
``A atenção consciente é o caminho para a realidade imortal''. É disto
que falo, a mente presente, plenamente atenta. Quando estamos
conscientes e presentes, não ignoramos nada. O ensinamento do
\emph{Paticca"-samuppada} (Génese Dependente) explica que as coisas
surgem como resultado da ignorância. Neste estado mental presente não
estamos a ignorar, existe uma observação directa da experiência.

As pessoas reclamam ``Tenho de cultivar os factores para a iluminação'',
`Não consigo chegar a lugar nenhum porque tenho de cultivar paciência e
outras coisas mais. Mas quando permanecemos presentes e nos entregamos
mais profundamente a esse estado, surge a alegria. Isto acontece não por
estarmos intensamente concentrados a tentar desenvolver alegria; pelo
contrário estamos a entregar"-nos a esse estado de presença de mente. E
com essa presença - conforme ela vai"-se tornando mais bonita -- surge a
alegria.

A dada altura, em vez de tentarmos descontrair nessa presença começamos
a ficar conscientes de que ela existe e que tudo surge dentro dela. Esta
é outra dimensão. Acontece naturalmente. Então, seja o que for que
estejamos a fazer, quer estejamos parados ou a andar, a movermo"-nos
rápida ou lentamente, as coisas são algo diferente, o mundo é algo
diferente. Não estou a dizer que atingi alguma coisa. Não atingi coisa
alguma. Estou apenas a partilhar a minha compreensão daquilo que é a
plena atenção.

Quando a mente é consciente torna"-se bastante evidente o que é bom ou
mau para nós. Coisas que o Buddha ensinou começam a fazer sentido e, não
temos de pensar sobre elas, pois de súbito, simplesmente compreendemos,
realizamos. Quanto mais confiarmos na consciência mais as coisas se
resolvem por si; tudo se resolve por si, até os nossos interesses
pessoais desaparecerem. Portanto, encorajo todos a terem fé em
transformarem"-se nessa presença consciente.

No livro ``Tales of Power'' aparece a descrição do que acontece quando
rochas esmagam o filho de Don Juan. Ele conta como viu o corpo do seu
filho em agonia. Quando olhou para o seu filho, Don Juan refere:
``Alterei a minha maneira de ver. Não vi o meu filho a morrer. Se
tivesse pensado no meu filho teria visto o seu belo corpo esmagado e um
grito de dor teria escapado do meu interior. Mas ``mudei a minha visão''
e portanto observei a sua vida pessoal a desintegrar"-se no infinito.
Pois que esta é a forma como a vida e a morte se entrelaçam. Não vi o
meu filho, vi a sua morte. E a sua morte foi igual a tudo o resto''.
Quando isto aconteceu Don Juan alterou a sua percepção desviando"-se da
percepção de ``filho''; e nessa alteração a sua mente transformou"-se num
espelho. Ele viu tudo aquilo de forma completamente diferente, mais
perto da realidade, tal com é.

Através de um constante reajuste da nossa percepção começamos a ver as
coisas com mais clareza. O Buddha disse que ele ensinou a Norma. Quando
continuamente movemos a nossa percepção em direcção à consciência
estamos a deslocar"-nos para a Norma. Na maior parte do tempo,
encontramo"-nos fora da Norma, pois estamos a ignorar - vivemos na
ignorância. Mas quando nos tornamos plenamente atentos, quando somos
essa plenitude de mente, então movemo"-nos em direcção à Norma e
permitimos que as coisas se reflictam dentro de nós. Somos um espelho.
Quando a consciência está clara transforma"-se num espelho sem
fronteiras.

Quando entramos num quarto onde se sente paz, se estivermos conscientes
dessa paz, a nossa mente também fica em paz. Temos tendência para nos
identificarmos com a dispersão mental, mas se nos tornarmos conscientes
da paz, seremos essa paz. Relativamente à paz nesta sala, costumo
sentar"-me aqui e experienciá"-la. Não tem fronteiras. Fecho os meus olhos
e existe paz em meu interior, não existem fronteiras, pois fronteiras
são meras construções. A paz que esta sala tem é infinita. Podemos ter
essa percepção. Esta é uma forma de sairmos do nosso mundo linear.
Quando existe \emph{dukkha} (sofrimento), a maior parte das vezes não
estamos a afastar"-nos dele, mas quando nos direccionamos para a presença
consciente, então estamos a afastar"-nos de \emph{dukkha}. Ao irmos na
direcção da consciência estamos a mover"-nos para a realidade imortal. É
uma mudança de percepção.

As pirâmides do Egipto estiveram em tempos cobertas de pedra calcária.
Nos dias de hoje, quando as vimos em contraste com um céu azul, elas são
simplesmente grandes blocos triangulares -- que apenas são atractivos
para mim e mais meia dúzia de pessoas! Mas originalmente elas estavam
cobertas de pedra calcária \emph{Tutah} e quando o sol lhes batia
reflectiam luz. Quando as pessoas olhavam para elas, em vez de ver
blocos triangulares em contraste com o céu azul, havia o efeito oposto.
Haveria o céu azul e as pirâmides pareciam janelas que davam para um
lugar luminoso, para além do céu. Ocorria uma mudança de percepção. Esta
mudança para uma forma não"-linear é semelhante. Não quer dizer que sejam
iluminados mas serão capazes de compreender mais amplamente, ver mais
assertivamente e abrirem"-se para a forma como as coisas são. Confio
nisto e vos ofereço.


\vfill
{\raggedleft\itshape\small
  Tradução de Appamādo Bhikkhu
\par}

