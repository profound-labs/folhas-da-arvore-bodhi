\chapterAuthor{Ajahn Chah}
\chapter{Samma Samādhi}
\tocChapterNote{Ajahn Chah}

\section{Desapego dentro da prática}

Observemos o exemplo do Buddha. Ele era exemplar quer, na sua própria
prática quer, nos métodos que usou para ensinar os discípulos. O Buddha
ensinou as bases da prática como métodos para que nos livremos do
orgulho. Ele não podia praticar por nós, e tendo ouvido tais
ensinamentos, devemos agora, continuar a praticar e a ensinar a nós
mesmos. Só desta forma teremos resultados, e não somente por o ouvirmos.

O ensinamento do Buddha pode apenas dar"-nos uma compreensão inicial do
\emph{Dhamma}, mas não pode fazer com que o \emph{Dhamma} fique nos
nossos corações. E porque não? Porque ainda não praticámos, ainda não
ensinámos a nós mesmos. O \emph{Dhamma} emerge com a prática.
Conhecem"-no através da prática. Se duvidarem do \emph{Dhamma}, duvidam
da prática. Os ensinamentos dos mestres podem ser verdade, mas somente
ouvir o \emph{Dhamma} não é, por si só, suficiente para sermos capazes
de o realizar. O ensinamento apenas indica qual o caminho. Para realizar
o \emph{Dhamma} temos de agarrar no ensinamento e trazê"-lo para os
nossos corações. A parte que é para o corpo, aplicamos ao corpo, a parte
que é para a fala aplicamos à fala e a parte que é para a mente,
aplicamos à mente. Isto significa que depois de ouvirmos o ensinamento
devemos ensinar a nós mesmos a reconhecer o \emph{Dhamma} como
tal.

O Buddha disse que aqueles que simplesmente acreditam nos outros não são
verdadeiramente sábios. Uma pessoa sábia pratica até se tornar um com o
\emph{Dhamma}, até ter confiança em si próprio, independentemente dos
outros.

Numa dada ocasião, quando o \emph{Venerável Sariputta} se encontrava sentado aos
pés do Buddha, escutando-o respeitosamente enquanto este proclamava o
\emph{Dhamma}, o Buddha virou"-se para ele e perguntou"-lhe ``\emph{Sariputta},
acreditas neste ensinamento?'' ao que o \emph{Venerável Sariputta} respondeu
``Não (eu) não acredito neste ensinamento''.

Ora, esta é uma boa ilustração: o \emph{Venerável} \emph{Sariputta}
ouviu e registou e quando disse que ainda não acreditava não estava a
ser insolente, disse a verdade. Ele apenas considerou aquele
ensinamento, mas porque ainda não tinha desenvolvido a sua própria
compreensão, disse ao Buddha que ainda não acreditava no ensinamento,
pois essa era a verdade. Estas palavras quase ressoam como se o
Venerável \emph{Sariputta} estivesse a ser rude, mas na verdade não
estava. Ele disse a verdade e o Buddha elogiou-o por isso: ``Muito bem,
muito bem, \emph{Sariputta}. Uma pessoa sábia não acredita de imediato,
ela primeiro investiga para depois acreditar''.

A convicção numa crença pode tomar várias formas. Uma dessas formas está
de acordo com o \emph{Dhamma}, enquanto as restantes não, tais como; a
crença cega dos tolos, o entendimento incorrecto, \emph{micchādițțhi} e
compreensão incorrecta. Aqui não damos ouvidos a mais ninguém.

Tomemos o exemplo do \emph{Brahmin} \emph{Dighanakkha}. Ele acreditava
apenas nele próprio e em mais ninguém. A dada altura, quando o Buddha
estava a descansar em \emph{Rajagaha}, \emph{Dighanakkha} foi ouvir os
seus ensinamentos, ou podemos dizer que \emph{Dighanakkha} foi ensinar o
Buddha pois ele tencionava expor os seus próprios pontos de
vista\ldots{} ``Eu sou da opinião de que nada me apraz, nada me
preenche''. Isto era a sua opinião. O Buddha ouviu e depois respondeu:
``\emph{Brahmin}, esta sua opinião também não o preenche''.

Quando o Buddha respondeu desta forma \emph{Dighanakkha} ficou
estupefacto\textbf{,} não sabia o que dizer. O Buddha explicou de muitas
maneira até o \emph{Brahmin} compreender. Ele parou para reflectir e
percebeu\ldots{} ``Uhmm, esta minha percepção não está certa''.

Ao ouvir a resposta do Buddha o \emph{Brahmin} abandonou as suas
opiniões presunçosas e de imediato realizou a verdade. Ele
transformou"-se naquele exacto momento, dando meia volta, tal como alguém
com a palma da mão virada para baixo, a vira para cima. Louvou o
ensinamento do Buddha da seguinte forma: ``Ao ouvir os ensinamentos do
\emph{Abençoado}, a minha mente foi iluminada; como quando alguém que
vive na escuridão se apercebe da luz. É como se a minha mente fosse um
cálice invertido que foi virado para cima, tal como quando um homem que,
tendo estado perdido, encontra o seu caminho''. Nessa altura um certo
conhecimento surgiu da sua mente, de dentro dessa mente, que foi
correctamente colocada na sua posição vertical. A percepção incorrecta
desvaneceu"-se e a correcta tomou o seu lugar. A escuridão desapareceu e
surgiu a luz.

O Buddha declarou que o \emph{Brahmin} \emph{Dighanakkha} foi alguém que
abriu o ``Olho do \emph{Dhamma}''. No princípio \emph{Dighanakkha}
estava agarrado às suas próprias opiniões e não tinha intenção de as
mudar. Mas quando ouviu o ensinamento do Buddha a sua mente viu a
verdade, viu que agarrar"-se àquelas percepções era incorrecto. Quando
surgiu o correcto entendimento, foi capaz de discernir o seu
entendimento anterior como erróneo, daí comparar a sua experiência com
uma pessoa que vivia no escuro e que encontrou a luz. Naquele momento o
\emph{Brahmin} \emph{Dighanakkha} transcendeu a sua compreensão
incorrecta.

Devemos transformar"-nos desta maneira. Antes de abandonarmos os nossos
defeitos, mudamos a nossa perspectiva e começamos a praticar
correctamente e bem. Mesmo que em determinada altura não tenhamos
praticado nem correctamente nem bem, ainda assim pensávamos que
estávamos certos e que éramos bons. Quando realmente vamos ao fundo
desta questão, ``endireitamo"-nos'', tal qual como virar uma mão para
cima. Isto significa que ``Aquele que sabe'', ou a sabedoria, desponta
na mente, capacitando então, a ver as coisas de uma nova forma. Surge um
novo tipo de consciência.

Assim, os praticantes devem treinar para desenvolver este conhecimento
(ao qual chamamos \emph{Buddho}, ``Aquele Que Sabe'') nas suas mentes.
Originalmente ``Aquele Que Sabe'' não está lá, o nosso conhecimento não
é claro, verdadeiro ou completo. Esse conhecimento é, portanto,
demasiado fraco para que possamos treinar a mente. Então a mente muda,
ou inverte"-se, como resultado desta consciência, chamada de sabedoria ou
realização, a qual excede o estado consciente antecedente. O anterior,
``aquele que sabe'' ainda não sabia realmente, sendo incapaz de nos
levar ao nosso objectivo.

Assim, o Buddha ensinou"-nos a observar interiormente, \emph{opanayiko}.
Olhar para o interior, não para o exterior. Ou\textbf{,} se olharmos
para o exterior devemos também olhar para o interior, para aí
observarmos a causa e o efeito. Busquem a verdade em tudo, porque os
objectos exteriores e interiores influenciam"-se sempre, mutuamente. A
nossa tarefa é desenvolver um determinado tipo de consciência até que
esta se torne mais forte que o estado de consciência anterior. Isto faz
com que sabedoria e realização aconteçam dentro da mente, tornando"-nos
aptos a reconhecer claramente o seu funcionamento, a sua linguagem e,
detectar as formas e os meios de todos os nossos defeitos.

Quando pela primeira vez deixou o seu lar para procurar a libertação, o
Buddha provavelmente não estava muito seguro do que fazer, tal como nós.
Ele tentou, através de muitas práticas desenvolver a sua sabedoria. Ele
procurou professores, tais como \emph{Udaka} \emph{Ramaputta}, para
praticar meditação, \ldots{} sentado com a perna direita em cima da
esquerda, a mão direita na esquerda\ldots{} o corpo alinhado \ldots{}
olhos fechados\ldots{} abrindo mão de tudo, até ser capaz de alcançar um
elevado nível de absorção, ou \emph{samadhi}\footnote{%
  O nível de \emph{nada}, uma das ``absorções sem forma'', por vezes
  chamada de sétimo \emph{jhāna} ou absorção.}.
Mas quando ele saiu
desse \emph{samadhi} voltou à sua antiga forma de pensar e percebeu que
iria apegar"-se de novo a ela, tal como anteriormente. Ao ver isto ele
percebeu que ainda não tinha realizado a sabedoria. A sua compreensão
ainda não tinha atingido a verdade, ainda estava incompleta, ainda
faltava algo. De alguma maneira, ao ver isto ele adquiriu alguma
compreensão -- a de que isto ainda não era o cume da prática -- mas
deixou aquele lugar para ir procurar um novo professor. Quando o Buddha
deixou o seu antigo professor, não o condenou; fez tal como a abelha que
tira néctar de uma flor sem danificar as pétalas.

O Buddha então prosseguiu e foi estudar com \emph{Alara} \emph{Kalama} e
atingiu um estado de \emph{samadhi} ainda mais elevado, mas quando saiu
desse estado, \emph{Bimba} e \emph{Rahula}\footnote{%
  \emph{Bimba}, ou \emph{Princesa Yasodhara}, a ex"-mulher do
  Buddha; \emph{Rahula}, o seu filho.
}
vieram de novo aos seus
pensamentos, assim como antigas memórias e sentimentos. Ainda tinha
luxúria e desejo. Reflectindo interiormente ele viu que não tinha
conseguido ainda o seu objectivo e, portanto, ele também deixou este
professor. Ouviu os seus professores e fez o seu melhor para seguir os
seus ensinamentos. Ele investigou continuamente os resultados da sua
prática; não foi uma questão de fazer coisas, para de seguida as
descartar e fazer outras.

Mesmo no que diz respeito a práticas ascéticas, após as ter
experimentado, ele percebeu que passar fome até sermos só pele e osso é
algo que diz somente respeito ao corpo. O corpo não sabe nada. Praticar
daquela forma era quase como executar uma pessoa inocente, ignorando o
verdadeiro ladrão.

Quando o Buddha viu isto percebeu que realmente a prática não diz
respeito ao corpo mas sim à mente. \emph{Attakilamathanuyogo}
auto"-mortificação -- o Buddha tinha-a experimentado e percebeu que esta
limitava"-se ao corpo. Na verdade, a iluminação de todos os
\emph{Buddhas} acontece na mente.

Quando consideramos o corpo ou a mente, podemos colocar ambos no mesmo
``saco'', no ``saco'' do Transitório, Imperfeito e Não"-eu --
\emph{aniccam}, \emph{dukkham}, \emph{anatta}. Eles são simples
condições da Natureza, surgem, dependendo de factores de suporte,
existem por um período de tempo e depois acabam. Quando existem
condições apropriadas, elas voltam, existem por um período de tempo e
depois acabam outra vez. Estas coisas não são um ``eu'', um ``ser'', um
``nós'' ou um ``eles''. Não existe ninguém nessas coisas, apenas
sensações. A felicidade não tem um ``eu'' intrínseco, o sofrimento não
tem um ``eu'' intrínseco. Nenhum ``eu'' pode ser encontrado, pois são
simples elementos da natureza que começam, existem e acabam, passando
por este constante ciclo de mudança.

Todos os seres humanos tendem a identificar"-se com o que surge, com o
que existe e com o que cessa. Agarram"-se a tudo. Não querem que as
coisas sejam da forma como estão e também não querem que seja deferente.
Por exemplo, tendo começado, (eles) não querem que as coisas terminem;
tendo experienciado felicidade, (eles) não querem sofrimento. Se surge o
sofrimento\textbf{,} querem que este acabe o mais rapidamente possível e
seria ainda melhor que não voltasse de todo. Isto é assim porque eles
vêm o corpo e a mente como sendo eles próprios, ou como pertencendo"-lhes
e portanto exigem que estas coisas sigam as suas vontades.

Este tipo de raciocínio é como construir uma barragem ou um dique sem
construir uma passagem para deixar fluir a água. O resultado é que a
barragem rebenta. E o mesmo acontece com este tipo de raciocínio. O
Buddha viu que pensar desta forma é a causa para o sofrimento. Vendo a
causa, o Buddha abandonou-a.

Esta é a Nobre Verdade da Origem do Sofrimento. As Verdades do
Sofrimento, da sua Origem, da sua Cessação e do Caminho que leva a essa
Cessação\ldots{}as pessoas ficam presas justamente aí. Se as pessoas
tiverem que ultrapassar as suas dúvidas é precisamente aqui que o farão.
Ao ver que as coisas são simplesmente \emph{rūpa} e nāma, ou
corporalidade e mentalidade, torna"-se óbvio que elas não são um ser, uma
pessoa, um ``nós'' ou um ``eles''. Elas seguem simplesmente as leis da
Natureza.

A nossa prática é conhecer as coisas desta forma. Não temos realmente o
poder de as controlar, não somos os seus donos. Tentar controlar traz
sofrimento, pois na verdade elas não são nossas para que as possamos
controlar. O corpo e a mente não somos nós, nem os outros. Se soubermos
esta realidade podemos ver claramente. Vemos a verdade, somos um com
ela. É como ver um pedaço de ferro quente, vermelho, aquecido na
fornalha. Todo ele está quente. Quer, lhe toquemos em cima, quer, em
baixo ou dos lados, está sempre quente. Não importa onde lhe tocamos,
está quente. É desta forma que deveríamos compreender as coisas.

Quando começamos a praticar\textbf{,} basicamente queremos obter ou
atingir algo - saber e ver - mas ainda não sabemos o que é que vamos
atingir ou saber. Havia um discípulo meu cuja prática estava repleta de
dúvidas e confusão. Mas ele continuou a praticar e eu continuei a
instruí"-lo até ele começar a encontrar alguma paz. Mas quando ele,
eventualmente, ficava um pouco mais calmo, levantava"-se e perguntava ``O
que faço a seguir?'' Pronto! Surgia novamente a confusão. Ele dizia que
queria paz, mas quando a alcançava não a queria e perguntava o que fazer
a seguir.

Nesta prática devemos fazer tudo com desapego. Como nos desapegamos?
Desapegamo"-nos quando vemos as coisas claramente. Conhecer as
características do corpo e da mente tal como são. Meditamos para
encontrarmos paz, mas ao fazê"-lo vimos aquilo que não é pacífico. Isto
acontece porque a natureza da mente é movimento.

Quando praticamos \emph{samadhi} colocamos a nossa atenção nas
inspirações e nas expirações na ponta do nariz ou no lábio superior.
Este elevar da mente para a concentração é chamado \emph{vitakka} ou
``elevar''. Quando temos então a mente elevada e fixa num objecto, a
isto chama"-se \emph{vicāra}, a contemplação da respiração na ponta do
nariz. Esta qualidade de \emph{vicāra} irá naturalmente misturar"-se com
outras sensações mentais e podemos pensar que a nossa mente não está
calma e que não vai acalmar, mas na verdade são os resultados de
\emph{vicāra} enquanto se entrelaçam com essas sensações. É claro que se
isto for demasiado longe na direcção errada, a mente irá perder a sua
concentração e aí deveremos redireccioná"-la, refrescá"-la, elevá"-la para
o objecto de concentração através do \emph{vitakka}. Assim que tivermos
estabelecido a nossa atenção o \emph{vicāra} predomina, misturando"-se
com as várias sensações mentais.

Quando vemos isto acontecer, a nossa falta de compreensão pode levar"-nos
a questionar: ``porque é que a minha mente divagou? Eu quero que ela
fique calma, porque é que ela não fica calma?'' Isto é praticar com
apego.

Na verdade, a mente está apenas a seguir a sua natureza, mas nós
acrescentamos algo a essa actividade por querermos que a mente fique
calma. Para além de tudo isto, surge a aversão, aumentando as nossas
dúvidas, o nosso sofrimento ou a nossa confusão. Portanto, se houver
\emph{vicāra}, reflectindo nos vários acontecimentos que se dão dentro
da mente, deveríamos sabiamente reflectir\ldots{} ``Ah, a mente é
naturalmente assim''. Ora aí está, esse é ``Aquele Que Sabe'' a falar,
dizendo"-nos para vermos as coisas tal como são. A mente é simplesmente
assim. Entregamo"-nos a isso e então a mente fica pacífica. Quando já não
estiver centrada trazemos de novo \emph{vitakka} e em breve haverá calma
outra vez. \emph{Vitakka} e \emph{vicāra} trabalham desta forma juntos.
Usamos \emph{vicāra} para contemplar as várias sensações que surgem.
Quando \emph{vicāra} se torna gradualmente mais disperso trazemos de
novo a nossa atenção através do \emph{vitakka}.

Aqui o ponto importante é que a esta altura a nossa prática deverá ser
levada a cabo com desapego. Ao ver o processo do \emph{vicāra} a
interagir com as sensações mentais podemos julgar que a mente está
confusa e tornarmo"-nos avessos a este processo. Aqui está a causa. Não
estamos contentes, pois queremos que a mente esteja calma. Esta é a
causa -- perspectiva errada. Se corrigirmos apenas um bocadinho a nossa
perspectiva, observando esta actividade como apenas a natureza da mente,
torna"-se por si só suficiente para descartar a confusão. A isto chama"-se
desapego.

Se não nos apegarmos, se praticarmos com desapego\ldots{} desapego
dentro da prática e prática dentro do desapego, se aprendermos a
praticar desta forma, então \emph{vicāra} naturalmente terá menos com
que se ocupar. Se a nossa mente deixar de estar perturbada, então
\emph{vicāra} tenderá a contemplar o \emph{Dhamma}, pois se não o fizer
a mente ficará de novo distraída.

Portanto existe \emph{vitakka} seguida de \emph{vicāra}, \emph{vitakka}
seguida de \emph{vicāra}, \emph{vitakka} seguida de \emph{vicāra},
continuamente até \emph{vicāra} tornar"-se gradualmente mais subtil. No
princípio \emph{vicāra} abrange toda a nossa percepção mental. Quando
percebermos que é apenas a actividade natural da mente, não nos
incomodará a menos que nos apeguemos. É como a corrente de água num rio.
Se nos tornarmos obcecados, perguntando ``porque corre?'' então iremos
naturalmente sofrer. Se compreendermos que a água corre porque assim é a
natureza, então não haverá sofrimento. \emph{Vicāra} é assim. Existe
\emph{vitakka} seguida de \emph{vicāra}, interagindo com as sensações
mentais. Podemos usar estas sensações como objecto de meditação,
acalmando a mente através da observação dessas sensações.

Se desta forma conhecermos a natureza da mente então desapegamo"-nos, da
mesma forma que deixamos a água correr. \emph{Vicāra} torna"-se cada vez
mais subtil. Talvez a mente passe a ter a tendência para contemplar o
corpo, ou a morte, por exemplo, ou algum outro tema do \emph{Dhamma}.
Quando o tema de contemplação está presente, surge um sentimento de
bem"-estar. O que é esse bem"-estar? É \emph{pīti} (êxtase). \emph{Pīti},
surge bem"-estar. Pode manifestar"-se como arrepios (``pele de galinha''),
uma sensação de frescura ou de leveza. A mente está extasiada. A isto
chama"-se \emph{pīti}. Existe também prazer (\emph{sukha}), o ir e vir de
várias sensações, e o estado de \emph{ekaggatārammana} (a mente focada
numa só direcção ou num só ponto; o oposto de dispersa).

Se falarmos em termos do primeiro estágio de concentração, temos:
\emph{vitakka}, \emph{vicāra}, \emph{pīti}, \emph{sukha},
\emph{ekaggatā}. Então como é o segundo estágio? Enquanto a mente se
torna progressivamente mais subtil, \emph{vitakka} e \emph{vicāra}
tornam"-se, comparativamente, mais densos, sendo assim descartados
deixando apenas \emph{pīti}, \emph{sukha} e \emph{ekaggatā}. Isto é algo
que a mente faz por si própria, não temos de conjecturar acerca disto,
apenas conhecer a natureza das coisas tal como são.

Quando a mente se torna mais refinada, \emph{pīti} será eventualmente
descartada, deixando apenas \emph{sukha} e \emph{ekaggatā}, e iremos
aperceber"-nos disso. Para onde é que \emph{pīti} foi? Não foi a lado
algum, o que acontece é que a mente vai"-se tornando, gradualmente, mais
subtil e, por isso, descarta aquelas qualidades que são demasiado densas
para ela. Tudo o que é demasiado denso é descartado até alcançar o ponto
de subtileza conhecido nos livros como \emph{Quarto} \emph{Jhāna}, o
mais elevado nível de absorção. Nesse ponto a mente retirou,
progressivamente, tudo o que se tornou demasiado denso, até permanecerem
somente \emph{ekaggatā} e \emph{upekkhā}, equanimidade. Não há nada mais
além disto, este é o limite.

Quando a mente está a desenvolver os estágios de \emph{samādhi} deve
proceder desta forma, mas compreendam, por favor, o básico da prática.
Queremos acalmar a mente mas ela não ficará calma. Isto é praticar por
desejo, contudo, não nos apercebemos disso. Desejamos calma. A mente
está sempre perturbada e nós ainda perturbamos mais por querermos que
fique calma. Este mesmo querer é a causa. Não vemos que o desejo de que
a mente acalme é \emph{tanha} (anseio). Apenas aumentamos o fardo.
Quanto mais desejamos calma, mais perturbada fica a mente, até que
desistimos. Acabamos por estar sempre a lutar, sentados às ``turras''
connosco próprios. Porquê? Porque não reflectimos na maneira como
predispomos a mente. Conheçam as condições da mente por aquilo que elas
são. Seja o que for que apareça, apenas observem. Trata"-se puramente da
natureza da mente, não é danosa a menos que não a compreendamos. Não é
perigosa se virmos a sua actividade pelo aquilo que é. Então praticamos
com \emph{vitakka} e \emph{vicāra} até a mente começar a acalmar e a
tornar"-se mais dócil. Quando surgem sensações, observamo"-las e
começaremos então, a conhecê"-las.

Contudo, normalmente temos a tendência para lutar com elas porque desde
o princípio estamos determinados a acalmar a mente. Assim que nos
sentamos os pensamentos vêm angustiar"-nos. Assim que designamos o nosso
objecto de meditação a nossa atenção vagueia por todos os pensamentos,
achando que esses pensamentos vieram para nos perturbar. Mas, na
verdade, o problema começa imediatamente ali, desse mesmo querer.

Se virem que a mente está somente a comportar"-se de acordo com a sua
natureza, onde os pensamentos vão e vêm naturalmente, não se foquem
nisso demasiado. Podemos compreender os caminhos da mente, assim como
compreendemos as crianças. As crianças dizem todo o tipo de coisas, não
sabem fazer melhor. Se as compreendermos, então deixamo"-las falar, pois
elas o fazem naturalmente. Quando abrimos mão, não existe nenhuma
obsessão com a criança. Podemos falar com os nossos convidados
imperturbavelmente, enquanto a criança tagarela e brinca à nossa volta.
A mente é assim. Não é danosa a menos que nos agarremos a ela e nos
tornemos obcecados. Isso é a verdadeira causa dos enredos.

Quando surge \emph{pīti}, sentimos um prazer indescritível que apenas
quem experiencia consegue apreciar. \emph{Sukha} (prazer) surge, e
também a qualidade de focar num só ponto. Ocorrem \emph{vitakka},
\emph{vicāra}, \emph{pīti}, \emph{sukha} e \emph{ekaggatā}. Todas estas
cinco qualidades convergem para um ponto. Apesar se serem qualidades
diferentes, elas estão todas reunidas num mesmo sítio, e podemos
observá"-las, como quando vemos diferentes tipos de fruta na mesma
tigela. \emph{Vitakka}, \emph{vicāra}, \emph{pīti}, \emph{sukha} e
\emph{ekaggatā}, são as cinco qualidades que podemos ver na mente. Se
alguém perguntasse ``Como é que \emph{vitakka} está lá, como é que
\emph{vicāra} está lá, como é que estão lá \emph{pīti} e
\emph{sukha}?'', seria difícil responder. Mas quando elas convergem na
mente, podemos ver como é, por nós próprios.

A esta altura a nossa prática torna"-se de certa forma especial. Devemos
ter interiorização e auto consciência e não nos dispersarmos. Saibam
definir as coisas. Estes são estágios de meditação, o potencial da
mente. Não tenham dúvidas de nada no que respeita à prática. Mesmo que
se afundem na terra ou voem pelo ar, ou mesmo que ``morram'' enquanto
estão sentados, não duvidem disso. Quaisquer que sejam as qualidades da
mente, fiquem pelo conhecimento destas. Não dispersem

Esta é a nossa base: ter \emph{sati}, interiorização e
\emph{sampajanna}, auto"-consciência, quer, estejamos em pé, quer, a
andar, sentados ou deitados. O que quer que surja, deixem"-no ``ser'',
não se agarrem a isso. Seja agradável ou desagradável, felicidade ou
sofrimento, dúvida ou certeza, contemplem com \emph{vicāra} e verifiquem
os resultados dessas qualidades. Não tentem rotular tudo, apenas
reconheçam. Percebam que todas as coisas que surgem na mente são simples
sensações. São transitórias; começam, existem e acabam. Isso é tudo o
que elas são, não têm um ``eu'' ou um ser, não são ``nós'' ou ``eles''.
Não vale a pena agarrarmo"-nos a nenhuma delas.

Quando vemos toda a \emph{rūpa} e \emph{nāma}\footnote{%
  \emph{Rūpa}: objectos materiais ou físicos; \emph{nāma}: objectos
  imateriais ou mentais. Os constituintes físicos e mentais do ser.
}
desta forma, com
sabedoria, então vemos os arquétipos. Apercebemo"-nos da transitoriedade
da mente e do corpo, da transitoriedade da felicidade, do sofrimento, do
amor e do ódio: tudo é impermanente. Ao observar isto a mente fica
cansada; cansada do corpo e dela própria, cansada de coisas que surgem e
cessam e que são transitórias. Quando a mente se ``desencanta'' procura
uma forma de largar todas estas coisas. Já não quer ficar presa a
coisas, apercebe"-se da falta de nexo deste mundo e da falta de nexo do
nascimento.

Quando a mente vê assim, onde quer que estejamos vemos \emph{aniccam}
(Transitoriedade), \emph{dukkham} (Imperfeição) e \emph{anatta}
(Não"-eu). Não resta nada ao qual nos agarrarmos. Quer, nos sentemos
debaixo de uma árvore, quer, no topo de uma montanha ou num vale,
conseguimos ouvir o ensinamento do Buddha. Todas as árvores parecerão
uma, todos os seres serão Um, não haverá nada de especial acerca de
nenhum deles em particular. Todos eles começam, existem durante um
período, envelhecem e morrem.

Vemos então o corpo e a mente com mais clareza e o mundo de forma mais
nítida. Eles tornam"-se mais claros à luz da Transitoriedade, à luz da
Imperfeição e à luz do ``Não"-eu''. Se as pessoas se agarrarem
seguramente às coisas, sofrem. É assim que começa o sofrimento. Se
virmos que o corpo e a mente são simplesmente como são, não surge
sofrimento algum, porque não nos agarramos de imediato a eles. Em
qualquer lugar temos sabedoria. Até mesmo ao observar uma árvore podemos
observá"-la à luz da sabedoria. Olhar para a relva e para os vários
insectos torna"-se alimento para reflexão. Quando chega a este ponto vai
tudo para o mesmo ``saco''. É tudo \emph{Dhamma.} Tudo é invariavelmente
transitório. Esta é a verdade, este é o verdadeiro \emph{Dhamma}, isto é
uma certeza. Como pode ser uma certeza? É uma certeza no sentido de que
o mundo é assim e nunca poderá ser de outra forma. Não existe nada mais
nele do que isto. Se conseguirmos ver isto, então terminamos a nossa
jornada.

No Budismo, no que respeita a pontos de vista, diz"-se que achar que
somos mais tolos que os outros, está errado; achar que somos iguais aos
outros, está errado; e achar que somos melhores que os outros, está
errado\ldots{} porque não existe nenhum ``nós''. Precisamos de
desenraizar os conceitos. A isto chama"-se \emph{lokavidū} -- conhecer
claramente o mundo como ele é. Se virmos então a verdade, a mente vai
indubitavelmente reconhecê"-la e ceifará a causa do sofrimento. Quando já
não existe causa, não poderá haver consequência. É desta forma que a
nossa prática deverá prosseguir.

As bases que precisamos desenvolver são: primeiro, ser honesto e
sincero; segundo, estar consciente de que erramos; terceiro, ter o
atributo de humildade dentro do nosso coração, ser frugal e contentar"-se
com pouco. Se vivermos contentes falando pouco e atentos em todas as
outras coisas, ver"-nos-emos a nós próprios, não seremos levados por
distracções. A mente ficará assente em \emph{sila}, \emph{samādhi} e
\emph{paññā}.

Portanto, aqueles que cultivam o caminho não devem ser descuidados.
Mesmo que estejam correctos, não sejam descuidados. E se estiverem
errados, não sejam descuidados. Se as coisas estiverem a correr bem e se
sentirem felizes, não sejam descuidados. Porque digo ``não sejam
descuidados?'' Porque todas as coisas são incertas. Tenham"-nas como tal.
Se estiverem a sentir paz, deixem a paz acontecer. Podereis realmente
desejar continuar a desfrutar da paz, mas deveis saber a sua natureza; e
o mesmo se aplica às qualidades desagradáveis.

O cultivo da prática mental pertence a cada indivíduo. O professor só
explica como treinar a mente, pois esta é única em cada indivíduo.
Sabemos o que está lá, ninguém pode conhecer a nossa mente tão bem
quanto nós. A prática requer este tipo de honestidade. Façam"-no
diligentemente, não a meio gás. Quando digo ``diligentemente'' quer isso
dizer que se devem exaurir? Não, não têm que se exaurir porque a prática
é feita na mente. Se souberem isto, então saberão o que é a prática. Não
precisam de muito. Usem apenas as bases da prática para reflectirem
interiormente sobre vós próprios.

Neste momento estamos a meio do Retiro das Chuvas. Para a maior parte
das pessoas é normal deixar a prática ``descarrilar'' depois de algum
tempo. Não são consistentes do princípio ao fim. Isto demonstra que a
sua prática ainda não está madura. Tendo determinado uma dada prática no
início do retiro, qualquer que esta seja, devemos levar essa resolução
até ao fim. Durante estes três meses pratiquem consistentemente. Devem
tentar. Seja o que for que determinaram como prática, levem isso em
consideração e observem se a prática está a enfraquecer. Se for o caso
façam um esforço para restabelecê"-la. Continuem a refrescar a prática,
tal como quando praticamos concentração na respiração. Enquanto a
respiração se dá, fora e dentro, a mente distrai"-se. Então restabeleçam
a atenção na respiração. Quando a vossa atenção vagueia de novo
tragam"-na de volta mais uma vez. Isto é o mesmo, no que diz respeito,
quer ao corpo, quer à mente a prática continua desta forma. Por favor
façam um esforço nesse sentido.

\bigskip

{\raggedleft\itshape
  Tradução de Appamādo Bhikkhu
\par}
