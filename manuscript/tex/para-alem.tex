\chapter{Para Além do Sentido de Si}

Ajahn Sumedho

Penso que há algo comum que interessa a todos -- uma vez que somos nós
próprios o tema das nossas vidas. Temos naturalmente este interesse,
pois temos que viver connosco a vida inteira, independentemente da
opinião que possamos ter sobre nós mesmos. A nossa percepção pessoal
pode ser assim algo que nos traga muitas dificuldades se nos virmos da
maneira errada. Mesmo sob circunstâncias confortáveis, se não nos virmos
da maneira correcta, acabaremos por criar sofrimento nas nossas mentes.
Buddha salientou que a maneira de resolver o problema não passa por
tentar fazer com que tudo corra bem e seja agradável na dimensão
exterior, mas antes o desenvolvimento da compreensão correcta e da
atitude correcta perante nós mesmos. É aqui que está toda força do seu
ensinamento.

Ao vivermos em Inglaterra, nesta época, esperamos conforto e todo o tipo
de direitos e privilégios. Isto torna a vida mais agradável de várias
maneiras, mas quando todas as nossas necessidades estão garantidas e a
vida é confortável, algo em nós não se desenvolve. Por vezes é a luta
contra as adversidades que nos faz desenvolver e amadurecer como seres
humanos. Recordo-me de que quando vivia em Londres, costumava caminhar
no ``Hampstead Heath'' pela manhã e observar as pessoas mais abastadas a
levar os seus cãezinhos a passear no ``Heath''. Pensava então, que não
seria assim tão mau nascer como um cãozinho de estimação aqui em
Inglaterra: ter uma senhora simpática sempre a mimar-nos, a fazer
pequenas roupinhas para o inverno e a dar-nos saborosos biscoitos de cão
para comermos. Parecia que uma vida de afecto e conforto seria muito
agradável! Mas a verdade é que a maioria de nós iria sufocar;
necessitamos de nos comparar com outros, de lutar e aprender, como ir
para além das limitações que pensamos ter a cada momento. Iremos
encontrar a derrota quando nos entregamos às nossas limitações através
da resignação. Então é claro que ficaremos deprimidos e miseráveis.

Porém, quando renunciamos aos condicionalismos mentais e nos contemos
com sabedoria, então encontraremos a libertação! A vida é a experiência
do condicionalismo e da contenção, é nascer num corpo humano e ter que
viver de acordo com as leis naturais do planeta terra. Mentalmente
podemos ascender aos céus mas fisicamente estamos condicionados pelas
limitações que se tornam cada vez mais restritivas à medida que
envelhecemos. Tal não precisa de ser visto como sofrimento, pois é assim
que as coisas são. Podemos desenvolver uma atitude diferente e aprender
a aceitar tais limitações -- não pela resignação negativa mas porque
percebemos que aquilo que procuramos está dentro de cada um de nós. Não
precisamos de procurar fora, não precisamos pensar que é algo
inacessível ou afastado de nós. Depende da nossa vontade de parar de
resistir, de acalmar, de ouvir e despertar para a nossa experiência
consciente. Claro que o grande obstáculo a isso é que temos uma noção de
nós mesmos como sendo isto, aquilo ou aquela coisa.

A noção que temos sobre nós próprios, é algo que se torna consciente
quando somos crianças; quando nascemos não existe a noção de um
\emph{'eu'} que é coisa alguma. À medida que crescemos, então aprendemos
aquilo que devemos ser, se somos bons ou maus, se somos adoráveis ou
não, se temos ou não aprovação. Desenvolvemos assim uma ideia de
``\emph{eu''}. Frequentemente comparamo-nos com os outros e temos
pessoas exemplos, de como devemos ser quando crescermos. Na minha
própria experiência reparei que o ego começou realmente a consolidar-se
quando fui enviado para a escola: fui atirado para salas de aula com
aquelas crianças estranhas e comecei a reparar quem seria o mais forte,
o mas duro, quem seria aquele que o professor mais gostava. Vemo-nos em
termos da nossa relação com os outros. Desenvolvemos esta conexão pela
vida fora a, menos que, deliberadamente escolhamos mudar e comecemos a
procurar uma forma mais profunda de viver as nossas vidas, do que sob os
condicionamentos da mente que foram adquiridos quando éramos mais
jovens. Mesmo quando envelhecemos ainda podemos ter atitudes de
adolescentes ou reacções infantis em relação à vida, as quais não fomos
capazes de resolver - apenas as suprimimos ou ignorámos. Tudo isto pode
ser muito embaraçoso ou chocante.

Existe uma maneira de falarmos do ``\emph{eu''} que soa muito doutrinal.
Os budistas podem dizer, por vezes, que não existe \emph{eu} como se
fosse uma proclamação em que temos que acreditar: é como se existisse um
Deus nas alturas dizendo ``Não existe um \emph{eu}!'' e há algo em nós
que resiste a essa declaração. Não parece ser verdadeiro o anúncio da
inexistência de um \emph{eu} - o que é essa experiência que se sente
agora mesmo? Parece-nos, pelo contrário, que ``aqui'' existe um bem
demarcado sentido de nós próprios! Estamos a sentir, a respirar, a ver e
a ouvir; reagimos às coisas, as pessoas podem criticar-nos e vamos
sentir-nos felizes ou tristes consoante o caso. Se isto não sou eu, o
que é então? Será que devemos contornar a situação como um crente
budista e acreditar que não temos um \emph{eu}? Ou, já que vamos
acreditar em alguma coisa, talvez seja melhor acreditar que temos um
\emph{eu}, porque assim, posso dizer coisas como: ``o meu verdadeiro eu
é perfeito e puro.'' Isto pelo menos dá-nos algum tipo de encorajamento,
inspirador para vivermos as nossas vidas em vez de dizerem que não
existe um \emph{eu}, uma alma, fazendo a total aniquilação de quaisquer
possibilidades. Estes são usos de linguagem: podemos dizer ``não existe
um \emph{eu}'' como uma proclamação, ou dizer ``não existe um
\emph{eu}'' como uma reflexão. O modo reflectivo serve para nos
encorajar a contemplar o \emph{eu}. O Buddha realçou o fato de que
quando observamos estas condições mutáveis com as quais temos tendência
para nos identificarmos, podemos ver que elas não são o \emph{eu}.
Aquilo em que acreditamos, ao qual nos agarramos e que assumimos como
uma certeza, não é aquilo que realmente somos: é apenas uma posição, uma
condição, algo que muda de acordo com o tempo e o lugar. Cada um de nós
experiencia a consciência através do corpo humano que temos, e é assim.

A consciência sensorial é uma função natural, não existindo sentido de
\emph{eu} em relação a ela. A única razão pela qual poderemos assumir um
\emph{eu} é porque a consciência opera em termos de sujeito e objecto;
para se ser consciente temos que ser uma entidade separada e então
operamos a partir desta posição de sermos este ser subjectivo que está
aqui. Podemos assim tornar-nos obcecados com uma interpretação muito
formal de tudo: cada reacção ou experiência, seja ela instintiva ou não,
pode ser interpretada no sentido de ser \emph{eu} e \emph{meu}. Podemos
interpretar as energias naturais do corpo de um modo muito pessoal como
se isto fosse \emph{eu}, o \emph{meu} problema, em vez de vê-las como
parte do conjunto que temos quando nascemos. Mesmo um recém-nascido, tem
instintos de sobrevivência e chora quando tem fome. Os bebés são
geralmente criaturas bonitas e, naturalmente, desejamos dar-lhes amor e
cuidar deles. Será que podemos achar que o bebé esteja a fazer tudo isto
deliberadamente - ``Estou a ser engraçadinho para que o Ajahn Sumedho me
segure ou a minha mãe me ame'' -- ou será apenas a maneira natural das
coisas, a natureza no seu modo de funcionamento? Estas são coisas
naturais mas vemo-las de um modo muito pessoal.

Assim, mantemos pontos de vista muito pessoais e carregamo-los pela vida
fora: ela é assim, ele é desta forma - e isto influencia a maneira como
reagimos e damos resposta a cada um, baseados na maneira que alguém se
apresenta (agradável, feliz, acolhedor; mau e desagradável; alguém que
nos insulta ou nos elogia). Podemos transportar ressentimentos pela vida
fora sobre termos sido insultados e nunca esquecer quem nos insultou.
Talvez o tivesse feito porque estava a passar um mau bocado, e mesmo
após trinta anos podemos, se quisermos, fazer disso um problema. Assim,
este \emph{eu} precisa de ser examinado, observado e contemplado, em
termos religiosos. Todas as religiões têm os seus ensinamentos sobre a
vacuidade do ego: de alguma maneira as religiões contemplam o abandono
das tendências egoístas da mente. Então, antes de dizermos, por exemplo,
que queremos alcançar o Reino de Deus, temos que abandonar as nossas
tendências pelos nossos fascínios e obsessões egoístas. Ou, se queremos
realizar o verdadeiro \emph{Dhamma}, temos que abandonar a maneira de
nos vermos a nós mesmos. Este pode ser outro mandamento do além, como
``Não devemos ser egoístas! Larguemos todo o egoísmo e tentemos ser
alguém mais puro!'' Todos nós concordamos com isto, ninguém aprecia a
ideia de ser cada vez mais egoísta, mas por vezes não sabemos como
deixar de o ser. Podemos ter grandes ideias como abandonar toda a nossa
riqueza e não nos apegarmos a nada, ficando assim mais perto de nos
tornarmos não egoístas -- mas, por vezes, o mais estranho é que quando
nos tornamos monges ou monjas, embora estejamos a pensar que nos
livrámos do egoísmo, verificamos que nos estamos a tornar mais e mais
egoístas. O egoísmo torna-se muito concentrado porque nestas
circunstâncias não podemos expandir-nos por uma vasta variedade de
coisas como na vida de leigos. Então, tornamo-nos muito mais conscientes
disso. E se nos culpabilizarmos então caímos numa situação sem saída,
pois começamos a interpretar a vida do ponto de vista ``Eu sou egoísta e
tenho que me livrar deste egoísmo.'' Um dos grandes problemas na nossa
maneira de pensar, é abandonar a premissa básica de que ``Eu sou esta
pessoa e tenho que fazer algo para deixar de ser egoísta, e tornar-me um
ser iluminado no futuro.''

Na nossa cultura, estamos condicionados a pensar desta maneira: devemos
ser bons rapazes e fazer isto e aquilo para que no futuro nos tornemos
alguém de valor e aceite na sociedade. Isto faz sentido no lado mundano
da vida porque começamos iletrados e portanto temos de aprender a ler e
depois temos de estudar todos os diferentes assuntos na escola para
sermos alguém no sistema. Se falharmos então tornamo-nos alguém que
falha. E falhar é desprezado. É interessante ensinar meditação a pessoas
que têm este medo de falhar pois estas pessoas sentem que vão falhar na
meditação. Mas não é possível falhar na meditação. Não se trata de
falhar, senão seria apenas mais outra forma de tentarmos provar a nós
próprios que somos capazes: ``Eu não posso agora. Se eu praticar com
determinação, serei um bom meditador e tornar-me-ei iluminado,
espero....'' E a dúvida surge: ``Mas acho que jamais serei iluminado.
Quem é iluminado?''

As pessoas gostam de verificar se Ajahn Sumedho é iluminado ou se o
Ajahn Viradhammo também é iluminado, ou pelo menos se já alcançámos um
nível avançado. Ou seremos apenas uns tipos que ainda não chegaram lá?
Mas existe uma maneira diferente de observar e pensar que é oposta à de
nos vermos como alguém que tem que fazer alguma coisa para se tornar
melhor do que se é presentemente. Isto é o que as pessoas gostam de
ouvir, não é? ``Eu tive todos os géneros de problemas e era muito
miserável, infeliz e depois praticando meditação vi a luz e agora
sinto-me feliz e completo.'' Do ponto de vista mundano e condicionado:
``Eu sou esta pessoa, eu tenho esta personalidade, eu sou Ajahn
Sumedho... sou muitas coisas... eu devo ser e eu não devo ser.'' Mas o
objectivo da meditação budista é a mudança das nossas atitudes usando as
funções mentais de reflexão ou de intuição.

Quando entramos na quietude da meditação, frequentemente o sentido de si
próprio domina-nos e somos invadidos por todos os tipos de memórias e
ideias a nosso respeito. Por vezes desejamos que ``se for meditar então
entrarei em quietude e vou libertar-me desde cenário angustiante de mim
mesmo.'' Por vezes a mente simplesmente pára e experienciamos vários
tipos de êxtase, ou uma paz que já havíamos esquecido ou na qual nunca
tínhamos reparado. Mas o sentido de si próprio ainda continuará a operar
pela força do hábito. Então desenvolvemos uma atitude de ouvir o
\emph{eu}, não em termos de acreditar ou desacreditar mas em reparar no
que é que realmente surge e cessa. Não importa se pensamos que somos o
melhor ou o pior, pois essa mesma condição surge e desaparece. Através
do largar ou de vermos a vacuidade do ego, e não na tentativa nos
livrarmos da situação, mas permitindo que ela se dissipe, podemos
experienciar a verdadeira natureza da mente, que é feliz e silenciosa.

Assim, existem momentos nas nossas vidas em que o \emph{eu} pára de
funcionar e em que entramos em contacto com o estado puro de experiência
consciente. É isso a que chamamos de estado de graça. Mas quando temos
essas experiências de felicidade, imediatamente desejamos tê-las
novamente, e independentemente do esforço que façamos para repetir esse
estado, enquanto estivermos apegados à ideia do desejo que ocorra
novamente, esse estado não voltará. Não funciona assim. Desejar tal faz
com que se torne impossível, e assim a atitude deverá ser de largar o
desejo. Não tentar suprimir o desejo, pois essa é outra forma de desejo;
o desejo de afastar o desejo é o mesmo problema: não funciona. Também
não funciona seguirmos o desejo. Mas, no estado de consciência desperta
vemos claramente o que na realidade se passa e, então, podemos
libertar-nos das causas do nosso sofrimento, tendo a sabedoria intuitiva
de o largar. Nesta vida como seres humanos, desde o nosso nascimento até
à morte, cada momento é uma oportunidade para compreendermos as coisas
da forma correcta. Sucesso ou falha deixam de ter um significado
especial pois mesmo quando falhamos, podemos aprender com isso. Não quer
dizer que não tentemos fazer o melhor possível e que não nos esforcemos,
mas o nosso objectivo deixa de ser o sucesso e passa a ser compreender a
vida.

Leva muito tempo a chegarmos à raiz desta noção de nós próprios, pois
está completamente infiltrada na nossa experiência consciente. Com a
meditação, tomamos atenção a coisas comuns como a respiração e o corpo e
então aprendemos a levar e a manter a nossa atenção no momento presente.

~

\begin{quote}
Tradução de Pedro Cruz
\end{quote}
