\chapterAuthor{Ajahn Candasiri}
\chapter{Porquê Ir a um Mosteiro}
\tocChapterNote{Ajahn Candasiri}

Esta é uma pergunta que todos devemos fazer a nós mesmos, quer, sejamos
monges, monjas, noviços quer, visitantes. Porque viemos? Precisamos de
ser claros sobre isto no sentido de tirar o maior benefício daquilo que
um mosteiro tem para oferecer. Se, de facto, isto não for claro, podemos
desperdiçar muito tempo a fazer coisas que podem diminuir os benefícios
que são aqui encontrados.

O Buddha falou de três fogos, três obstáculos que afligem os seres
humanos. Essas três coisas fazem"-nos agir constantemente, nunca nos dão
oportunidade para descansar ou descontrair; elas são, a cobiça, o ódio e
a ilusão, em língua Palī, \emph{lobha, dosa} e \emph{moha}. O Buddha,
também por compaixão, apontou qual era o antídoto. Na verdade estes três
fogos são baseados em instintos naturais. Por exemplo, a avidez ou o
desejo sensual, a energia sexual e o desejo por comida são aquilo que
permite ao ser humano sobreviver. Sem o desejo sexual, nenhum de nós
estaria aqui agora! E, obviamente, também sem fome ou desejo por comida
não seríamos levados a ingerir a nutrição que necessitamos para manter o
corpo em razoável estado de saúde. Contudo, a dificuldade surge, quando
perdemos a sensibilidade sobre o que é necessário e procuramos a
gratificação sensual por si só.

Outro tipo de instinto de sobrevivência é a nossa resposta ao perigo,
quer, reagindo ou atacando o que é percebido como uma ameaça à nossa
sobrevivência, quer, tentando escapar disso. Esta é a base para
\emph{dosa}, ódio ou aversão. Claramente, também isto tem um lugar
importante na natureza. Mas, podemos ficar confusos e o que acabamos por
defender é não tanto o corpo físico, mas o sentido do eu pessoal que
percepcionamos sobre nós próprios, em relação a outro.

O terceiro fogo, que naturalmente precede os dois anteriores, é a
ilusão, \emph{moha}; é não ver com clareza ou não compreender as coisas
na sua realidade, não perceber o que é ser um Ser Humano.
Identificamo"-nos a nós e aos outros como personalidades, ou eus. Mas
isto são apenas ideias ou conceitos, os quais redimensionamos em relação
a outros conceitos de quem ou o quê deveríamos ser. Então, se alguém vem
e desafia esse eu, isso pode criar"-nos uma forte reacção;
instintivamente atacamos, defendemos ou tentamos fugir da presumível
ameaça. Realmente, é algo de loucos, quando pensamos sobre isto.

Tal como referi, tendo o Buddha apontado a natureza da doença, também
apontou a cura. Isto foi transmitido sob a forma de simples
ensinamentos, os quais nos podem ajudar a libertar dessas doenças, como
também evitar fazer coisas que as agravem.

Isto conduz"-me à verdadeira razão que nos traz a um mosteiro. Nós
queremos libertar os nossos corações da doença, dos laços do desejo e da
confusão; e reconhecemos que o que nos é ofertado aqui é a possibilidade
de realizar isto. É claro que pode haver outras razões: algumas pessoas
realmente não sabem porque vieram, elas simplesmente sentem"-se atraídas
pelo lugar.

Então o que é que se passa no mosteiro que é diferente do que se passa
fora? É um local que nos relembra a nossa aspiração e potencial. Lá
estão as belas imagens, do Buddha e dos seus discípulos que parecem
irradiar um sentimento de calma, tranquilidade e estado de alerta.
Também aqui encontramos uma comunidade de monges e monjas que decidiram
viver seguindo o estilo de vida que o Buddha recomendou para curar essas
doenças.

Tendo reconhecido que estamos doentes e que precisamos de ajuda,
começamos a ver que a cura está na direcção oposta aos caminhos do
mundo. Vemos que se nos vamos curar, precisamos primeiro, compreender a
causa da doença, e qual é o desejo. Então, precisamos de compreender os
nossos desejos para ficarmos livres e sermos um Eu separado deles.
Então, em vez de seguirmos os nossos desejos, examinamo"-los de perto.

A disciplina que aqui se pratica é baseada em preceitos, os quais,
usados sabiamente, criam um sentido de dignidade e de auto"-respeito.
Eles refreiam"-nos de acções ou palavras que são prejudiciais a nós e aos
outros, e delineiam um padrão de simplicidade ou renúncia.
Questionamo"-nos: «O que é que eu realmente preciso?» em vez de responder
às pressões da sociedade materialista.

Mas como os preceitos nos ajudam a perceber estes três fogos? De uma
certa forma o que a nossa disciplina monástica nos oferece é um modo
(atitude) com o qual podemos observar o desejo mal ele desponte.
Deliberadamente assumimos uma forma de alerta, prevenindo de seguirmos
todos os nossos desejos e, quando os identificamos, repararmos como
mudam. Normalmente, quando somos apanhados no processo dos desejos, não
existe qualquer noção de objectividade, pois há a tendência para
ficarmos totalmente identificados com eles, sendo então, muito difícil
detectá"-los e de fazer algo sobre eles, em vez de sermos arrastados por
eles.

Assim, como no caso da luxúria ou da aversão, podemos reconhecer que
estas são energias naturais que todos têm. Não estamos a dizer que é
errado, por exemplo, ter desejo sexual ou mesmo de o seguir nas
apropriadas circunstâncias, mas reconhecemos que isso é para um
propósito particular e que trará um certo resultado. Como monges e
monjas nós decidimos que não queremos ter crianças. Também reconhecemos
que o prazer da gratificação é muito fugaz, em relação às possíveis
implicações a longo prazo e responsabilidades. Então escolhemos não
seguir o desejo sexual. Contudo, isto não significa que não o
experienciamos; que assim que rapamos as nossas cabeças e que pomos um 
manto (hábito) parámos imediatamente de experienciar qualquer tipo
de desejo. Na verdade, o que pode acontecer é a nossa experiência desses
desejos aumentar quando vivemos num mosteiro. Isto acontece porque na
vida de leigos, podemos fazer todo o tipo de coisas para nos sentirmos
bem, normalmente sem termos consciência do que estamos a fazer. Algumas
vezes, existe apenas uma subtil noção de desconforto, seguida por um
movimento rumo ao exterior, de modo a alcançar algo que nos alivie,
movimentando"-nos de uma coisa para a outra. No mosteiro já não é tão
fácil fazer isto. Nós ``atamo"-nos'' deliberadamente de modo a podermos
olhar as nossas motivações, energias ou desejos que de outra forma nos
manteriam normalmente em movimento.

Agora poderão perguntar: Mas que tipo de liberdade é esta? Prendemo"-nos
numa situação em que somos constantemente reprimidos, tendo sempre que
nos conformar? Temos sempre de comportar"-nos de uma determinada maneira;
de fazer vénias de uma determinada maneira e em determinadas alturas;
cantar a uma determinada velocidade e tom; sentar num lugar específico
ao lado de pessoas ou atrás, no meu caso de Ajahn Sundara, durante os
últimos quinze anos!\ldots{} Que tipo de liberdade é esta?

Isto traz liberdade à escravidão do desejo. Mais do que sem qualquer
esperança, cegamente ser puxado de um lado para o outro pelo nosso
desejo, somos livres para escolher actuar das maneiras que são mais
apropriadas, em harmonia com aqueles ao nosso redor.

É importante perceber que ``libertação dos desejos'' não significa ``não
ter desejos''. Poderíamos sentir"-nos culpados debatendo"-nos
interiormente se pensássemos dessa forma. Tal como disse anteriormente o
desejo faz parte da natureza, apenas tem sido distorcido como resultado
do nosso condicionalismo, do nosso crescimento, dos valores da sociedade
e da educação. Não nos veremos livres dele sem mais nem menos - apenas
porque queremos, ou porque sentimos que não deveríamos ter desejo; é
requerida, na verdade, uma abordagem mais subtil.

A forma monástica e os preceitos ajudam"-nos a criar um espaço pacífico
em torno dessas energias de desejo, de forma a, que tendo surgido, elas
possam extinguir"-se por falta de combustível. É um processo que requer
grande humildade, pois primeiro temos que reconhecer que o desejo está
lá, o que pode trazer muita clareza sobre os nossos defeitos.
Particularmente na vida monástica, os nossos desejos podem ser
extremamente mesquinhos; o sentido do nosso eu pode ser trazido à tona
em coisas muito triviais. Por exemplo, pode ser que tenhamos uma ideia
convicta sobre a forma como se devem cortar as cenouras; assim, se
alguém sugere que o façamos de maneira diferente podemos ficar muito
alterados e tomar uma atitude defensiva! Então precisamos ser muito
pacientes, muito humildes.

Felizmente, existem alguns simples pontos de referência, ou Refúgios,
que pode providenciar segurança e uma noção de perspectiva, no meio do
mundo caótico dos nossos desejos. Estes Refúgios são, evidentemente,
Buddha, \emph{Dhamma}, \emph{Sangha.} O Buddha, o nosso professor, é
também aquele que existe interiormente em nós, que vê as coisas com
clareza, não sendo confundido ou perturbado pelas sensações impressas; o
\emph{Dhamma}, o ensinamento ou a verdade, torna as coisas como são na
realidade, bastante diferentes das nossas ideias sobre elas; e o
\emph{Sangha}, a linhagem ou comunidade daqueles que praticam,
impulsiona a nossa aspiração de viver de acordo com aquilo que sabemos
ser verdade, em vez de seguir todos os tipos de impulsos confusos e
egoístas que podem emergir.

O Buddha sugeriu algumas formas simples de transformar a nossa vida
nesse sentido. Essas são chamadas as ``Fundações da Consciência''. Uma
das que eu uso bastante na minha prática é a plena atenção do corpo. O
nosso corpo pode ser um bom amigo, pois ele não pensa! A mente, com os
seus pensamentos e conceitos pode sempre confundir"-nos, mas o corpo é
muito simples e podemos notar como é que ele está no momento. Por
exemplo, se alguém age ou fala de uma forma que eu me sinto intimidada,
eu posso notar a minha reacção instintiva, que é criar tensão numa
atitude defensiva, e talvez responder de um modo agressivo. Contudo,
quando eu estou consciente do processo eu posso escolher não reagir
dessa maneira. Em vez de respirar fundo, e de seguida empolgar"-me, eu
posso concentrar"-me na expiração relaxando de modo a tornar"-me uma
presença menos ameaçadora para a outra pessoa. Se, através da plena
atenção, eu poder abrir mão da minha atitude defensiva, os outros também
podem relaxar em vez de se perpetuar o processo de reactividade. Deste
modo podemos trazer um pouco de paz ao mundo.

\enlargethispage{2\baselineskip}

As pessoas que visitam mosteiros muitas vezes mencionam a pacífica
atmosfera que lá encontram. Mas isto não é porque todos se sentem muito
pacíficos ou experienciando a Graça ou felicidade continuamente; eles
podem estar a experienciar todos os tipos de coisas. Na verdade, uma
irmã disse que ela nunca havia experienciado tanta fúria homicida ou tão
poderosos sentimentos de luxúria até ela ter entrado para a comunidade
monástica (\emph{Sangha})! O que é diferente num mosteiro é a prática.
Então seja o que for que os monges e as monjas estejam a passar, eles
estão pelo menos, a fazer o esforço de terem isso presente; suportar com
paciência, contrariando o sentir de que não deveria ser assim ou de
tentarem mudar.

A forma monástica providencia uma situação, na qual, a renúncia e a
restrição são as condições próprias para o surgimento de sentimentos
ardentes; mas também existe a presença de outros \emph{samanas} que
ajudam a reafirmar a confiança. Quando estamos realmente atravessando
algum processo mais difícil, podemos falar com um irmão ou irmã mais
experiente cuja resposta será provavelmente algo do género: Oh sim, não
te preocupes, isso vai passar! Isso também me aconteceu! É normal, é
simplesmente parte do processo de purificação. Sê paciente. Então
encontramos a confiança para continuar, mesmo quando tudo parece
desmoronar"-se, ou tudo parece completamente louco dentro de nós.

Indo a um mosteiro, encontramos pessoas que querem olhar e perceber a
origem causadora da ignorância humana, do egoísmo e de todas as coisas
abomináveis que acontecem no mundo; pessoas que querem olhar dentro de
seus próprios corações e testemunhar a avareza e a violência que outros,
lá fora, estão sempre preparados para criticar. Através de experienciar
e de saber essas coisas aprendemos como fazer paz, aqui mesmo em nossos
corações, de forma, a que elas cessem. Então talvez, em lugar de reagir
perante a ignorância da humanidade e contribuir para a confusão e
violência que vemos à nossa volta, sejamos capazes de agir ou falar com
sabedoria e compaixão, de maneira que ajude a trazer uma noção de bem -
estar e harmonia entre as pessoas.

Então não é uma fuga, mas uma oportunidade de darmos a volta e
enfrentarmos todas as coisas que tínhamos evitado em nossas vidas. Com
calma e corajosamente, compreenderemos as coisas com mais clareza,
começando a libertação das nossas dúvidas, ansiedades, medo, avareza,
ódio e de todo o resto, que constantemente nos prendem em reacções
condicionadas. Aqui temos o suporte de bons amigos e uma disciplina e
ensinamentos para nos ajudar a mantermo"-nos num percurso no qual por
vezes parece impossível continuar!

Possamos todos realizar a verdadeira liberdade.


\vfill
{\raggedleft\itshape\small
  Tradução de Appamādo Bhikkhu
\par}