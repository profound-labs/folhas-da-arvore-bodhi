\chapterAuthor{Ajahn Sumedho}
\chapter{Reparar no Espaço}
\tocChapterNote{Ajahn Sumedho}

A maior parte do nosso sofrimento vem da actividade pensante habitual.
 Se tentamos pará"-la por termos aversão ao pensar, não conseguimos, e
continuaremos a pensar e a pensar. Então o importante não é vermo"-nos
livres dos pensamentos, mas compreendê"-los. E nós fazemos isto através
da concentração nos espaços existentes na mente em vez de nos
concentrarmos nos pensamentos.

Em meditação podemos estar alerta e com atenção; é como ouvir"-estar com
o momento tal como ele é -- simplesmente ouvir. O que estamos a fazer é
trazer à consciência as coisas tal como são, reparando no espaço e na
forma -- o Incondicional e o Condicional. Por exemplo, podemos prestar
atenção ao espaço numa sala. A maioria das pessoas provavelmente não
iria reparar no espaço; elas olhariam para as coisas -- as pessoas, as
paredes, o chão, a mobília. Mas, para observar o espaço, o que fazemos?
Retiramos a nossa atenção das coisas direccionando-a para o espaço. Isto
não significa livrarmo"-nos das coisas, ou negar"-lhes o direito que elas
têm de estarem ali. Significa apenas, que não nos devemos concentrar
nelas, nem dispersar continuamente a nossa atenção de uma coisa para a
outra.

O espaço numa sala é pacífico. Os objectos de uma sala podem excitar,
repelir ou atrair, mas o espaço não tem qualidades que excitem, repelem
ou atraem. Mas mesmo que o espaço não atraia a nossa atenção, podemos
ser completamente conscientes dele, e tornamo"-nos conscientes quando já
não estamos absortos nos objectos da sala. Quando reflectimos no espaço
da sala sentimos uma sensação de calma pois todo o espaço é o mesmo: o
espaço à tua volta e o espaço à minha volta não são diferentes. Não é
meu, não posso dizer ``este espaço pertence"-me'' ou ``aquele espaço
pertence"-te''.

O espaço está sempre presente. Ele torna possível que estejamos juntos,
contidos dentro de uma sala, num espaço que é limitado por paredes mas
que também existe fora da sala. O espaço contém o edifício inteiro, o
mundo inteiro. Então o espaço não é limitado por objectos de qualquer
espécie; não é limitado por nada. Se quisermos podemos ver o espaço como
limitado dentro de uma sala, mas na realidade o espaço é ilimitado.

\section{Mente ``espacial''}

Reparar no espaço que envolve pessoas e coisas, proporciona uma maneira
diferente de olhar para elas e ao desenvolver esta visão espacial é uma
maneira de nos abrirmos a nós próprios. Quando se tem uma mente
``espaçosa'' existe espaço para tudo. Quando temos estreiteza de mente,
então só existe espaço para algumas coisas. Tudo, então tem de ser
manipulado e controlado, para somente ter aquilo que achamos que é certo
-- aquilo que queremos lá -- tudo o resto tem de ser empurrado para
fora.

A vida com uma visão estreita torna"-se reprimida e restrita: é sempre
uma luta. Existe sempre tensão envolvida, uma vez que é necessária uma
enorme quantidade de energia para manter tudo em ordem, durante todo o
tempo. Se tivermos uma visão estreita da vida, a desordem tem de ser
ordenada por nós e então vamos estar sempre ocupados, manipulando a
mente e rejeitando coisas ou tentando agarrar aquelas que não queremos
largar. Isto é \emph{dukkha} (sofrimento) proveniente da ignorância, sob
a qual não compreendermos as coisas tal como são. Na verdade nós dizemos
``o espaço nesta sala'', contudo, a sala está no espaço -- todo o
edifício está no espaço. Olhando de uma forma, as paredes limitam o
espaço na sala, mas olhando de outra forma, vemos que o espaço é
ilimitado.

O espaço é algo em que não reparamos, pois ele por si só, não apela nem
capta a nossa atenção. Não é como uma flor muito bonita ou como um
desastre terrível; não é algo maravilhoso ou terrível que capte
directamente a nossa atenção. Podemos ficar hipnotizados num instante
por algo excitante, fascinante ou terrível; mas não podemos fazer isso
com o espaço, ou podemos?! Para notar o espaço temos de nos acalmar,
temos que o contemplar. Isto é assim porque o espaço não tem nenhuma
qualidade extrema; é simplesmente espaço. Flores podem ser extremamente
belas, com pétalas brilhantes vermelhas, cor"-de-laranja ou violetas, com
maravilhosas formas que deslumbram as nossas mentes. Alguma outra coisa,
como o lixo, pode ser feio e repelente. Apesar de não ser muito notório,
sem o espaço não haveria nada mais. Não seríamos capazes de ver nada. Se
enchermos um quarto com coisas de tal forma que pareça sólido, ou mesmo
enchendo-o de cimento, não sobrará mais espaço no quarto. Então, claro,
não podemos ter maravilhosas flores ou qualquer outra coisa; seria
apenas um grande bloco. Seria inútil, não é? Então precisamos de ambos,
precisamos de apreciar a forma e o espaço. Eles são o perfeito casal, o
verdadeiro casamento, a perfeita harmonia -- espaço e forma. Podemos
contemplar espaço e forma e, da perspectiva alargada que então
desenvolvemos, surge a sabedoria.

\section{O som do silêncio}

Podemos aplicar esta perspectiva à mente utilizando a palavra ``Eu'' de
uma forma consciente para ver o espaço como um objecto. Na mente podemos
ver que existem pensamentos e emoções -- as condições mentais -- que
surgem e que desaparecem. Normalmente somos deslumbrados, repelidos ou
cercados por esses pensamentos e emoções. Passamos de uma coisa para
outra, reagindo, controlando, manipulando ou tentando livrarmo"-nos
delas. Então nunca temos qualquer perspectiva em nossas vidas.
Tornamo"-nos obcecados, quer com a repressão, quer com a indulgência
dessas condições mentais -- somos apanhados nesses dois extremos. Com a
meditação temos a oportunidade de contemplar a mente. O silêncio da
mente é como o espaço de uma sala. Está lá sempre, mas é subtil -- não
se mostra. Como não tem qualquer qualidade extrema que estimule ou
prenda a nossa atenção, temos de estar atentos para o notar. Uma forma
de focar a atenção no silêncio da mente é reconhecendo o som do
silêncio. Podemos usar o som do silêncio (o som primordial, o som da
mente ou qualquer outra coisa que lhe queiramos chamar) com grande
destreza, trazendo-o à tona e prestando"-lhe atenção. Ele tem um tom
muito elevado que é bastante difícil de descrever. Mesmo que tapemos os
ouvidos ou se estivermos até debaixo de água podemos ouvi"-lo. É o som de
fundo que é independente do aparelho auditivo. Sabemos que é
independente, pois podemos ouvi"-lo mesmo que tenhamos os ouvidos
tapados.

Ao concentrarmos a nossa atenção no som do silêncio por um período,
começamos realmente a conhecê"-lo. Desenvolvemos um modo de conhecimento
com o qual podemos reflectir. Não é um estado concentrado onde nos
absorvemos; não é um tipo supressivo de concentração. A mente está
concentrada num estado de equilíbrio e abertura, mais do que absorvida
num objecto. Podemos usar essa concentração aberta e equilibrada como
uma forma de ver as coisas em perspectiva, uma forma de deixar as coisas
seguirem o seu rumo em vez de as agarrarmos.

Agora, quero realmente que investiguem este modo de conhecimento para
que percebam como devem deixar as coisas fluir, em vez de apenas ter a
ideia de que assim deve ser. Vocês podem sair dos ensinamentos budistas
com a ideia de que devem soltar as coisas (tensões mentais), de que
devem deixá"-las ir. Então quando descobrirem que não o conseguem fazer,
facilmente, podem pensar: ``Oh não, eu não consegui abrir mão das
coisas!''. Este tipo de julgamento é outro problema do ego em que se
pode cair: ``apenas os outros conseguem abrir mão mas eu não consigo. Eu
deveria abrir mão das coisas porque o Venerável Sumedho disse que toda a
gente deveria largar as coisas''. Este julgamento é outra manifestação
do nosso ``pequeno eu'', não é? E é apenas um pensamento, uma condição
mental que existe temporariamente dentro da capacidade espacial da
mente.

\section{Espaço à volta dos pensamentos}

Tomemos a simples frase ``Eu Sou'', e reparem, contemplem, e comecem a
reflectir no espaço que envolve estas duas palavras e, sustenham a vossa
atenção. Olhem para o próprio pensamento, examinando e investigando
realmente. Agora já não se podem observar a vós próprios a pensar como
habitualmente, pois assim que notarem que estão a pensar, o pensamento
pára. Vocês podem continuar preocupados, ``Pergunto"-me se isto irá
acontecer. Então se isso acontece\ldots{} murmurando, murmurando: Oh, estou a
pensar! - e então o pensamento pára''.

Para observar o processo dos pensamentos, pensem deliberadamente em
algo: escolham um pensamento como ``Eu sou um ser humano'', e
simplesmente olhem para ele. Se olharmos para o seu princípio, podemos
reparar que antes de dizer, ``Eu'', existe uma espécie de espaço vazio.
Então se pensarem na vossa mente, ``Eu - sou - um - ser - humano'',
verão o espaço entre as palavras. Não olhamos para o pensamento para
sabermos se temos pensamentos inteligentes ou estúpidos. Em vez disso,
estamos a pensar deliberadamente de forma a notar o espaço à volta de
cada pensamento. Desta forma começamos a ter uma perspectiva da natureza
impermanente dos pensamentos.

Isto é apenas uma maneira de investigar para então podermos reparar no
vazio quando não existem pensamentos na mente. Tentem focar"-se nesse
espaço, vejam se conseguem realmente concentrar"-se nele antes e depois
do pensamento. Por quanto tempo o conseguem fazer? Pensem ``Eu sou um
ser humano'', e mesmo antes de formular tal pensamento fiquem nesse
espaço imediatamente antes de o dizerem. Então isso é atenção plena, não
é? A vossa mente está vazia mas existe também a intenção de pensar um
determinado pensamento. Então pensem, e no fim do pensamento, tentem
percorrer o espaço que lhe sucede. A vossa mente continua vazia?

A maior parte do nosso sofrimento vem da actividade pensante habitual.
Se tentamos pará"-la pela aversão que temos a pensar, não conseguimos;
simplesmente continuamos a pensar, a pensar, a pensar\ldots{} Então o
importante não é libertarmo"-nos do pensamento, mas compreendê"-lo. E
fazemos isto através da concentração no espaço existente na mente, em
vez de nos concentrarmos nos pensamentos.

As nossas mentes podem ser apanhadas por pensamentos de atracção ou de
aversão a determinados objectos, mas o espaço circundante desses
pensamentos não é atractivo ou repulsivo. O espaço à volta de um
pensamento atractivo e o espaço à volta de um pensamento repulsivo não é
diferente, ou é? Concentrando"-nos no espaço entre os pensamentos,
passamos a ser menos apanhados nas nossas preferências no que diz
respeito aos pensamentos. Assim, se notarem que um pensamento de culpa,
auto"-compaixão ou paixão continua a surgir, podem trabalhar com isso
desta forma -- pensando nisso deliberadamente, trazendo-o realmente a um
estado consciente e reparando no espaço à sua volta. É como olhar para o
espaço numa sala: nós não vamos a uma sala para olhar para o espaço,
pois não? Estamos simplesmente atentos a ele, porque está lá sempre. Não
é nada para encontrar dentro do armário ou na próxima sala, ou debaixo
do chão -- está mesmo aqui e agora. Apenas temos que nos abrir à sua
presença: e então reparamos que ele está lá. Mas, se estivermos
concentrados nas cortinas ou nas janelas ou nas pessoas, não vamos
reparar no espaço. Mas não temos de nos livrar de todas as coisas para
notarmos o espaço. Em vez disso podemos simplesmente abrir"-nos à sua
presença. Em vez de concentrarmos a nossa atenção apenas numa coisa,
abrimos a nossa mente completamente. Não estamos a escolher um objecto
condicionado, mas em vez disso estamos conscientes do espaço no qual
esse objecto condicionado existe.

\section{A posição de Budha -- conhecedor}

Com a mente podemos apelar interiormente para a nossa ``abertura de
atenção''. Quando temos os olhos fechados podemos ouvir as vozes
internas que sussurram na mente. Elas dizem, ``Eu sou isto\ldots{}eu não
deveria ser assim''. Podemos usar essas vozes para irmos para o espaço
entre os pensamentos. Escusamos de fazer um grande problema das
obsessões e medos que percorrem a nossa mente; podemos abrir a nossa
atenção e ver essas obsessões e medos como condições mentais que vêm e
que vão no espaço. Desta forma, mesmo um mau pensamento pode levar"-nos
para o vazio. Esta forma de conhecimento é muito útil uma vez que põe
fim à luta mental na qual tentamos livrar"-nos dos ``maus'' pensamentos.
Podemos ser justos com o ``diabo''. Sabemos que ele é algo impermanente.
Ele surge e desaparece na mente. Então não teremos de fazer nada dele.
Diabos ou anjos -- é tudo o mesmo no que se refere à nossa atitude.
Antes, ao ter um mau pensamento começávamos logo a criar um problema:
``O diabo anda atrás de mim. Tenho de me livrar dele''. Agora, quer,
seja tentar livrar"-nos do diabo, quer, seja tentar agarrarmos os anjos,
tudo é sofrimento (\emph{dukkha}). Se adquirirmos esta postura do Buddha
-- conhecedor, conhecendo as coisas tal como são, então tudo se torna
\emph{Dhamma} (verdade, ensinamento). Tudo se torna na verdade do
caminho que É. Vemos que todas as condições mentais surgem e cessam -- o
bom juntamente com o mau, o útil com o inútil. Isto é o que entendemos
por reflexão -- começamos a ver as coisas como elas são, em vez de
assumirmos que estas deveriam ser de qualquer forma específica, e
passamos então, a contemplar, a reparar. O meu propósito não é dizer"-vos
como é, mas encorajar que descubram pessoalmente. Não se ponham por aí a
dizer: ``O Venerável Sumedho disse"-nos como é''. Eu não estou a tentar
convencer"-vos de um qualquer ponto de vista, eu estou apenas a
apresentar um caminho para ser considerado por cada um, um caminho de
reflexão na vossa própria experiência, uma maneira de conhecerem a vossa
própria mente.

\bigskip

\QA{Pergunta:} Algumas pessoas falam há cerca dos \emph{janas,} estados de
absorção na meditação, no Budismo. O que são esses estados e como é que
eles se enquadram nos conceitos de plena atenção, \emph{insight} e
reflexão?

\QA{Resposta:} Os \emph{janas} ajudam a desenvolver a mente. Cada \emph{jana}
é um refinamento da consciência e, como um conjunto, eles ensinam"-nos a
concentrar a nossa atenção em objectos cada vez mais subtis. Através da
plena atenção e da reflexão, não pela força de vontade, tornamo"-nos
bastante conscientes da qualidade e do resultado daquilo que estamos a
fazer. Desenvolvemos grande habilidade nesta prática e experimentamos a
graça que vem da absorção em estados cada vez mais refinados de
consciência. O Buddha recomendou a prática dos \emph{janas} como um
meio, mas não como um fim em si mesmo. Se deixarmos que se torne um fim
em si mesmo, tornamo"-nos apegados ao refinamento e sofremos, visto uma
grande parte da nossa vida humana não ser refinada e sim bastante
grosseira. Em contraste com a prática dos \emph{janas,} a meditação
\emph{vipassanā} (meditação de \emph{insight}) foca"-se na realidade das
coisas, a impermanência das condições e o sofrimento que vem com os
apegos. A meditação \emph{vipassanā} ensina"-nos que o caminho para sair
do sofrimento não é através do aumento do refinamento da consciência,
mas sim através de não nos agarrarmos a nada -- nem mesmo ao desejo de
nos absorvermos em qualquer nível de consciência.

\bigskip

\QA{Pergunta:} Então \emph{insight} é reflectir na avareza da mente?

\QA{Resposta:} Sim, o \emph{insight} leva a que reparemos sempre no resultado
da avareza e desenvolve a Correcta Compreensão. Por exemplo a
contemplação das Quatro Nobres Verdades permite obter a Correcta
Compreensão e então essa visão"-pessoal e esse conceito"-pessoal são
penetrados com sabedoria. Quando existe Correcta Compreensão não estamos
a praticar os \emph{janas} a partir de uma atenção egoísta; eles
representam um meio hábil de cultivar a mente, mais do que uma tentativa
de alcançar um resultado pessoal. As pessoas enganam"-se quando abordam a
meditação com a ideia de alcançar ou realizar algo. Isso vem sempre do
problema básico da ignorância e da visão pessoal, combinados com o
desejo e a tentativa de agarrar"-se a algo. E isso cria sempre
sofrimento.

\vfill
{\raggedleft\itshape\small
  Tradução de Appamādo Bhikkhu
\par}