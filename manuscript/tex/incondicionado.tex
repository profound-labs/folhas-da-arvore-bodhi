\chapterAuthor{Ajahn Sumedho}
\chapter{O Incondicionado}
\tocChapterNote{Ajahn Sumedho}

Ora, então já só temos Sexta e Sábado, e Domingo o retiro acaba.

Reflictamos então sobre o resultado do retiro até aqui, não em termos de
``bom'' ou ``mau'', pois por vezes as pessoas dizem: ``tive um mau
retiro ou um bom retiro''. Mas mau retiro também é bom e se houvesse só
bons retiros nunca se aprenderia nada.

As convenções religiosas têm as suas formas, as quais são precisamente
convenções. Assim por exemplo, questões sobre os budistas não
acreditarem em Deus é um dos casos, em que pode confundir-se e misturar
convenções. Tal como diferentes linguagens são só convenções. Quanto à
palavra ``Deus'', o que é que isso quererá significar? O que acontece na
generalidade é que a palavra ``Deus'' é usada como se as pessoas
concordassem e pensassem todas da mesma maneira. O que é que se quererá
dizer no contexto cristão ou no contexto judaico? ``Deus'' é a palavra
utilizada. Mas no contexto budista por exemplo fala-se em ensinar deuses
e homens, não é de ensinar Deus e os homens... deuses neste caso não
equivale a Deus no sentido Cristão da palavra. No entanto, tudo isto são
palavras que procedem de formas, onde existe uma concórdia de linguagem,
principalmente, quando nos reportamos em termos de ``desenvolvimento
espiritual'', e assim misturamos tudo (religiões, crenças), gerando
confusão. Portanto, ao ensinar dentro do contexto budista, deve-se ficar
dentro da terminologia budista.

Mas este Retiro não é sobre comparação de religiões, mas sobre meditação
e de como saber usar a convenção budista.

Quer o Buddha, tenha acreditado ou não em ``Deus'', esse nunca foi o
tipo de abordagem que assumiu em relação à vida espiritual, mas fê-lo
num contexto diferente, quase na direcção oposta, o que deu realce às
``Quatro Nobre Verdades''. E estas não são verdades metafísicas, pois
não se trata de tentar definir a ``realidade última'' ou até de usar
termos para outros factores que não os negativos como o
``Incondicional'', o ``Incriado'', ``Cessação'', ``\emph{Nibbāna}''
(\emph{Nirvāna)}.

As religiões teístas por seu lado, geralmente começam muito mais por
doutrinas metafísicas, ``eu acredito em Deus'', contudo, no Budismo,
pelo menos na Tradição Theravada, a preocupação é: ``o sofrimento
existe'', ``a sua causa existe'', ``a sua cessação existe'' e ``o
caminho para a cessação do sofrimento existe''. Ou seja, toma a atitude
inversa em oposição à religião teísta. Ora a ``Primeira Nobre Verdade''
é baseada numa experiência bastante comum e certamente que não é uma
verdade metafísica, é uma realidade existencial, não é assim? O
sofrimento é uma experiência comum a todos os seres humanos, todos os
seres sencientes. E pondo isso dentro do contexto de uma ``Nobre
Verdade'', que nobreza é que se pode encontrar no sofrimento? A mim
parece-me um facto maléfico, em termos da minha mente americana, o que
eu quero é fugir disso, como é que nos livramos disso? Como é que saímos
dum determinado sofrimento? Então, esta abordagem do Buddha é, para
compreendermos o sofrimento e como nós o criamos. Portanto, ao dar uma
palestra, estou constantemente a apontar, a definir as coisas como são;
a experiência existencial de sentar, respirar, sentir, de estar nesta
forma sensível. Eu tenho estado num corpo humano, eu tenho estado
consciente. Assim, não estou a dizer-vos o que deviam pensar sobre isto,
mas a indicar e a encorajar para o despertar e a observar para serem
mais conscientes. Isto é consciência, acordar para a realidade deste
momento, assim tal como é, que inclui felicidade, sofrimento ou o que
quer que seja, que estejam a experienciar agora mesmo.

A ênfase que se dá a toda a fenomenologia condicional é impermanente. E
o ``incondicionado'', o ``incriado'', o ``não-nascido'', o
``não-originado'' existe, logo existe também a saída para fora do
``condicionado'', do ``criado'', do ``nascido'', do ``originado''.

Ou seja, toda esta forma de ensinar, de indicar, de falar, não é para
acreditarem, eu não estou a pedir para acreditarem no que digo, mas sim
para observarem as condições - a maneira como é - conscientemente
experimentando dentro da vossa forma humana, dentro da maneira como a
vossa mente trabalha; das emoções que estão a ter, das energias, das
experiências energéticas e, que podem estar a experienciar agora. E não
é ajuizar se é bom ou mau, mas ser o que é, deste modo, este ser o que
é, assim tal e qual como é.

O Buddha depois da sua Iluminação sempre se referiu a Si próprio como o
``\emph{Tathāgata}'', ``\emph{Tathāta}'', esta palavra ``\emph{tathā}''
em Pāli significa ``a integridade do que é'' ou ``aquilo tal qual é'', o
``como é''. Ele não se referiu a Si, dizendo: ``Eu costumava ser o
Príncipe Siddhārtta e o meu Pai era um Rei, a minha Mãe era uma Rainha.
Frequentei as melhores escolas, casei, tive um filho e depois deixei-os,
quando optei por ser asceta por seis anos, onde me vi na futilidade de
me maltratar e, quando me sentei debaixo de uma árvore tornei-me
iluminado''. Aquilo que é o agora, ``\emph{tathāta}'',
``\emph{tathāgata}'' é ``assim se torna Um que é o agora'', é como uma
referência, aquilo que é presente. Não é o Príncipe Siddhārta, não é o
asceta Gautama, nem é sequer o Buddha, no sentido de dizer ``eu sou o
Buddha'' num acto de proclamar-se Ele próprio a esse nível. Ora tomemos
aqui atenção à erudição da palavra, ``\emph{Tathāgata''} tornou-se
frequentemente, em várias tradições budistas, uma espécie de título
superlativo, quando na realidade significa ``aquilo que é precisamente
agora''. Então este ``\emph{tathāta}'', ``\emph{tathā}'' e
``\emph{tathāgata}'' ou qualquer coisa com esse prefixo, é traduzido
como ``a qualidade tal d'aquilo que é agora'', não é uma pessoa, não é
alguém com uma história, um passado, com uma auto-biografia,
credenciais, status, sucesso ou seja o que for, mas que é aquilo, que é
``agora''.

E eu sempre achei esta perspectiva bastante interessante, porque implica
aprender o verdadeiro significado da linguagem. Vivi na Tailândia
durante os meus estágios iniciais de meditação e, portanto, eu tive que
aprender meditação através dos termos da Língua Tai, de traduções
Tai-Inglês e de Pāli, e então, é claro que a Língua tailandesa
desenvolveu-se culturalmente sob a influência e o contacto com o
Budismo, incluindo bastantes palavras do Pāli, à semelhança do Latim com
o Inglês o qual inclui bastantes palavras. E, então aprender noutra
Língua, achei um grande desafio, porque na nossa própria Língua pode-se
pensar e tomar como certo que percebemos tudo. O Inglês é a minha Língua
natal, é Inglês Americano, mas é um hábito-língua que se aprende
facilmente, sendo do género de se ir absorvendo, como quando se ouve na
infância a nossa Mãe falar ou qualquer outra pessoa.

Então aprender meditação noutra Língua como a tailandesa, achei uma
grande revelação, mas com certa dificuldade em perceber, pois não era a
minha Língua natal. Tive que considerar o sentido e o significado das
palavras. A Língua tailandesa é uma Língua de características psíquicas,
onde há todo um leque de termos, por exemplo sobre \emph{``Citta''}
(mente-coração) e \emph{``Jay''} (vencer-conquistar); contém diferentes
níveis de felicidade e miséria, enlevo ou depressão, intermináveis tipos
de combinações para descrever sentimentos emocionais e estados mentais.
Portanto, traduzir isso para o inglês e aprender como usar a Língua
inglesa dentro do contexto tailandês e entender como realmente funciona,
não só em termos de expressão gramática ou de estrutura da Língua, mas
também psiquicamente. Compreender, como de facto, as palavras têm
efeito, afectando a consciência, na medida em que a consciência é o
alicerce do Ser. Isto é consciência, a Língua, as palavras, os
conceitos, os eus.

Como por exemplo dizer ``Eu''...só este pronome pessoal ``Eu''...e então
nós temos ``eu'' e ``meu''. É evidente que ``Eu'' não implica
necessariamente egoísmo quando aponta à realidade do Ser. Porém, quando
entro em pontos de vista e opiniões fortes, egoísmo e obsessão, quando
fico obcecado com o ``eu'', obcecado comigo, com o meu ser, com o que é
meu...então estas palavras ``eu'' e ``meu'' são extremamente poderosas
para se reflectir sobre elas; podem conduzir a este sentido de
importância pessoal, do que eu penso, de que estes são os meus
pensamentos, esta é a minha vida, os meus direitos...esta é a era dos
direitos, exigindo os meus direitos.

E, depois, este ``Eu'' pode facilmente conduzir a este sentimento de
``individualidade pessoal''. Mas, ``Eu'' também pode querer dizer
``não-pessoal'' - não um, ``eu'' pessoal separado, em termos de alguém
com uma história, um corpo ou coisa do género - mas mais para indicar a
realidade do estar presente. ``Eu sou'' é uma afirmação honesta, e
quando nos exprimirmos com palavras sobre nós, é sempre com, ``Eu sou''.

Eu refiro-me a isto na Lenda do Buddha, a seguir à Iluminação, quando
Ele se dirige para Varanasi e encontra o asceta que Lhe pergunta, «Vós
pareceis particularmente resplandecente e radiante, o que foi que
acabastes de descobrir?». E Buddha responde simplesmente: «Eu Sou O
perfeitamente iluminado». Eu costumava interrogar-me sobre isto e dizia
«se Ele estava iluminado, como poderia dizer uma coisa dessas?».
Sinceramente acho que é algo perfeitamente estúpido de dizer se me
perguntassem. Imaginem, alguém vos pergunta e vocês respondem «Eu Sou
o/a perfeitamente iluminado/a», pois quem quer que dissesse isso eu não
confiaria. Eu já cheguei a conhecer pessoas aqui em Amarāvati, uma entre
as quais se proclamou Deus. Mas, na verdade, o Buddha disse o que disse,
pelo menos na escritura, não sei até que ponto a história é fidedigna,
mas é uma boa história e o asceta foi-se embora e não acreditou. Então,
será que o Buddha estava a dizer uma mentira, ou foi só presunção? Ou
foi o ``Eu'' não pessoal? Portanto, isto é só uma reflexão sobre o uso
deste pronome. Será que ``Eu'' sempre se refere a mim como uma pessoa?
Ou é antes um testemunho da realidade? ``Eu sou o iluminado'', ``Eu sou
o caminho'', ``Eu Sou...''... e claro, para muitos de nós parece-se como
sendo o ego, porque geralmente é assim que se usa a palavra, o pronome
``eu''. E é óbvio, se alguém sente o seu ego envolvido, então vai
interpretar como uma espécie de objectivo pessoal a ser alcançado,
tornando-se personalidade... ou não será?

Por exemplo, no Advaita (Vedānta), eles usam o ``Eu Sou'' como ``quem
sou eu'' ou usam a reflexão no ``Eu Sou'' de uma forma sábia. E então os
budistas dizem «oh, no Advaita eles têm o ``Eu'' superior e o ``Eu''
inferior e o ``\emph{Ātman}'' e é tudo uma porcaria, nós os budistas é
que estamos certos...». Mas, será que realmente sabemos o que estamos a
dizer? Será que compreendemos verdadeiramente a nossa própria convenção?
Porque temos preconceitos e juízos tendenciosos a respeito de outras
religiões, a respeito de outras convenções das quais não nos inteiramos
nem compreendemos, nem as praticamos para as perceber, simplesmente as
julgamos do ponto de vista da nossa própria convenção. É como a
presunção de ser inglês não é? O Império Colonial Britânico, com certa
presunção, ensinava as pessoas a serem civilizadas, porque a nossa
civilização tinha muito mais brio do que qualquer outra, quando na
realidade não compreendiam ninguém, e isto é ser o ``eu'' e o ``meu'',
que são convencidos, onde existe preconceito tendencioso e prejudicial
que vem da ignorância.

E, então, apontando para a consciência, qualquer que seja a nossa raça,
religião, classe ou o quer que seja, na realidade nós somos seres de
consciência, isto é consciência. E a linguagem é algo que aprendemos e
usamos em consciência. E neste Retiro aquilo para o qual aponto, não é
para a convenção como algo que tenham que dominar, mas sim para o modo
como devem usar a convenção do Budismo Theravada. Assim, não é para
criar ou acrescentar mais presunção à que já possa haver. É usar isso,
exactamente para ver e penetrar através da presunção, através da
ignorância, preconceito ou apego que possa existir, é ir mais além, para
ver o sofrimento que experimentamos ou que eu experimento quando estou
agarrado a ideias, opiniões, posições e sentimentos de auto-importância.
Deste modo, qualquer conceito, pensamentos ou atitudes que eu possa ter,
ou nem ainda estar consciente, acaba quando começo a observar e
reflectir no que realmente estou a pensar e a sentir. Eu costumava ser
convencido de que não era convencido, porque eu não suporto gente
convencida. Foi quando alguém falou em presunção dizendo que determinada
pessoa era muito convencida, que eu comecei a reparar na forma como eu
costumava julgar as outras pessoas como convencidas; reparei como estava
a ser bem convencido, chegando a dizer: «eu não sou assim».

Portanto, o ``Eu Sou'', como já indiquei anteriormente, se o ponho ao
nível pessoal «Eu sou Ajahn Sumedho», deixa então de ser universal e
torna-se pessoal, «Eu sou Americano-Britânico», se o puser em termos
universais pode-se dizer «Eu sou... Amor». O que será isso? Agora aqui
entramos no uso da palavra inglesa ``\emph{Love}'' (Amor) e é claro que
como bem sabem, esta palavra é usada quase para tudo e mais alguma coisa
hoje em dia, mas é uma palavra bem poderosa da Língua inglesa, e fez o
seu caminho para quase todas as línguas no planeta por esta altura. Não
pode, de facto, ser posta de lado. Mas quando falamos sobre ``Amor'' ou
amor romântico, ou amor pessoal, amor incondicional, amor Cristão... o
que é que queremos dizer com esta palavra? Então os budistas usam
``\emph{Metta}'' e fica logo tudo resolvido, não é verdade? Escrevem-se
letras em que não se diz ``\emph{Love}'' (Amor) mas diz-se
``\emph{Metta}''. Pois, na verdade ``\emph{Love}'' (Amor) pode soar
incrivelmente pessoal, como por exemplo em ``being in love'' (estar
enamorado/apaixonado) com uma certa afectividade e intimidade pessoal.
Ou pode significar ``\emph{Metta}'' (Loving/Kindness --
Amabilidade/Gentileza), ou então usamos o termo ``Amor incondicional''
ou será apenas só mais outro tipo de ideia ou grande ideal? «Não seria
tão bom haver amor incondicional?» «Ai o meu amor é incondicional». Soa
bonito e é inspirador, mas ``amor'' muitas vezes na forma como é usado,
significa apenas gostar. Portanto gostar e desgostar, o que muitas vezes
se quer dizer é, ``eu amo este sítio ou esta pessoa'', porém, quando
essa pessoa não faz o que nós queremos já não gostamos dela. Depois
ficamos muito confusos porque queremos amar alguém incondicionalmente,
mas logo, não gostamos do que essa pessoa está a fazer, porque quando se
pensa que essa pessoa está a fazer alguma coisa da qual não gostamos,
deixamos de gostar dela. Assim, a memória ou a percepção é ``ele
falhou'' ou ``traiu-me'' ou ``desiludiu-me'', ou seja, isto é o
sentimento de ``eu'', de ``gostar'' ou de ``desgostar''.

Quando eu uso o termo ``amor incondicional'', qual será a realidade do
amor incondicional neste preciso momento? E, então, a mente pára, não é
verdade?.. ``o que é amor incondicional neste preciso momento? É real ou
será só uma ideia superior? Estarei eu a sentir amor incondicional agora
mesmo ou não? O que será agora? Se é incondicional, é intemporal, não
depende de condições, não depende de empatia, de ser simpático ou
amável, de concórdia, ou aprovação ou de qualquer outra coisa. Num nível
pessoal eu gosto mais de certas coisas do que de outras e não gosto de
outras tantas, aprovo certas facções, desaprovo outras. São
preferências, opiniões e diferentes pontos de vista de acordo com a
forma como eu estou condicionado a pensar, da forma que eu espero e
exijo da vida para comigo próprio.

De alguma forma, na minha própria experiência ao usar a palavra
``Amor'', faço por não o definir ou ficar a pensar demasiado nas
infinitas descrições sobre ``Amor incondicional'', mas reconhecer a
realidade que é. Tive uma experiência sobre isto, que já contei a alguns
dos presentes anteriormente, e que se passou já há alguns anos, sobre
ter estado muito irritado com alguém, muito magoado, enraivecido, muito
perturbado. Assim, sempre que esta pessoa surgia na minha consciência,
eu sentia esta raiva, esta aversão, e então, comecei a observar isto que
chegou a tornar-se extremamente terrível e sufocante, quando eu na
realidade não queria sentir isso, porém era o que estava a ocorrer. E,
então, resolvi escrever a minha aversão. Lembro-me de estar ali sentado
uma tarde inteira a escrever tudo o que me vinha à cabeça, como nunca
antes; afirmações do pior, que nem tentei ser amável ou politicamente
correcto, ou decente, ou qualquer outra coisa, eu simplesmente escrevia
todo o ódio, sem precedente. E finalmente ao acabar, (enchi três
páginas), queimei e enfiei isso pela retrete abaixo.

Mas o significado deste discurso crítico de raiva em três páginas, foi
que eu trouxe à superfície a memória desta pessoa e, depois disso nada
permaneceu. Até na minha mais vívida imaginação, já não havia mais nada
para pensar de sórdido ou horrível sobre aquela pessoa. Desta forma, a
mente ficou vazia. Depois trouxe novamente a imagem desta pessoa à
memória e perguntei, o que é que este vazio queria dizer? E o vazio
disse ``Eu amo-te'' a essa pessoa, à memória dessa pessoa. Isto para mim
foi uma revelação muito importante porque, na realidade, o que jaz sob
esta raiva e cólera, é o Amor. Contudo, não sabemos isto, quando estamos
envolvidos nos detalhes e dentro da própria raiva. Na verdade, se
tentarmos parar esse envolvimento raivoso, pois o que queremos é ser
amáveis com amor e simpatia, podemos, então ser sensatos e entender as
outras pessoas. Posso até pôr-me no lugar delas e compreender porque
dizem coisas terríveis, mas, nem mesmo assim, estou a ser correcto ao
nível da razão, mas ao nível emocional. No entanto, ou me deixo envolver
nisso por pensar, vaguear ou odiar e fico preso na emoção ou então, paro
e não me deixo envolver. Ao reflectir desta forma e tendo até expressado
por escrito a minha tensão mental, trouxe-a à superfície e encarei esta
fúria e raiva. Ao escrever, consegui pelo menos trazer a emoção a um
nível superior de evidência, numa exposição crítica que me permitiu
conhecer e aceitar o que eu estava a sentir e a pensar, de modo a
descrever a fúria, a raiva e a mágoa que eu estava a experimentar. Na
verdade chegou a um fim.

Passado algum tempo, não havia mais nada para dizer e, então houve uma
compreensão, ``amor incondicional subjaz a tudo''. Não nos apercebemos
desta realidade, porque estamos perdidos nos nossos apegos, nos nossos
hábitos, pontos de vista e opiniões, nas nossas ilusões, nas nossas
personalidades.

Se não conhecerem o que é a vossa personalidade ou, se não souberem
verdadeiramente o que são as emoções, vão ajuizar de um nível racional,
``raiva é má e o amor, é bom e eu gosto'', depois também se acredita nos
gostos; ``este é um bom monge, aquele é um mau monge''. Assim, surgem e
desaparecem várias preferências, e entretanto, alguém de quem gostamos
hoje, amanhã podemos já não gostar. O gostar é mesmo algo em que não
podemos confiar. E, então, tomei consciência de que há sempre condições
para se gostar. Quando por exemplo, vemos mães com crianças e as
crianças estão a ser mesmo teimosas, difíceis, abomináveis e
impossíveis, verifica-se que a criança está a ser completamente
insuportável nesse momento, no entanto, as mães não dizem ``eu odeio a
criança'' - pelo menos a maioria - e por debaixo da frustração, do não
gostar e da saturação, está este Amor Incondicional. Digamos que para
mim, não foi um amor romântico, não foi sentimental. Neste sentido, o
Amor Incondicional aceita tudo, percebe, sem condições. Não é como ``eu
amo-te só quando te portares bem, ou se te comportares decentemente e
não humilhares nem envergonhares a família'', ``mas quando fizeres
coisas de que eu não gosto eu já não te amo mais''. Mas é possível
pensar desta maneira, ``eu já não amo mais quem fala mal de mim, quem
espalha rumores, quem me calunia, quem me maltrata...'' e então ``eu já
não te amo mais.'' Ou, outras condições para não se gostar, são por
exemplo situações em que me insultam ou que abusam de mim, já não
consigo gostar dessas pessoas nesse momento. Mas se eu reconhecer e
estiver ciente de que mesmo por debaixo dos meus gostos e antipatias que
surgem e esmorecem conforme as condições, está este Amor Incondicional,
pode-se cultivar a prática de \emph{Metta} (amabilidade -gentileza) ou
Amor Incondicional.

Quando se pratica \emph{Metta} connosco e depois inter-agimos com os
outros, podemos ficar demasiado sentimentais ou mimados e, eu admito,
que algumas formas de comportamento podem-se tornar insatisfatórias para
mim, mas, e se for apenas um tipo de simpatia verbal e pretensiosismo
sentimental? Portanto, é aqui que se pergunta, afinal de contas o que é
na realidade, \emph{Metta}? Quando realmente investigamos o uso de
\emph{Metta}, estamos a aceitar todas as condições, tudo assim tal como
é, sem o gostar ou deixar de gostar, não tendo a ver com aprovar ou
desaprovar. Assim nós emitimos Amor para o Senhor da Morte, para os
Demónios, assim como para os Anjos, os bons e os maus. Não é uma questão
de mais Amor para o Anjos e só um bocadinho para os Demónios, ou uma
questão de percentagem ou de quem merece. Quem será que merece o meu
Amor? É claro que vou ser decente com os Demónios e talvez lhes dê
1\%''. Mas isso não é incondicional, ou será que é? Ou é ser
naturalmente sentimental e pensar mais uma vez naquele modo, ``tu
mereces mais que aquela pessoa'' e então entra-se em maneiras de ver
pessoais, opiniões, gostos, preferências e por aí fora. Amor
incondicional, é esta habilidade para aceitar e isso deve ser feito
durante a própria meditação, no agora, quando estiverem conscientes e a
experienciar emoções desagradáveis. Tentarem cultivar esta habilidade
para aceitar a sensação, este sentimento, esta raiva ou ressentimento,
ou ciúme, ou medo, o quer que seja, sem condições. Nem sequer é uma
questão de aceitar ou livrarmo-nos de, mas é admitir qualquer condição
que seja, de ser o que é e, simplesmente, apliquem isso a vós próprios,
à vossa própria experiência enquanto estão aqui sentados. Quando tiverem
que lidar com negatividade, como raiva e outros sentimentos que possam
surgir, experienciados interiormente, é possível que não gostem.

Deste modo, em meditação, e num Retiro como estes, é que todos temos
esta oportunidade de permitir o medo, pois o medo é muitas vezes uma
emoção que normalmente rejeitamos ou resistimos. Por isso, às vezes
experimentamos medo num sítio como este, que na realidade até é bastante
seguro. Não há aqui nada que cause medo, no entanto, há terror, medo,
ressentimentos e todos os tipos de estados estranhos, negativos que
podemos deixar entrar na consciência. De facto, podemos cultivar esta
atitude do Amor incondicional, ou \emph{Metta}, ou Consciência
(\emph{Sati -- Mindfulness}). Consciência (\emph{Sati - Mindfulness}) é
permitir seja o que for, que seja o que é ou a qualidade que é. Se é
estúpido ou sem sentido nenhum, sujo, ou perverso, ou seja lá o que for.
Poderão querer descrever isso, mas se o fizerem estão a tornar a
situação mais complicada do que é na realidade, porque estão a julgar de
determinada forma. É o que é, a condição começa e acaba. E isto não é
desprezar ou ignorar a qualidade, mas também não é abusar do julgamento
dessa qualidade. A ``Presença Consciente'' (\emph{Awareness)} permite o
discernimento para receber algo. Mas a tendência para julgar a
qualidade, vem mais uma vez da mente crítica, ``isto é mau, isto é bom,
isto está certo, aquilo está errado''.

Então, eu acho que este modo de reflectir, na realidade, é uma
purificação. Quando se consegue fazer isto, mesmo que o que vem à
consciência possa parecer contaminado e impuro, vai permitir a
liberação, a permissão para partir. Estão a libertar-se estes estados
miseráveis. E, através da consciência, o que é que eles fazem? Sim, eles
param, eles vão-se embora. Se isto não for feito, então pode-se
suprimi-los outra vez, e resiste-se, constantemente não permitindo esta
purificação, esta pureza natural, ao tentar controlar e rejeitar o que
não se gosta e, o que não se quer ver, o que não se quer saber, do que
se tem medo, continuando a reprimir.

Portanto, é aqui que se encontram as referências ao Buddha,
\emph{Dhamma,} \emph{Sangha}. Porque é que eu valorizo isso, porque é
que tomo refúgio no Buddha, \emph{Dhamma,} \emph{Sangha}? Porque isto é
uma forma e uma convenção, mas para mim, porque o desenvolvi, isto é uma
referência, uma recordação. Estas três frases: \emph{Buddham Saragnam
Gachami} (eu tomo refúgio no Buddha)\emph{; Dhammam Saragnam Gachami}
(eu tomo refúgio no \emph{Dhamma})\emph{; Sangham Saragnam Gachami} (eu
tomo refúgio no \emph{Sangha}), que me lembram, até daquilo que eu possa
experimentar, de horrível, de maligno ou de assustador, de louco, de
demente, seja qual for a qualidade que possa sentir, mas o meu refúgio,
é ver isso em termos daquilo que é \emph{saõkhārā} (formação mental), a
surgir e a desaparecer. Ver a cessação disso, não é rejeitar, mas
permitir que o seja e de seguida realizar a cessação. Porque essa
``Presença Consciente'', então de facto está a observar; já não se é
mais a pessoa a possuir a condição, é-se Buddho, o Buddha a observar e a
conhecer o \emph{Dhamma}, tudo o que está sujeito a surgir e tudo o que
está sujeito a findar.

Ora, quando algo acaba, o que realmente acontece é largar algo, não é
rejeitar esse algo. Rejeição implica sempre aversão, ``eu não quero
isto'' e aí está aversão e, nunca nos libertamos através da rejeição de
alguma coisa, apenas suprimimos e quanto mais o fizermos, mais essas
raivas e esses medos nos fulminam por dentro, cortam, corroem-nos
completamente, comendo o fígado e o baço. Então é como guardar estes
demónios dentro de nós e depois perguntar porque somos miseráveis. O que
se tem que fazer é abrir a porta amplamente, porque eles também não
querem ficar cá dentro, deixem-nos sair, libertem-nos, dêem a amnistia a
todos os prisioneiros, um grande gesto, é um gesto de Amor, de Amor
Incondicional não é? Depois o que sai fora é do género, eu comparo-o com
um clister e o que sai é nojento, mas uma vez terminado, sentimo-nos
muito melhor. Depois, isto também nos transmite um sentimento da beleza
da vida. Apesar de toda a negatividade, maldade e egoísmo, da violência
que se ouve interminavelmente nos média, acerca do egoísmo e da
corrupção da humanidade, se provarmos isto a nós próprios, realiza-se
que o que jaz por debaixo de tudo é Amor... Amor Incondicional.

Eu não espero que acreditem em mim, mas isto é só uma sugestão, uma
forma de olhar, não tem nada a ver com gostar. É uma realidade
incondicional. É real, não é só um momento inspirado na minha vida, em
que de repente amo toda a gente. Não é isso.

Reconheçamos o Incondicionado, como na ``Terceira Nobre Verdade'' - a
cessação do sofrimento - realizar essa realidade. O que começa acaba.
Permitir que as coisas acabem. A morte é o fim de uma condição, da
cessação, da morte. E nós temos medo da morte. Portanto, muitos de vós
vêm dizendo que uma sensação de medo aparece na vossa prática à medida
que se aproximam do vazio. Experimentar o vazio e o ``não eu'',
revela-se bastante assustador. Porque não estamos preparados para isso
emocionalmente. Vivemos emocionalmente condicionados pelos extremos e,
então queremos alegria e tememos o sofrimento, queremos o sucesso, e a
euforia e tememos o falhanço e por vezes dizemos; ``não presto para
nada, a vida não vale nada, não tem sentido nenhum'', e por aí adiante.

Realmente, os hábitos emocionalmente extremistas e as emoções, estão
relacionados com este dualismo dos oito \emph{Dhammas} mundiais. Não
obstante, emocionalmente, não nos sabemos relacionar, não estamos
aprimorados. As nossas emoções, porque se enquadram no desconhecido e no
incerto, fazem-nos perder o sentido. Se nos definirmos como isto ou como
aquilo, ficamos com uma certa noção do que se é, mas se de repente
alguém nos tira as nossas identidades, não sabemos quem somos e isso é
muito assustador. Mesmo que nos assumamos bastante negativamente, tal
como ``Eu sou um falhanço absoluto sem esperança alguma'', ao menos
definimo-nos, ao menos sabemos quem somos. Contudo, se nos convencermos,
``eu sou uma forma superior da humanidade'', neste caso, tanto nos
podemos engrandecer como menosprezar, mas quando não há ``eu'' as coisas
mudam.

Lembro-me de há já alguns anos, em Wat Pah Pong, quando ainda estava com
Ajahn Chah, atravessei um período, em que senti que estava a morrer ali
- como uma voz interna - senti que estava a morrer naquele mosteiro... e
tinha este tipo incrível de grito cá dentro, a agudizar ``eu quero
viver'', esta voz interna ``eu estou a morrer, eu quero viver'' e de
repente indaguei-me e comecei a olhar à volta; `` isto é um sítio de
morte, este mosteiro, o Budismo fala só de morte e aniquilação'', e logo
surgiram todos os tipos de suspeitas e medos, ``talvez seja mesmo uma
religião diabólica ou exterminadora''. E, então este ``eu quero viver'',
era como um grito dentro de mim, mas apesar de todo este tremor, eu
consegui ter suficiente visão interior para na realidade não acreditar
nisso, para não vacilar, mas foi muito avassalador, muito forte. Ou
seja, quanto mais desenvolvo esta ``consciência presente''
(\emph{awareness)}, maior se torna a sua força e, o sentimento de mim
próprio como ego, começa a entrar em pânico, ``quem sou eu então? Eu vou
morrer, eu tenho medo da morte''. E então muitas vezes usamos as
palavras ``vida'' e ``morte'' como um par dualista e dizemos ``vida'' ou
``morte''. Mas, na forma de encarar as coisas no sentido budista, é
antes ``nascimento'' e ``morte'', nascimento e morte sim, vão juntos. A
palavra ``vida'' então, quererá significar o quê? Será vida eterna?
Poderia ser. No entanto, nessa minha experiência, não se trata de
condições eternas, condições que duram para sempre, mas subitamente,
toda a noção de morrer, de mim a morrer, morrer sozinho e a morte
estando morta, não existe mais morrer então, isto é o que Ajahn Chah
costumava sempre dizer, «morre antes de morrer» que em Tai diz-se
``\emph{dtai gon dtai}'', e Ajahn Chah estava constantemente a dizer
isto, ``morre antes de morrer''. E então esta morte do ego, este
poderoso sentimento da minha distinção de ``eu'' e ``meu'', à medida que
confio nesta ``Presença Consciente'', este poderoso sentimento de ``eu''
em manter o ``eu'', a minha distinção, a minha singularidade, de repente
morreu. Eu podia então deixar-me morrer se realizasse e visse isso, uma
vez que eu estava a morrer e a sofrer a todo o tempo, pois era algo que
não podia continuar.

Ora, o que acontece na realidade é que assumimos e actuamos como se
fossemos este ego o tempo inteiro; este ego sou eu mesmo e está comigo o
tempo todo e mesmo quando estou a dormir, ainda assim, sou Ajahn
Sumedho. E depois há a forma como as pessoas falam carregadas da mais
variada terminologia, em que assumem todas aquelas tendências latentes,
como raiva reprimida, contorcendo-se das profundezas e, talvez existam
até todo o tipo de energias negras cá dentro à espera de sair com medos
e tudo à volta. Como então acabar com isto? Na abordagem budista, o
Buddha encarou este problema através do reconhecer que condições dão
origem a outras condições -- ``génese dependente''. Ou seja, maneiras de
reconhecer a sua origem, como quando há condições para a raiva e a raiva
surge, sendo algo que está latente, algo que subjaz internamente, a não
ser que se assuma e identifique, com a realidade, do que está a
acontecer. E então eu acho que compreender ``génese dependente'', é um
modo muito mais proveitoso de olhar para a experiência, porque, por
exemplo, quando o Sol brilha, quando chove, quando somos elogiados,
insultados, quando se sente com saúde ou doente, as emoções estão sempre
de acordo com as condições.

A plena consciência da condição, não é a condição. Não é uma condição a
olhar para outra condição. Então esta ``Consciência presente'', está
também consciente da personalidade ou do momento em que as condições
para se ser uma pessoa ou uma personalidade surgem e eu sou assim; sou
pessoa feliz, pessoa infeliz, pessoa irritada, pessoa enlevada, pessoa
deprimida, pessoa ofendida, pessoa aborrecida, pessoa assustada. Mas com
a ``Presença Consciente'', à medida que se reconhece o que é ``Presença
Consciente'', chega-se à simplicidade última. Como o espaço, não há nada
de complicado a seu respeito. Não é nada que dependa de outras condições
para ser consciente. Ou seja, até no meio do inferno se pode ser
consciente. Então na Escola Mahāyāna, eles têm assim; ``Um Lótus que
floresce por entre o Inferno é indestrutível'', e depois há uma pintura
desta Flor de Lótus extremamente delicada e tudo em seu redor a arder
tremendamente. Esse é o meu Ícone, ser este Lótus é um símbolo de pureza
na religião oriental. ``Um Lótus que floresce por entre o Inferno é
indestrutível'', Lótus indestrutível, ``Presença Consciente''. E o
inferno contínua à volta, mas esta ``Presença Consciente'' é
indestrutível, por isso é que aqui se trata de reconhecê-la, realizá-la,
não de tentar possuí-la ou criá-la. É aqui, exactamente, que as palavras
nos podem desviar ou conduzir no mau sentido, porque concebemos e
definimos a ``consciência'' à nossa maneira. Queremos encontrar o que é,
definir e falar sobre isso. Mas nada mais é do que isto. Na realidade,
nada é, com excepção de que é apenas consciência no presente. Não se
trata de seleccionar, não é concentrarmo-nos ou ter preferência por umas
coisas sobre as outras e tentar controlar o que quer que seja, mas sim
de reconhecer esta situação pela qual nós estamos condicionados, como
voltar a nascer, a morrer, a ter medo, a gostar e a não gostar. Isto é o
nosso condicionamento cultural e as nossas personalidades que estão
construídas sobre estas ilusões e, toda esta linguagem que usamos,
reforça esta ilusão.

Portanto, estas são as três amarras que nos cegam o caminho da ``Quarta
Nobre Verdade'', que se baseia em \emph{Sammā-ditthi} (compreensão
correcta). Estas três amarras são então o ego (\emph{sakkāya-ditthi}),
do ponto de vista da personalidade que é um ponto de vista criado; o
processo cultural condicionador, e a dúvida que é o resultado do
pensamento. Se pensamos demasiado, duvidamos demasiado. Se pensar
demasiado, constantemente, um monge não pára de questionar, inquieto. Na
Universidade encontra-se muita gente a duvidar, porque pensa. Em
Berkeley, havia tantas pessoas cépticas, cínicas. Reparem, quando se
pensa muito, quando se pensa sobre nós próprios, sobre a nossa prática,
sobre o Budismo, sobre outra coisa qualquer, começa-se a duvidar sobre
isso e, então, este pensar cria esta dúvida (\emph{vicikichchā}). É por
isso que este realizar, reconhecer a ``Presença Consciente''
(\emph{Awareness)}, é o alvo da ``Quarta Nobre Verdade''. É um meio
inteligente de usar algo tão vulgar como o sofrimento, reconhecer as
causas do apego, do desejo, largar (abrir mão) e então realizar a
cessação, porque o que surge tem um fim. Fica-se então desperto, a
``Presença Consciente'' liga-nos à ausência, à cessação de algo. E
quando há cessação na consciência, não é morte, eu ainda respiro, ainda
me sinto vivo, mas o sofrimento acaba e então quando se diz, que toda a
fenomenologia condicional é \emph{Dukkha} (insatisfação - sofrimento),
não se trata de desvalorizar a beleza, nem a bondade, nem a graciosidade
do mundo condicionado, mas trata-se antes de realçar a sua natureza
condicional, para que não haja apego a qualquer condição e manter-se
fora da ignorância.

Parte de nós falta, não estamos completos, há uma falha, falta qualquer
coisa e enquanto a nossa identidade estiver nesse nível de fenomenologia
condicional, não importa o que façamos, vamos sentir que há algo que
falha na nossa vida. E então podemos ir à procura de alguém, ou ir à
procura de poder ou colocação profissional, mas não importa o quanto
procuremos lá fora, haverá sempre um sentimento de falta, de
imperfeição... de falha. Porque a falha consiste em não sermos
totalmente conscientes. Cria-se esta divisão, separação e até a outra
pessoa pode fazer-nos sentir completos, mas quando se vai embora,
ficamos novamente incompletos. Portanto é só uma completude ilusória.

O Buddha apontou para a Realidade Imortal em vez de nos agarrarmos só às
condições boas, ao que se torna bom, ou a viver a vida ignorando a parte
má e tomando só atenção à boa. Mas aprendemos de ambas, ambas têm valor
idêntico, o bom e o mau, o certo e o errado. Porque todas as condições
terão um fim. E então a cessação é paz, libertação. E isso é Amor.

O Amor subjaz a tudo neste planeta, neste Universo em que vivemos.
Seremos só máquinas gulosas?.. não, nós amamos, sentimos Amor, sentimos
algo mais profundo do que somente preferências pessoais. Para lá desse
tipo de egoísmo e cegueira do ser humano, existe também um sentir para a
Realidade Imortal, pelo espiritual, pela verdade última. E disso que a
religião trata. Todas as religiões existentes no mundo têm diferentes
tipos de reconhecimento e formas de o proclamarem, mesmo usando símbolos
diferentes, mas é para onde todas as religiões apontam, porque isto é
algo que se encontra na condição humana. Então, em termos de prática de
``\emph{Metta}'' apercebemo-nos de que, quanto mais entramos em contacto
com este Amor incondicional, mais serenamos e confiamos. Esta é uma das
verdadeiras formas com que podemos ajudar todos os seres sencientes
neste preciso momento. A um nível pessoal podem pensar, «como é que eu
vou ajudar? Toda aquela gente a morrer no Iraque e todo o tipo de
violência que vai por aí fora?». Porém, quanto mais os seres humanos
reconhecerem, acordarem para a verdade última, então isto será para o
benefício de todos, porque está tudo interligado; não é uma questão de
ser só eu a sentir-me liberto do sofrimento e ``que o resto das pessoas
vá para o inferno'', porque já não sou eu. Já não sou eu como uma
entidade isolada, com esta forma humana. Então, compaixão,
\emph{Karunā-Muditā-Upekkhā}, os \emph{Brahma-vihāras}, vêm daí. Estas
são as respostas ao sofrimento e à beleza que experimentamos através da
consciência, como entidades independentes do condicionamento pessoal que
nos cega para a verdade última.

... ora então ofereço isto como reflexão...

\bigskip

{\raggedleft\itshape
  Tradução de Dhammiko Bhikkhu
\par}
